\vbox{
\setlinear
\setcoordinatesystem units <2pt,2pt>
$$
%axes
\putrule from -100 5 to 100 5
\plot 97 6 100 5 97 4 /
\plot -1 12 0 15 1 12 /
\put {$r$} at 100 2
\put {$z$} at -4 15
\put {$C_0$} at -30 8
\put {$\Omega_-$} at 45 -25
\put {$\Omega_+$} at 45 13
\putrule from -60 5 to -60 13
\plot -59 10 -60 13 -61 10 /
\put {\bh{n}} at -65 13
\setdashpattern <2pt,3pt,7pt,3pt>
\putrule from 0 15 to 0 -65
%Bubble1
\setsolid
\circulararc 360 degrees from -10 -15 center at 0 -15
\plot 0 -15 7 -22 /
\put {$r_1$} at 4 -14
\betweenarrows {$\gamma_1$} from -20 -15 to -20 5
\setdashpattern <4pt,2pt>
\putrule from -23 -15 to 0 -15
\put {$C_1$} at -13 -11
%Bubble1
\setsolid
\circulararc 360 degrees from -15 -45 center at 0 -45
\plot 0 -45 11 -55 /
\put {$r_2$} at 7 -47
\betweenarrows {$\gamma_2$} from -47 -45 to -47 5
\setdashpattern <4pt,2pt>
\putrule from -50 -45 to 0 -45
\put {$C_2$} at -18 -41
$$
{\baselineskip 15pt
\medrm
\parindent 0pt
Figure 5.1. The initial position of bubbles $C_1$ and $C_2$ below the free-surface,
$C_0$. In most calculations presented, the free-surface is not flat as depicted,
but takes the shape of a bubble for which the film cap has just ruptured,
together with a meniscus extending to infinity (see also figure 5.2).
}
}
