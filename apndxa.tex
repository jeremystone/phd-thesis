\c {\bigrm Appendix A: Derivation of integral equations}
\c {\bigrm for chapter 2.}
\vskip 15pt
\hbox{\bf (i) Source formulation.}
\vskip 5pt
For the source formulation, we require the normal derivative of the
potential which is given for all $\bd{x}$ by the expression
$$\phi(\bd{x})=k+{1\over 2\pi}\int_C\sigma(\bd{x}')
\log{1\over|\bd{x}'-\bd{x}|}ds'.\eqno(A1)$$
Writing $\phi=\Re[W]$, where $W$ is the corresponding complex
potential, we have
$$\eqalign{W(z)&=k+{1\over 2\pi}\int_C \sigma(z'(s))
\log{1\over z'(s)-z}ds\cr
&=k+{1\over 2\pi}\int_C {\sigma(z')\over\tau(z')}\log{1\over z'-z}dz',}$$
where $s$ is the arc-length and
$\tau(z')=dz'/ds$ is the unit tangent at $z'\in C$.
The potential defined above is continuous across $C$ (see \S 2.2.3) and
so there is no difficulty evaluating it.

Differentiating and evaluating at $z^\star\in\Omega_\pm$, we get
$${dW\over dz^\star}={1\over 2\pi}\int_C {\sigma(z')\over\tau(z')}
{dz'\over z'-z^\star}.$$
This expression is not continuous across $C$, and so we write it in
terms of a continuous Cauchy principal value integral and a
discontinuous term, thus enabling simple evaluation in the limiting
cases as $z^\star\rightarrow z\in C$.

Writing this last expression in the form
$${dW\over dz^\star}={1\over 2\pi}\int_C \left({\sigma(z')\over\tau(z')}-
{\sigma(z)\over\tau(z)}\right){dz'\over z'-z^\star}+
{1\over 2\pi}{\sigma(z)\over\tau(z)}\int_C {dz'\over z'-z^\star},$$
and using Cauchy's theorem for the second integral, we get
$${dW\over dz^\star}={1\over 2\pi}\int_C \left({\sigma(z')\over\tau(z')}-
{\sigma(z)\over\tau(z)}\right){dz'\over z'-z^\star}+
\cases{i{\sigma(z)/\tau(z)},&$z^\star\in \Omega_+,$\cr
0,&$z^\star\in \Omega_-.$}$$
If $\sigma/\tau$ is piecewise smooth on $C$ then it is
H\"older continuous so that the integral term is continuous as 
$z^\star$ crosses $C$. We may thus trivially take the limit as
$z^\star\rightarrow z$. The corresponding complex velocities we
denote by $Q_\pm$ for $z^\star\in\Omega_\pm$.

To get the normal (and tangential) derivatives of the real potential,
$\phi$, note that if $n$ is the normal pointing into the gaseous phase
and contours are followed in the usual positive sense,
then $n=i\tau$ and so, resolving in normal and tangential
directions, we may write, formally
$$Q_\pm=\left(i{\partial\phi\over\partial n}_\pm+
{\partial\phi\over\partial s}_\pm\right)\tau=
\left({\partial\phi\over\partial n}_\pm-
i{\partial\phi\over\partial s}_\pm\right)n.$$
Hence clearly
$${\partial\phi\over\partial n}_\pm=\Re[Q^*_\pm n],
\quad\hbox{and}\quad
{\partial\phi\over\partial s}_\pm=\Re[Q^*_\pm \tau],\eqno (A2)$$
where an asterisk denotes complex conjugation.
This gives the normal derivative on the fluid side of the bubble as
$${\partial\phi\over\partial n}={1\over 2\pi}\int_C
\bigl(\sigma(\bd{x}')\bh{n}(\bd{x})-
\sigma(\bd{x})\bh{n}(\bd{x}')\bigr)\cdot
{\bd{x}'-\bd{x} \over |\bd{x}'-\bd{x}|^2}ds',
\quad\bd{x}\in C,\eqno (A3)$$
and for the tangential derivative,
$${\partial\phi\over\partial s}={1\over 2\pi}\int_C
\bigl(\sigma(\bd{x}')\bh{t}(\bd{x})-
\sigma(\bd{x})\bh{t}(\bd{x}')\bigr)\cdot
{\bd{x}'-\bd{x}\over|\bd{x}'-\bd{x}|^2}ds',
\quad\bd{x}\in C.\eqno (A4)$$
\vskip 15pt
\hbox{\bf (ii) Dipole formulation.}
\vskip 5pt
For the dipole formulation, the potential is given, for all $\bd{x}$, by
$$\eqalign{\phi(\bd{x})&=k-{1\over 2\pi}\int_C
\mu(\bd{x}'){\partial\over\partial n'}\log{1\over|\bd{x}'-
\bd{x}|}ds'\cr
&=k+{1\over 2\pi}\int_C \mu(\bd{x}')\bh{n}(\bd{x}')\cdot
{\bd{x}'-\bd{x}\over|\bd{x}'-\bd{x}|^2}ds'.}$$
Recall from the derivation of the dipole formulation ({\S 2.2.3}) that
$\phi$ is not continuous, so we aim to write it in a more convenient form.
So that the normal and tangential velocities may be derived
easily, we first write it in complex form for $z^\star\in\Omega_\pm$, $z\in C$
$$\eqalign{W(z^\star)&=k+{1\over 2\pi}\int_C \mu\bigl(z'(s)\bigr)
{n\bigl(z'(s)\bigr)\over z'(s)-z^\star}ds\cr
&=k-{1\over 2\pi i}\int_C \mu(z'){dz'\over z'-z^\star}.}$$
This may be written as
$$W(z^\star)=k-{1\over 2\pi i}\int_C \bigl(\mu(z')-\mu(z)\bigr)
{dz'\over z'-z^\star}-
\cases{\mu(z),&$z^\star\in\Omega_+,$\cr 0,&$z^\star\in\Omega_-.$}$$
Taking the limit as $z^\star\rightarrow z$ and evaluating the real part
for the case $z^\star\in\Omega_-$ gives
$$\phi(\bd{x})=k+{1\over 2\pi}\int_C
\bigl(\mu(\bd{x}')-\mu(\bd{x})\bigr)\bh{n}(\bd{x}')\cdot
{\bd{x}'-\bd{x} \over |\bd{x}'-\bd{x}|^2}ds'.
\eqno (A5)$$
To get the derivative, and hence the complex velocities
for $z^\star\in\Omega_\pm$, note that
$${dW\over dz^\star}=-{1\over 2\pi i}\int_C \mu(z')
{\partial\over\partial z^\star}
\left({1\over z'-z^\star} \right)dz',$$
where this integral is interpreted as a finite-part integral (see for example
Martin and Rizzo, 1989).
Integrating by parts over the closed contour, $C$, gives
$${dW\over dz^\star}=-{1\over 2\pi i}\int_C {d\mu\over dz'}
{dz'\over z'-z^\star}.$$
Taking $z\in C$, this may be written as
$${dW\over dz^\star}=-{1\over 2\pi i}\int_C
\left({d\mu\over ds}(z'){1\over\tau(z')}-
{d\mu\over ds}(z){1\over\tau(z)}\right){dz'\over z'-z^\star}-
\cases{{d\mu/dz'}(z),&$z^\star\in \Omega_+,$\cr
0,&$z^\star\in \Omega_-,$}$$
which, in the limit $z^\star\rightarrow z$ and on use of (A2), immediately
gives the velocities on the fluid side of the bubble as
$${\partial\phi\over\partial n}=-{1\over 2\pi}\int_C
\left({d\mu\over ds}(\bd{x}')\bh{t}(\bd{x})-
{d\mu\over ds}(\bd{x})\bh{t}(\bd{x}')\right)\cdot
{\bd{x}'-\bd{x}\over|\bd{x}'-\bd{x}|^2}ds',
\quad\bd{x}\in C,\eqno (A6)$$
and
$${\partial\phi\over\partial s}={1\over 2\pi}\int_C
\left({d\mu\over ds}(\bd{x}')\bh{n}(\bd{x})-
{d\mu\over ds}(\bd{x})\bh{n}(\bd{x}')\right)\cdot
{\bd{x}'-\bd{x}\over|\bd{x}'-\bd{x}|^2}ds',
\quad\bd{x}\in C.\eqno (A7)$$
\vskip 15pt
\hbox{\bf (iii) Vortex formulation.}
\vskip 5pt
For the vortex formulation, the potential is precisely that for the
source formulation multiplied by $i$. Hence from the results of (i),
the velocities on the fluid side of the bubble are
$${\partial\phi\over\partial n}=-{1\over 2\pi}\int_C
\bigl(\Gamma(\bd{x}')\bh{t}(\bd{x})-
\Gamma(\bd{x})\bh{t}(\bd{x}')\bigr)\cdot
{\bd{x}'-\bd{x}\over|\bd{x}'-\bd{x}|^2}ds',
\quad\bd{x}\in C,\eqno (A8)$$
and
$${\partial\phi\over\partial s}={1\over 2\pi}\int_C
\bigl(\Gamma(\bd{x}')\bh{n}(\bd{x})-
\Gamma(\bd{x})\bh{n}(\bd{x}')\bigr)\cdot
{\bd{x}'-\bd{x}\over|\bd{x}'-\bd{x}|^2}ds',
\quad\bd{x}\in C.\eqno (A9)$$
It can be seen from the form of (A8,9) that $\Gamma$ here corresponds to
$d\mu/ds$ in the dipole formulation.





