\vbox{
\c{\bigrmb Chapter 5.}
\vskip 1cm
\c{\bigrm BURSTING BUBBLES.}
\vskip 15pt
\hbox{\bf 5.1 Introduction.}
\vskip 5pt
} 
\c{\it 5.1.1 Summary.}
\vskip 5pt
When a small bubble bursts at a gas/liquid interface,
the fluid motion subsequent to film rupture is
driven by surface tension. The 
high curvature at the rim of intersection of the bubble with the
free-surface creates a large pressure difference
which rapidly pulls it outward and downward. 
At the same time, the base is drawn inwards as
the cavity wall straightens. Eventually, the converging flow near the
base causes a liquid jet to be thrown upward at high speed,
often several metres per second. 
The jet, as it rises, may break up into a number of drops. The corresponding
downward jet may be expected to advect vorticity from the boundary
layer around the bubble cavity into the region below.
Various aspects of this motion have been studied experimentally
by a number of researchers --- Kientzler, Arons, Blanchard and Woodcock (1954),
Newitt, Dombrowski and Knelman (1954), Garner, Ellis and Lacey (1954)
and MacIntyre (1972).
A numerical model based on an inviscid boundary integral 
scheme is used in this chapter to model these bursting processes. 
Viscous effects are included in an attempt to model the 
boundary layer and the subsequent distribution of vorticity in the
downward jet region.
\vskip 15pt
\c {\it 5.1.2 Background.}
\vskip 5pt
The first event occurring when a bubble bursts is the rupture of the
thin lamella that separates the bubble from the atmosphere above
the free-surface.
There is a substantial body of literature relating to films
and film drainage. The assumptions
that must be made when dealing with thin films are completely 
different from those used below for a bubble in a high Reynolds 
number flow. Here, viscosity plays an important role in 
determining the drainage and stability properties of lamellae.
Film stability has three main causes (see for example 
Bikerman, 1973). 
The first of these is
surface viscosity, whereby the presence of surfactants causes the
film surfaces to have a higher viscosity than the central parts.
This viscosity can be non-Newtonian and may also give the surfaces of the 
film crust-like properties which can, like a solid,
resist stresses up to a certain yield point. Another stability mechanism
is the Marangoni effect --- a kind of restoring property for
contaminated surfaces. If a film thins over a small region, the 
surface area will increase locally thus decreasing the surfactant
concentration there. The resulting surface tension gradient can pull the 
surrounding surface towards the thin region. The fluid beneath the surface
flows with it because of viscosity, thus thickening the film.
Repulsion of the opposite sides of the surface 
due to electrical dipoles is the third cause of stability.
Acting against these electrostatic forces there are 
London-van der Waals forces that tend to pull the two sides of the
film together, thus increasing the tendency to rupture. However
both of these forces act over a very short range (of the
order of $100nm$) and
have an effect in only the latter stages of thinning.
Early on, while the film is thicker,
there are two main mechanisms for drainage.
Obviously, one is gravity.
The second is due to Plateau borders 
(see Bikerman, 1973). These are 
regions of high surface curvature, 
particularly in foams, causing low pressures at the `corners' of a bubble. 
This mechanism is not so important when applied to a single 
bubble, where the curvature is almost
constant before it bursts.

With respect to bubbles approaching a free-surface, 
experimental and theoretical aspects are considered
in a number of papers.
Allan, Charles and Mason (1961) measured the thickness of the film
above a bubble. They found that the thinnest part of the film
moves out from the centre to a circular rim.
The radius of this rim, $c$, which remains almost constant, 
was found to agree well with the expression 
$c=\sqrt{E_o/6}$, based on a balance of surface tension and buoyancy
forces, where $E_o$ is the E\"otvos number.
The thinning rates found also match theoretical
estimates based on a pair of parallel discs or on the 
film between a rigid sphere and a free-surface. They also observed 
that the presence of
surfactants markedly reduced the thinning rates.

Hahn, Chen and Slattery (1985) used a lubrication theory approximation
including the effects of London-van der Waals forces
to model the draining of a film above a bubble. 
In this way they were able to obtain
estimates for the rupture time, taken as the time for the rim 
thickness to become zero. 
However, they mentioned that the geometry used, which
prevents any asymmetric instabilities, and the assumption that there
is no tangential motion on the surfaces due to the presence of
surfactants, means that the model gives only an upper bound for 
the rupture time. Electrostatic
repulsion was also neglected.

The effect of the approach speed of a bubble
towards a free-surface was investigated by Kirkpatrick and Lockett (1974).
They found that for larger velocities (above about $1cm/sec$
for $2.5mm$ radius bubbles in water),
the thinning process
does not have time to complete before the bubble is decelerated to
rest by surface tension, which may then push it back into the 
fluid. In contrast, a bubble that is released just below the surface
and so does not have time to accelerate to its terminal velocity
is found to burst almost immediately, without bouncing. A simple mathematical 
model, based on the fact that the rate of thinning 
is inversely related to the area of the film was developed, 
leading to the same general conclusion. Minor impurities in the water
were found to be insignificant, but with a larger amount of surfactant
($0.6$ molar sodium chloride)
the coalescence times became longer, with
little to distinguish between low and high approach velocities.

A film is said to be
unstable (Scheludko, 1962), if there is some critical thickness at which point
small perturbations on the surface will
grow causing a hole to appear (Vrij, 1966) which,
provided that it is big enough (Taylor and Michael, 1973), then expands due to
surface tension, thus rupturing the film. 
One of the earliest investigations of film rupturing was by 
Lord Rayleigh (1891), who applied high-speed photography to 
view the bursting of soap films. Newitt et al (1954) show a 
photograph which exhibits clearly the breakup of a liquid film
into a `lace-like' structure of liquid threads, each of which breaks
up into a number of tiny droplets. 
The fluid in the film gathers up into a toroidal rim which 
often breaks irregularly into a number of threads (Rayleigh, 1891;
Ranz, 1959) which then break into more tiny droplets. 
This rim itself is expanding too rapidly to
be broken up by capillary ripples, as in the case of the jet (MacIntyre, 1972).
Instead,
this breakup is due to the effects of variations of surface tension and film 
thickness and, in the case of bubble burst, this is coupled with the 
effect of turbulence from the escaping air.
The droplets formed in the film are much smaller than the
jet drops mentioned below (Newitt et al, 1954; Garner et al 1954) and
are projected sideways by the expansion of the bursting film and upwards
by the rush of gas as the pressure in the bubble is released.
Droplet speeds may be as high as $10m\,s^{-1}$ (Resch et al, 1986).
By exploiting a simple
energy balance argument, Culick (1960) showed that the speed 
of the retreat of the film is given approximately by 
$v=\sqrt{2\sigma/h\rho}$ where $h$ is the film thickness,
usually just a few microns.
By comparing the portion of the energy used in the inelastic 
acceleration of fluid entering the rim with the viscous energy dissipation,
he showed that the length of fluid being
accelerated is about $h\mu v/\sigma$, which is typically of the 
order of the film thickness.

The subsequent bubble bursting motion which takes place on a longer
time-scale --- a few milliseconds as opposed to
times of the order of $100\mu s$ for the film rupture (Resch et al, 1986)
--- is better suited to study by high-speed photographic methods. 
After film rupture, 
what remains of the toroidal film rim was observed by MacIntyre (1972)
to follow a ripple down the sides of the bubble. This flow converges at the 
base of the bubble and a high-speed jet is thrown upwards.
By using dyes in the fluid, it was shown by MacIntyre
that the liquid in the jet originates in a thin layer surrounding the 
bubble crater.
Very little mixing of the dye with surrounding fluid during jet formation 
indicated that the flow at the base of the jet is irrotational. MacIntyre was also able
to observe the corresponding downward jet of fluid that must occur on
grounds of momentum conservation by its effect of propelling tiny bubbles
into the fluid beneath the bursting bubble.
Kientzler et al (1954) also show clear pictures of the complete bursting process
from immediately after the film rupture up to and including the formation and breakup
of the high-speed liquid jet. 

The fact that a narrow jet breaks up into a number of drops is well known.
Rayleigh (1878), by considering the change in surface area
of a jet subjected to a symmetric disturbance, showed it to be stable provided
that the wavelength of the perturbation is shorter than 
the circumference of the
jet. The  most destructive mode was shown to have wavelength
$\lambda\approx 4.508\times 2a$ where $a$ is the jet radius.

The evidence above relating to the origin of the material in the jet 
is significant in the study of cell damage by bubbles, since 
it has been reported by Blanchard and Syzdek (1972) that bacteria 
tend to become adsorbed onto bubble surfaces. Similar results have been
found for insect cells by Bavarian, Fan and Chalmers (1991). This 
was followed up by Chalmers and Bavarian (1991) who claimed
that the hydrodynamic forces due to the shear in the boundary layer
around the walls of the bubble cavity
are sufficiently large to kill cells. They
also postulated a second mechanism based on Culick's (1960)
finding of the small length
of fluid accelerating as the rupturing film recedes.
Cells may be struck by the advancing toroidal rim as they sit on
a stationary part of the bursting lamella. Experiments
by Kowalski (1991) also highlight this as a likely cause of
cell damage.
\vskip 15pt
\c {\it 5.1.3 Aims.}
\vskip 5pt
The primary aim of this chapter is to produce a numerical
model of a bursting bubble, which agrees with the experimental 
results indicated above. In particular, the model should predict 
the high-speed liquid jet, together with
the corresponding downward jet.
Calculations can then be made of the stresses imposed on a particle
in the vicinity of a bursting bubble.
The effects of viscosity, leading to a boundary layer and high shear rates, are 
also of interest as they may further increase the damaging potential of the
burst. An estimate of the contribution to the stress on a cell
due to shear in the downward jet region is also sought. Such information
may provide useful evidence for or against particular
proposed cell damage mechanisms. 

We do not address a numerical study
of the film rupturing process itself, nor do we discuss here
the stresses on a particle resting on the film as it breaks:
this will be left for the future. 
We will use the term `burst' to refer to the events that occur
after the completion of the film rupturing process.
Although we neglect the dynamics of the film rupture it is 
important to consider the effect that the film has on the position of
a bubble at a free-surface.
Since it is clear that the burst will be driven by the potential energy
stored in the initial bubble configuration, both in terms of
surface energy and buoyancy,
the starting shape and height of the bubble with respect to the free-surface
is critical in determining the consequent flow.
\vskip 15pt
\hbox{\bf 5.2 Problem statement.}
\vskip 15pt
\c{\it 5.2.1 Inviscid formulation.}
\vskip 5pt
The photographic evidence of Kientzler et al (1954) indicates that
Reynolds numbers for a bubble bursting, 
from the time just after film rupture to the 
rise of the jet, are of the order of $1000$.
We may therefore assume that, up until the formation of the jet,
any vorticity is limited to thin boundary
layers around the air/water interfaces, so that the velocity
distribution may be reasonably represented by
potential flow, $\bd{u}=\nabla\phi$.
Again, as we are only interested in bubbles in close
proximity to the free-surface, we assume that any rising bubbles
remain of fixed volume.

The solution domain, $\Omega_-$
(see figure 5.1), is defined to be the 
semi-infinite region bounded above by the free-surface, $C_0$, and
internally by the bubbles \hbox{$C_m$, $(m=1,\ldots,M)$}.
The free-surface coincides with the $(x,y)$-plane at infinity. 

Formally, this problem is a simple extension of that in section 4.3
with a free-surface included.
Here bubble $m$ is situated on the $z-$axis at a distance 
$\gamma_m(>0)$ below $C_0$.
As the bursting phenomenon is dominated by surface tension forces,
especially for the smaller bubbles, we scale times with respect to 
$(\rho a^3/\sigma)^{1/2}$
and pressures by the factor $\sigma/a$, $\sigma$ being the surface tension
and $a$ the radius of bubble $1$.

The code was written to allow bubbles to either rise up to an
initially flat free-surface and burst as they reach it,
or so that a bubble will have just burst
at $t=0^-$, when the calculation is started. The latter option
allows us to include more physical reality into the problem 
without worrying too much about surface drainage. The method
for finding the initial surface shape in this case is discussed in section 5.2.2.

The E\"otvos number, $E_o={4\rho ga^2/\sigma}$, again
enters as an explicit parameter in the boundary conditions. It
measures the bubble size and represents the 
square of the ratio of the two time-scales associated with the
collapse, due to surface tension,
of a spherical cavity whose contents remain fixed
at the ambient pressure of the fluid and with the rise of a bubble due
to gravity. 
During bubble burst, this parameter therefore measures
the relative importance of bubble rise and bubble collapse.
For the smallest bubbles, the effect of buoyancy
will clearly be secondary to surface tension forces.
\vskip 10pt
\input fig51
\vskip 10pt
If we again define 
$$f\Big|_{C_m}=\phi\Big|_{C_m}+k_m,\eqno(5.2.1)$$
with $k_m$ given by (4.3.5) for $m=0,\ldots,M$
(note that $k_0=0$ for all times),
the dynamic condition for updating $f$ on the
boundaries is 
$${Df\over Dt}\Big|_{C_m}={1\over 2}|\bd{u}|^{2}-
{E_o\over 4}(z+\gamma_m)+
\kappa-\kappa_m, \quad m=0,\ldots,M.\eqno (5.2.2)$$

We again use the constraint equations (4.3.8) to fix the
bubble volumes as they rise.
\vskip 15pt
\c {\it 5.2.2 Initial configuration.}
\nobreak
\vskip 5pt
\nobreak
Certain questions have to be addressed in order to produce a good 
model of the bursting process.
As the film above the bubble will be very thin, we
assume for the purposes of this chapter that the film itself will
have no significant effect on the subsequent motion.
The rupture of the film may have dire consequences for cells
adsorbed onto the upper surface of the bubble, but
this is neglected in this investigation.
 
If we can ignore the film, then
when a bubble is `burst' numerically, a decision 
still has to be made as
to the amount of film above the bubble to remove before reconnecting
the bubble cavity to the free-surface. This will invariably be
greater than the total amount of fluid in an actual
film because of the difficulty in calculating
the bubble and surface motion with
the bubble much closer to the surface 
than the length of a single boundary integral element (see \S 5.3).
However, numerical experimentation has shown that, provided not too
much fluid above the bubble is thrown away, this
has little effect on the subsequent bursting motion.
It would, in any case, be inappropriate to assume that the 
irrotational model can accurately predict the fluid flow
in the film when it becomes thin.
In the case where surfactants are present various physico-chemical phenomena
such as surface viscosity may become important so that stress-free
interfacial boundary conditions become invalid.
It is thus hard to see how the rupture time 
can be determined based on the calculated thickness of the
film other than by matching a modified lubrication
layer for the film onto the inviscid model.
The rupture-time calculations of Hahn et al (1985) 
are likewise difficult to implement, firstly because they are 
inaccurate as indicated in section 5.1.2, but also because of the 
problem of identifying with any certainty
their initial time, when the draining rate 
of the film above the bubble is independent of radial position.
It is suggested, however, that this time 
will be close to the time that the bubble comes to rest at the surface.

It seems clear, that one of the most important factors determining 
the motion following film rupture, in terms of the energy released, 
is the height
of the top of the bubble above the equilibrium free-surface position. 
The experimental study of Kirkpatrick and Lockett (1974)
indicates that bubbles moving at speeds approaching terminal
velocity, upon reaching the free-surface, tend to 
bounce a couple of times before bursting.
It has also been observed (see for example Newitt et al 1954 or 
Allan et al, 1961) that a bubble may rest for a short time at the  
free-surface before bursting.
In any case, we assume that the thinning of the film
and the instabilities which
eventually rupture it when it reaches a critical
thickness are to some extent asynchronous with the bouncing 
of the bubble and that,
on average, 
a bubble will burst at its static equilibrium position when the 
buoyancy force is equal to the downward component of surface tension.
For a spherical bubble in a pure liquid, 
the non-dimensional height above the surface of the top of the
bubble is approximately
$$h=1-\sqrt{1-E_o/6}.\eqno(5.2.3)$$
This is equivalent to the formula given by
Allan et al (1961) for the radius
of the rim of intersection of the free-surface
and the bubble, in terms of the bubble radius.
Here, we use a pressure balance to find the equilibrium bubble 
and meniscus shapes, employing equation (5.2.3) as a first approximation.
If $\sigma_{ij}$ are the surface tensions of the interfaces $(i,j)$,
(see figure 5.2) then 
force balances in the $r$ and $z$ directions give us respectively,
$$\eqalignno{
\sigma_{13}\cos\theta_c=&\sigma_{12}\cos\phi_c+
\sigma_{23}\cos\psi_c-{K\over r_c}&(5.2.4)\cr
\noalign{\hbox{and}}
\sigma_{13}\sin\theta_c=&\sigma_{12}\sin\phi_c+\sigma_{23}\sin\psi_c,
&(5.2.5)}$$
where $K$ is the line tension from the intersection of the three interfaces.
A subscript $c$ denotes evaluation at this contact line.
In general, the value of $K$ depends on the geometric configuration 
of the interfaces 
as well as the physical properties of the fluids 
(see Ivanov, Kralchevsky and Nikolov, 1986).
\vskip 10pt
\input fig52
\vskip 10pt
It is convenient at this point to follow Ivanov et al (1986) and
rescale lengths with respect to the 
mean radius of curvature, $b$, of the lowest point of the bubble.
Correspondingly scaled quantities will be indicated by an overbar.
We also introduce the quantity $\beta=\rho g b^2/\sigma$ (similar to the 
E\"otvos number, $E_o$), where $\sigma=\sigma_{12}=\sigma_{23}$.
If we write $\bar{\tau}=\sigma_{13}/\sigma$ and $\bar{K}=K/b\sigma$,  
the experimental evidence of Ivanov et al (1986) suggests that 
$$|\bar{\tau}-2|\ll 1,\quad\hbox{and}\quad|\bar{K}|\ll 1.\eqno(5.2.6)$$
As angles $\theta_c$, $\phi_c$ and $\psi_c$ must be acute, (5.2.4) and (5.2.5) give
approximate solutions, 
$\theta_c=(\phi_c+\psi_c)/2$ and
$\phi_c=\psi_c$, namely $\theta_c=\phi_c=\psi_c$.

Let $(\bar{R},\bar{Z})$ be the rectangular coordinates of a point
in a half-plane containing the axis of symmetry, centred on the lowest
point of the bubble crater.
A simple consideration of the pressure jump across the interface 
$(1,2)$ shows that
the shape of the bubble cavity is described by the differential system
$$\eqalignno{
{d\bar{R}\over d\phi}=
&{\bar{R}\cos\phi\over(2+\beta \bar{Z})\bar{R}-\sin\phi},&(5.2.7)\cr
{d\bar{Z}\over d\phi}=&-\tan\phi{d\bar{R}\over d\phi},&(5.2.8)}$$
with boundary conditions
$$\bar{R}=\bar{Z}=0,\quad\hbox{at}\quad\phi=\pi.\eqno(5.2.9)$$
We denote the values of $\bar{R}$ and $\bar{Z}$ at $\phi=\phi_c$ 
by $\bar{Z}_c$ 
and $\bar{R}_c$ respectively.

Similarly, the meniscus (interface $(2,3)$) has shape 
$\bar{z}=\bar{h}(\bar{r})$
which is governed by the second-order
equation
$$\bar{h}''=\left(-{\bar{h}'\over r}+\beta\bar{h}(1+\bar{h}'^2)^{1/2}\right)
(1+\bar{h}'^2),\eqno(5.2.10)$$
with the two-point boundary conditions
$$\bar{h}'(\bar{R}_c)=-\tan\psi_c,\quad\hbox{and}\quad\bar{h}(\infty)=0.
\eqno(5.2.11)$$
Aymptotic solutions for both the bubble cavity and meniscus shapes 
exist for small values of $\beta$ and $\bar{R}_c$ respectively 
(see Ivanov et al, 1986
and Lo, 1983). However, for larger bubbles these 
expressions are less accurate and 
so a numerical solution is used here.

The system given by (5.2.10) and (5.2.11)
can be solved using a technique similar to that of Princen
(1963), by choosing a value for $\bar{h}_c\equiv\bar{h}(\bar{R}_c)$ and  
performing a simple bisection search
depending on whether the solution tends to infinity or dips below 
the $z=0$ plane.
It is not possible for the interface shape in static
equilibrium to have a maximum above or a minimum below the 
$(x,y)$-plane: the curvature at such a stationary point 
would be inconsistent with the hydrostatic pressure difference.
Since the meniscus meets the bubble at $\bar{r}=\bar{R}_c$, 
we have that the bubble
position in the fluid is given by
$$\bar{z}=\bar{Z}(\phi)+\bar{h}_c-\bar{Z}_c.\eqno(5.2.12)$$

We further assume the film separating the bubble
from the atmosphere to be very
thin and thus gravity forces negligible compared to the effect 
of pressure so that
the film may be taken to be spherical with radius $\bar{c}$.
Using this to equate the pressure in the bubble given in terms of the 
pressure jump across the spherical
film, with that given by the hydrostatic pressure at the bottom
plus a corresponding jump due to 
surface tension there, gives 
$${4\over\bar{c}}=2+\beta(\bar{Z}_c-\bar{h}_c).\eqno(5.2.13)$$
Since the dome is spherical, $\bar{R}_c=\bar{c}\sin\theta_c$ and,
bearing in mind that the critical angles are assumed equal,
(5.2.13) can be written as
$$\bar{R}_c\left(1+{\beta\over 2}(\bar{Z}_c-\bar{h}_c)\right)=
2\sin\phi_c.\eqno(5.2.14)$$

Equation (5.2.14) fixes the height of the bubble in the 
fluid, but we also need an equation to
set the volume.
The volume of the bubble with the new scalings is
$$\bar{V}=4\pi(E_o/4\beta)^{3/2}/3.\eqno(5.2.15)$$
Note that the volume, $\bar{V}$, is calculated by adding the 
volume for the region of the bubble below the ring of
intersection, $\bar{V}_{12}$, which can be found numerically, to the 
volume of the spherical dome with radius 
given by equation (5.2.13), thus
$$\bar{V}=\bar{V}_{12}+\pi\left[{2\over3}\big(\bar{c}^3-
(\bar{c}^2-\bar{R}_c^2)^{3/2}\big)-
\bar{R}_c^2(\bar{c}^2-\bar{R}_c^2)^{1/2}\right].\eqno(5.2.16)$$
The solution method is to firstly select an approximation
for the unknown variables, $\phi_c$
and $\beta$. These are given by the assumption that the bubble 
is almost spherical, radius $a$ with the
top a non-dimensional height above the free-surface given by (5.2.3), so that
$$\bar{R}_c^{(0)}=\sqrt{E_o/6},\eqno(5.2.17)$$ 
thus
$$\phi_c^{(0)}=\arcsin(\bar{R}_c^{(0)}),\eqno(5.2.18)$$
and
$$\beta^{(0)}=E_o/4.\eqno(5.2.19)$$
For this initial approximation, we must restrict $E_o$ to be less than $6$, 
which imposes a maximum on the bubble radius of about $0.34cm$.
Newton iteration is used to calculate the values of $\phi_c$ and
$\beta$ subject to
the equations (5.2.14) and (5.2.15) with $\bar{V}$ given by (5.2.16).

An alternative bursting mechanism 
which will work well only for
small bubbles, but that illustrates the importance of a physically realistic
bursting procedure for larger bubbles can be used for bubbles that have
been followed numerically from a distance below the free-surface. 
This is to simply burst the bubble when its uppermost nodes become 
closer to the free-surface than some critical distance.
Any points on the bursting
bubble or free-surface closer together than some other prescribed distance
are removed. The remains of the bubble and free-surface
are then re-joined to form an indented free-surface. We assume that 
the burst takes place in an instant, so that the potentials on the
other parts of the bubble are unaffected. 
Note, however, that when the $m$th bubble bursts the constant, $k_m$, 
must be subtracted
from the  potentials on that bubble before re-connecting it
to the free-surface.
To smooth the new surface
slightly at the join, a new node is added at the 
midpoint of the first nodes removed from the 
old surfaces and the potential at this node
is simply taken as the average of the potentials at the 
old two nodes.
The calculation is then allowed to continue.
This method of bursting can be anticipated to 
result in a significant overestimate of the energy release
when a large bubble bursts, as shown below.
\vskip 15pt
\hbox{\bf 5.3 Solution by boundary integral method.}
\vskip 5pt
The solution method relies on writing Laplace's equation for the
potential, $\phi$, in the form of 
an integral equation, which can be solved in a discrete form.
Define the surfaces $\partial\Omega$, the
boundary of $\Omega_-$, and $C=C_0(R)\cup \tilde C_0(R)
\cup C_1 \cup\cdots\cup C_M$,
where $\tilde C_0(R)$ is a spherical arc of radius $R$, centred at a
point $\bd{x}^\star\in\Omega_-\setminus\partial\Omega$,
and take  $C_0(R)$ to be
the finite portion of $C_0$ extending as far as
$\tilde C_0(R)$, in such a  way that $\tilde C_0(R)\cup C_0(R)$ encloses
all of the $C_m$, $(m=1,\ldots,M)$.
If we now chose a point $\bd{x}$ on $\partial\Omega$,
we may use Green's integral formula written in the form
$$\phi(\bd{x}^\star)-\phi(\bd{x})=\int_C
\left(G(\bd{x}^\star,\bd{x}'){\partial\phi\over\partial n}(\bd{x}')-
(\phi(\bd{x}')-\phi(\bd{x})){\partial G\over\partial n'}
(\bd{x}^\star,\bd{x}')\right)dS',\eqno(5.3.1)$$
where
$$G(\bd{x}^\star,\bd{x}')={1\over 4\pi |\bd{x}^\star
-\bd{x}'|},
\eqno (5.3.2)$$
is the fundamental solution of the three-dimensional Laplace 
equation. As the fluid is at rest at infinity,
we assert that $\phi(\bd{x})\rightarrow 0$
as $|\bd{x}|\rightarrow\infty$
and so for $\bd{x}'\in\tilde C_0(R)$, 
$dS=O(R^2)$,
$\phi=O(1/R)$
and $\partial\phi/\partial n=O(1/R^2)$ as $R\rightarrow\infty$.
It is clear that the integral over 
$\tilde C_0(R)$ (with the exception of the term in $\phi(\bd{x})$
which approaches the value $-\phi(\bd{x})/2$)
behaves as $1/R$ and so vanishes when we take the limit
$R\rightarrow\infty$. We also take the
limit $\bd{x}^\star\rightarrow\bd{x}$, which is trivially
done, due to the regularity of the integral.
As the principal value integral of the remaining term 
in $\phi(\bd{x})$ is now zero, 
$${1\over 2}\phi(\bd{x})=\int_{\partial\Omega}
\left(G(\bd{x},\bd{x}'){\partial\phi\over\partial n}(\bd{x}')-
\phi(\bd{x}'){\partial G\over\partial n'}
(\bd{x},\bd{x}')\right)dS',\quad \bd{x}\in \partial\Omega.
\eqno(5.3.3)$$
Finally, substituting $f$ for $\phi$ as in (5.2.1)
and using the fact that $k_0(t)\equiv 0$, gives the expression
$${1\over 2}f(\bd{x})=k_m+\int_{\partial\Omega}
\left(G(\bd{x},\bd{x}'){\partial\phi\over\partial n}(\bd{x}')-
f(\bd{x}'){\partial G\over\partial n'}
(\bd{x},\bd{x}')\right)dS',\quad \bd{x}\in C_m,
\quad m=0,\ldots,M.
\eqno(5.3.4)$$
The integral equation (5.3.4) is
in a form suitable for use in the program described below
which is a modified version of the code of Best and Kucera (1992), which
was originally designed for cavitation 
and explosion bubbles in an infinite fluid.

Initially, nodes are equally spaced by arc length on the bubbles, but
on the free-surface, they are placed
on the portion $0\le r\le R_{max}$ of $B_0$, for some large $R_{max}$,
with the non-linear distribution
$$r_0i={iR_{max}\over 2N_0}\left(1+(i-N_0-1)^{-2}\right),
\quad i=0,\ldots,N_0,\eqno(5.3.5)$$
so that points are distributed more densely near the axis, where 
resolution is particularly important. Subsequently, $R_{max}$ is allowed to alter
dynamically with the position of the end node on the free-surface,
namely
$R_{max}(t)=r_{0N_0}$.
Similarly we define $Z_{max}(t)=z_{0N_0}$ and $F_{max}(t)=f_{0N_0}$.

In order to evaluate the integral over the infinite free-surface,
an approximate expression is required for the quantities in the integrand
of (5.3.4).
The motion of a bursting bubble may be thought of as a superposition
of harmonic modes of which the spherically symmetric mode, 
corresponding to the isotropic collapse of a cavity due to surface tension
may be represented by a potential sink. The potential may therefore be 
expected to contain a term $-\dot RR^2/r$, with the cavity radius,
$R(t)$, governed by a differential equation
similar to the Rayleigh equation for spherical cavitation 
bubbles, but driven by surface tension rather than a 
constant pressure difference. 
A consideration of the linearised boundary
conditions for large $r$ indicates that at $z=0$,
$\partial^2\phi/\partial t^2\sim -\partial\phi/\partial z$.
This condition is not satisfied by the source or sink alone but is satisfied 
by a potential that falls off as $r^{-3}$.
Indeed, the solution to the linearised free-surface
response to a sink contains a source of equal strength as an image in
the free-surface, thus resulting in a dipole far field. For a source of
strength $m(t)$ at $\bd r=\bd x=(0,0,-h)$,
using methods similar to those in section 3.3,
it can be shown that the potential in this case,
in the absence of surface tension, is given by
$$\phi(\bd r)=m(t)\left\{{1\over |\bd r-\bd x|}-{1\over |\bd r-\bd x'|}
\right\}+
2\int_0^\infty e^{\xi(z-h)}J_0(\xi r)
\int_0^t m(\tau)\xi^{1/2}\sin[\xi^{1/2}(t-\tau)]d\tau d\xi,$$
where $\bd x'=(0,0,h)$ is the position of the image singularity
and $J_0$ is a zeroth order Bessel function.
As the source strength in the integrand, $m(t)$, is bounded on the interval $(0,t)$,
we can expand the integrand of the time integral in a power series
and interchange the order of integration and summation. Upon using the
fact that the zeroth order 
Hankel transform, with respect to $r$, of a unit source at the origin
is $\xi^{-1}e^{\xi z}$ for $z\le 0$, and differentiating repeatedly 
with respect to $z$, we can write the potential in the form (see
for example Stoker, 1957, p190)
$$\phi(\bd r)=m(t)\left\{{1\over |\bd r-\bd x|}-{1\over |\bd r-\bd x'|}
\right\}+
2\sum_{i=1}^\infty\left\{{\partial^i\over\partial z^i}
\left(1\over|\bd r -\bd x'|\right)
\int_0^t m(\tau)(-1)^{i-1}{(t-\tau)^{2i-1}\over(2i-1)!}d\tau\right\}.$$
Hence at the free-surface, the dipole term of the summed quantity
dominates the potential so that $\phi\sim r^{-3}$ as $r\rightarrow\infty$.
The assumption of a dipolar far field is also consistent with the fully
non-linear boundary conditions and implies that the free-surface 
elevation is also of order $r^{-3}$ for large $r$. Thus for the purposes of the
numerical computations, we write for $r\ge R_{max}$
$$f_0(r)=F_{max}\left({R_{max}\over r}\right)^3\quad\hbox{and}\quad
z_0(r)=Z_{max}\left({R_{max}\over r}\right)^3.\eqno(5.3.6)$$
A similar asymptotic technique was used in the 
boundary integral method of Oguz and Prosperetti (1989),
to calculate the motion of the contact line between a drop and 
a free-surface.

As described in Chapter 4, cubic splines are used to interpolate
the surface nodes.
Since the spline for the free-surface 
only reaches as far as $r=R_{max}$, the end conditions
there are
$${\partial z_0\over\partial s}={-3z/r\over
\left(1+(3z/r)^2\right)^{1/2}},\quad
{\partial r_0\over\partial s}=\left(1+(3z/r)^2\right)^{-1/2}\quad
\hbox{and}\quad
{\partial f_0\over\partial s}={-3f/r\over
\left(1+(3z/r)^2\right)^{1/2}},\eqno(5.3.7)$$
thus ensuring continuity of first derivative
with the analytic portion of the free-surface.
The corresponding asymptotic form for the normal derivative
of the potential, for large $r$ is also $r^{-3}$, thus
we assume
$$\psi_0(r)=\Psi_{max}\left({R_{max}\over r}\right)^3,
\eqno(5.3.8)$$
for $r\ge R_{max}$, where $\Psi_{max}=\psi_{0N_0}$.
\vskip 15pt
\hbox{\bf 5.4 Viscous effects.}
\vskip 15pt
\c{\it 5.4.1 Boundary layer approximation.}
\vskip 5pt
As mentioned in section 5.1, it has been proposed that a possible
cause of cell damage is high shear stresses
in the boundary layer of a bursting bubble.
Independent calculations by Chalmers (1992) suggest
that the shear stresses occurring
in the downward jet, particularly for small
bubbles, may be large enough to cause cell damage.
It is therefore of interest to calculate the vorticity development
in the downward jet region to try and ascertain the magnitude
of stresses there.
It is also important to gain information as to the effect of viscosity
on the interface motion, particularly on the development of the jet
where high rates of strain are expected.

There have been a number of studies where weak viscous effects have 
been included into otherwise inviscid boundary integral formulations.
Miksis, Vanden-Broek and Keller (1982) included a modified
boundary condition into calculations of the steady state 
shape of a rising  bubble (see Chapter 4).
Their modification took into account
only the change in the normal stress at the surface due to viscosity,
and ignored the pressure drop across the boundary layer itself
which, as they noted, becomes particularly 
important on the lower part of the bubble.

This was taken a stage further by
Lundgren and Mansour (1988), where an analysis
of the boundary layer equations resulted in expressions for 
the pressure difference and normal component of the velocity 
perturbation due to the viscous layer.
For completeness, we indicate the method of
Lundgren and Mansour and then describe a way in which this
can be extended so that the 
tangential component of the boundary layer velocity
can be obtained. With this information it is possible to
identify approximate initial conditions for  
the vorticity distribution as the jet is about to form and 
thus gain greater understanding of the flow in the 
downward jet.

Lundgren and Mansour's method
(referred to as LM) is as follows.
The velocity field is written in the 
form $\bd{v}=\bd{u}+\bd{U}$,
where $\bd{u}=\nabla\phi$ is the usual potential flow field
and $\bd{U}=\nabla\times\bd{A}$ is the rotational flow.
For uniqueness, $\bd{A}$ is taken to be zero outside
the boundary layer. Since the flow is axisymmetric,
$\bd{A}=A\uh{\theta}$. Likewise, the total pressure may be written
as $p^\star=p+P$, where $P$ is the perturbation 
in pressure due to the boundary layer.
The viscous
boundary conditions are the usual ones for a free-surface.
Firstly, a balance of normal stress on either side of all
interfaces is required:
$$-p_m(t)+\kappa=-p^\star+2Re^{-1}\bh{n}\cdot\nabla\bd{v}\cdot\bh{n},
\quad m=0,\ldots,M.\eqno(5.4.1)$$
Secondly, due to the relatively low dynamic
viscosity of the air, there should be no tangential stress at the surfaces:
$$\bh{t}\cdot\nabla\bd{v}\cdot\bh{n}+
\bh{n}\cdot\nabla\bd{v}\cdot\bh{t}=0.\eqno(5.4.2)$$
Here, $\bh{t}$ and $\bh{n}$ are the tangent and normal to the generator of the 
axisymmetric bubble.
The Reynolds number, $Re$, is given by $(\sigma a/\rho\nu^2)^{1/2}$.

On assuming that the boundary layer is thin, with thickness 
$\delta$, and that the 
variation along a surface is of order unity, LM
use the zero tangential stress condition (5.4.2) to 
make the approximation
$${\partial U_t\over\partial n}= -2\bh{t}\cdot\nabla\bd{u}
\cdot\bh{n}+O(\delta^2),\eqno(5.4.3)$$
which shows that $U_t=O(\delta)$ with $A=O(\delta^2)$
and from mass conservation $U_n=O(\delta^2)$. 
Here, subscripts $n$ and $t$ refer to 
normal and tangential components of a local curvilinear coordinate
system fitted to the instantaneous interface shape.
A consideration
of the normal and tangential components of the Navier-Stokes
equations shows that for viscous terms to be retained,
the boundary layer thickness must be related to the Reynolds number
by $\delta=Re^{-1/2}$. Upon neglecting variations of $P$
across the boundary layer, LM are able to integrate the 
tangential equation across it and produce an
equation for the development of $A$ at the surface. Likewise
the normal equation is integrated across the boundary layer
to give an expression for $P$ at a surface, in terms of $A$
and the irrotational velocity, $\bd{u}$.
Before giving these expressions it  is
convenient, for brevity, to introduce the
notation:
$${D\{\bd{w}\}h\over Dt}\equiv{\partial h\over\partial t}
+\bd{w}\cdot\nabla h.\eqno(5.4.4)$$
This is a generalisation of the usual material derivative,
where the rate-of-change of a function of the flow field
is taken following points that move with a velocity $\bd{w}$.
This notation will prove useful when we consider the 
calculation of the tangential velocity due to the boundary layer,
but is appropriate here.
The final boundary conditions found by LM give, 
in the context of the bubble bursting problem,
$${D\{\bd{u}+U_n\bh{n}\}A\over Dt}=
-A\left(2{\partial u_t\over\partial s}-2\kappa^{(t)}u_n+
{\bd{u}\cdot\bh{r}\over r}\right)+2Re^{-1}\left(
{\partial u_n\over\partial s}+\kappa^{(t)}u_t\right)+O(\delta^3),\eqno(5.4.5)$$
where $U_n=(\partial (rA)/\partial s)/r$ and
$\kappa^{(t)}$ is the curvature of the interface in 
a plane through the axis of symmetry.
For the potential,
$$\eqalign{{D\{\bd{u}+U_n\bh{n}\}\phi\over Dt}=&
{1\over 2}|(\bd{u}+U_n\bh{n})|^2
+p_m(0)-p_m(t)-{E_o\over 4}(z+\gamma_m)
+\kappa-\kappa_m\cr
&+2A\left({\partial u_n\over\partial s}+\kappa^{(t)}u_t\right)
+2Re^{-1}\left({\partial u_t\over\partial s}
-\kappa^{(t)}u_n+{\bd{u}\cdot\bh{r}\over r}\right)+O(\delta^3).}\eqno(5.4.6)$$
The modification to the pressure is likewise given by 
$$P=2A\bh{t}\cdot\nabla\bd{u}\cdot\bh{n}+O(\delta^3).\eqno(5.4.7)$$
The model of LM is insufficient if we wish to go on to approximate
the flow field after the boundary layer separates from the
bubble cavity as the jet is about to form, since it does not
include a scheme for calculating the total tangential velocities
in the layer: the contribution to the tangential velocities is
an order of magnitude higher than
to the normal velocities.
To do this, we solve a partial differential
equation based on the tangential component of the 
boundary layer equation (LM's equation (4.18) with a total Lagrangian
derivative),
$${dU_t\over dt}+U_t\bh{t}\cdot\nabla\bd{u}\cdot\bh{t}=
Re^{-1}{\partial^2U_t\over\partial n^2}+O(\delta^2),\eqno(5.4.8)$$
subject to the boundary conditions (5.4.3) at the surface
and $U_t\rightarrow 0$ outside the thin boundary layer.
                                      
Equation (5.4.8) can be solved using a finite difference scheme, by
fitting a grid of points to the boundary layer
region. If we wish to continue to use
the model of LM and the boundary integral scheme, it is possible to 
use (5.4.8) as an equation for the Lagrangian evolution of particles 
moving through the fluid in the viscous layer. As the problem is
inherently unsteady,
these grid points will move in relation to one another and thus it will
be difficult to keep track of which nodes are closest to each other
for the purpose of calculating derivatives. 
(For a review of techniques for following closest nodes in
such problems
see for example Boris, 1989). Alternatively, a scheme could be     
devised whereby points are re-meshed after each time step
to keep them on a more convenient grid. However a technique is used 
here in which  the need to re-position at each time step is removed, 
except when the usual repositioning of surface 
nodes is performed in the boundary integral scheme. The idea uses the
fact that the bubble or free-surface is a stress-free interface,
and consequently a material line that is normal to a surface
will remain, locally, orthogonal --- see appendix C.
Since this orthogonality is only local to the point on the surface,
we `linearise' the Lagrangian time derivative by writing
$${d\over dt}={D\{\bd{v}\}\over Dt}={D\{\bd{v}_0-\eta(\bh{n}\cdot\nabla
\bd{v})\big|_0\}\over Dt}
+(\bd{v}-\bd{v}_0+\eta(\bh{n}\cdot\nabla\bd{v})\big|_0)\cdot\nabla,
\eqno(5.4.9)$$
where $\eta$ is the normal distance from a point to the surface
and the subscript $0$ denotes evaluation at the surface.
It is clear from (5.4.9) that as $\eta\rightarrow 0$ the approximate 
Lagrangian derivative --- following points moving at
velocities $\bd{v}^\star\equiv\bd{v}_0-\eta(\bh{n}\cdot\nabla\bd{v})\big|_0$
--- becomes exact. Writing (5.4.8) in terms of this
derivative gives us
$${D\{\bd{v}^\star\}U_t\over
Dt}=Re^{-1}{\partial^2U_t\over\partial n^2}-
U_t\bh{t}\cdot\nabla\bd{u}\cdot\bh{t}-
(\bd{v}-\bd{v}_0+\eta(\bh{n}\cdot\nabla\bd{v})\big|_0)
\cdot\nabla U_t+O(\delta^2).\eqno(5.4.10)$$
The virtue of this approach is that if we follow
mesh points which move with velocities given by
$\bd{v}^\star$, then 
as this is the exact Lagrangian derivative at the surface where
the zero tangential stress condition (5.4.2) is applicable,
the mesh lines normal to the surface will remain normal at 
the surface. The linearisation further ensures that these lines
remain straight since
the transformation mapping the line to its new position a short time later
is linear (as $\bh{n}\cdot\nabla\bd{v}^\star=
(\bh{n}\cdot\nabla\bd{v})\big|_0$).
The advantages over
more usual Lagrangian finite difference schemes are that normal
derivatives are always simple to evaluate;
there is no need to store the position of the mesh points,
only their distances from the surface; and, as
mentioned above, there is no need to re-position the points
other than occasionally as dictated by the underlying boundary integral
scheme (see \S 5.4.2).

The new term on the right-hand side of (5.4.10) as compared with 
(5.4.8) can be shown to be of order $\delta^2$ in the boundary
layer. As we can assume that the potential flow field changes
slowly in the boundary layer, we can view the potential flow
part of $\bd{v}^\star$ as the error in the Taylor expansion of
$\bd{u}$ and so if $\eta=O(\delta)$,
$$(\bd{v}-\bd{v}_0+\eta(\bh{n}\cdot\nabla\bd{v})\big|_0)
\cdot\nabla U_t=(\bd{U}-\bd{U}_0+
\eta(\bh{n}\cdot\nabla\bd{U})\big|_0)\cdot\nabla U_t
+O(\delta^2).\eqno(5.4.11)$$
Splitting the remaining velocities into their normal and tangential
components, shows the first
two terms to be of order $\delta^2$, leaving
$$\eta\left({\partial U_t\over\partial n}\Bigg|_0\bh{t}+
{\partial U_n\over\partial n}\Bigg|_0\bh{n}\right)\cdot\nabla U_t+
O(\delta^2).\eqno(5.4.12)$$
This is of order $\eta\delta$ which is of order $\delta^2$
in the boundary layer. Knowing $U_t$ to order $\delta$ is 
consistent with calculating $A$ to order $\delta^2$ since
$U_t\approx \partial A/\partial n$.

The second term on the right of (5.4.10) can also be approximated.
Expanding in a Taylor series,
$$\bh{t}\cdot\nabla\bd{u}\cdot\bh{t}=
(\bh{t}\cdot\nabla\bd{u}\cdot\bh{t})\big|_0-
\eta\left[\bh{n}\cdot\nabla(\bh{t}\cdot\nabla\bd{u}\cdot\bh{t})
\right]\Big|_0
+O(\eta^2).\eqno(5.4.13)$$
Since this term is multiplied by $U_t$, the first order correction
in (5.4.13) may be ignored in the boundary layer.
The final form of (5.4.10) is thus
$${D\{\bd{v}^\star\}U_t\over
Dt}=Re^{-1}{\partial^2U_t\over\partial n^2}-
U_t(\bh{t}\cdot\nabla\bd{u}\cdot\bh{t})\big|_0
+O(\delta^2).\eqno(5.4.14)$$
Derivatives of $U_t$ are calculated by fitting quadratics to
nodal values along a normal to the surface, with the value on the 
surface given by (5.4.3) and $U_t$ assumed zero outside the boundary
layer. This is then repeated to give the second derivatives.

The rate-of-change of the distance between the surface node
and a corresponding mesh node on the normal to the surface
is given by the difference in normal velocity between the 
surface and the mesh point. Thus if we
now view $\eta$ as a parametrisation of a mesh normal 
to a surface with the linearised time derivative, we see that
$${\partial\eta\over\partial t}=\eta(\bh{n}\cdot\nabla
\bd{v})\big|_0.\bh{n}.\eqno(5.4.15)$$
Writing the continuity equation for the potential flow in terms of normal and 
tangential derivatives, gives
$${\partial u_n\over\partial n}+{\partial u_t\over\partial s}
-\kappa^{(t)}u_n+{\bd{u}\cdot\bh{r}\over r}=0,\eqno(5.4.16)$$
so that (5.4.15) may be written as
$${\partial\eta\over\partial t}=
-\eta\left({\partial u_t\over\partial s}-\kappa^{(t)}u_n+
{\bd{u}\cdot\bh{r}\over r}\right)\Bigg|_0+O(\delta^2).\eqno(5.4.17)$$
Note that as the linearised derivative matches up with the total derivative
at the boundary, the evolution of $A$ and of $\phi$ can be approximated
by using equations (5.4.5) and (5.4.6), however (5.4.6) first needs
modifying to take account of the tangential component of the derivative 
by adding a term $U_t\bh{t}\cdot\nabla\phi$ to both sides.
This order $\delta$ term simply corrects for the
additional tangential motion without
otherwise affecting the potential distribution on the surface. 
We write (5.4.6) as
$$\eqalign{{D\{\bd{v}_0\}\phi\over Dt}=&
{1\over 2}\big(|\bd{v}_0|^2-U_t^2\big)
+p_m(0)-p_m(t)-{E_o\over 4}(z+\gamma_m)
+\kappa-\kappa_m\cr
&+2A\left({\partial u_n\over\partial s}+\kappa^{(t)}u_t\right)
+2Re^{-1}\left({\partial u_t\over\partial s}
-\kappa^{(t)}u_n+{\bd{u}\cdot\bh{r}\over r}\right)+O(\delta^3).}
\eqno(5.4.18)$$

We assume that the fluid starts off stationary so that all
perturbation quantities due to the boundary layer are initially zero. 
For the case of a bubble bursting, we are thus neglecting any
vorticity that would have been created in the boundary layer as the bubble
rose to the surface.

Certain limiting cases as $r\to 0$ need to be calculated. By symmetry,
$u_t$, $\partial u_n/\partial s$ and $\partial^2 u_t/\partial s^2$
all vanish on the axis. From this it follows that 
$\bd{u}\cdot\bh{r}/r\to\bh{t}\cdot\nabla\bd{u}\cdot\bh{t}$
as $r\to 0$.  For motions starting in a perfect 
state of rest, $A=0$ on the axis initially and so equation (5.4.5) 
will ensure that $A$ remains zero there throughout the motion.

Moore (1963), calculated expressions for the boundary layer on a
spherical bubble moving through a liquid. He found the boundary
layer approximation to be invalid in a region of width
$\delta^{1/3}$ near to the rear of the
bubble due to separation. 
Vorticity in this region is confined to a layer of thickness $\delta^{1/3}$
and so viscous forces are no longer as important as inertial forces.
Closer still to the
axis of symmetry, the layer thickens further to form a wake, where
stream surfaces are eventually cylindrical and parallel to the axis. 
The wake region has width $\delta^{1/2}$ so that again diffusive
viscous effects are negligible, with the movement of vorticity being
dominated by advection. 
We shall assume that a similar, thin wake structure
also exists in the case of a non-spherical rising bubble
or for a bursting bubble until the jet forms.
Jet formation will cause vorticity from a larger region near the underside of
the bubble cavity to be advected into the downward jet.
From a modelling viewpoint we can correct for this thin wake to a certain
extent, provided that the spacing between nodes is not too small,
by using an appropriate expression for the perturbation to the 
normal velocity at the node on the axis beneath the bubble. 
This normal velocity only directly affects adjacent nodes through the
arc-length derivatives taken on a cubic spline for $A$.
Indeed, the radial derivative of $A$ which is, apart from a factor of 2, 
precisely the
normal velocity at the end node, is required here in order to
fit the cubic spline to $A$.
In order to calculate $U_n$ here, we can use the 
vertical component of the Navier-Stokes equations. Moore's
estimates indicate that inertial terms dominate both viscous and pressure gradient
terms, so that we may write
$$\eqalign{{d\over dt}\left({\partial A\over\partial s}\right)&=
-{\partial A\over\partial s}{\partial^2\phi\over\partial n^2}+
{1\over Re}{\partial^2 U_n\over\partial s^2}\cr
&=2{\partial A\over\partial s}\left({\partial u_t\over\partial s}-\kappa^{(t)}u_n\right)
+{1\over Re}{\partial^2 U_n\over\partial s^2},}\eqno(5.4.19)$$
on using (5.4.16). The highest order viscous term is retained in order
to start the calculations off, as at the time the boundary layer
calculations commence, all perturbations are assumed zero.

If calculations also include rising bubbles, for a cubic spline to be fitted
to $A$ the value of $\partial A/\partial s$ is also
required at the top, on the axis of symmetry.
Similar to equation (C2), it can be shown that
$${\partial\over\partial s}\left({dA\over dt}\right)=
{d\over dt}\left({\partial A\over\partial s}\right)
+{\partial A\over\partial s}{d\lambda\over ds}{d\over dt}\left(
{ds\over d\lambda}\right).\eqno(5.4.20)$$
From (C2) we also obtain
$$\bh{t}\cdot\nabla\bd{u}\cdot\bh{t}={d\lambda\over ds}{d\over dt}
\left({ds\over d\lambda}\right)
+O(\delta).\eqno(5.4.21)$$
Combining (5.4.20), (5.4.21) and (5.4.5) and taking the limit as $s\rightarrow 0$,
gives an equation
for the time evolution of $\partial A/\partial s$ on the axis above a bubble,
$${d\over dt}\left({\partial A\over\partial s}\right)=
-4\left({\partial u_t\over\partial s}-\kappa^{(t)}u_n\right)
{\partial A\over\partial s}
+{2\over Re}\left({\partial^2 u_n\over\partial s^2}+
\kappa^{(t)}{\partial u_t\over\partial s}\right)+O(\delta^2).\eqno(5.4.22)$$
The condition on the slope of $A$ at $r=R_{max}$, is found 
by a consideration of the 
limiting form of (5.4.5).
Making use of (5.3.6) and (5.3.8), we see that $A=O(1/r^4)$ as $r\to\infty$.

When the walls of the bubble cavity start to move inwards immediately before
forming the jet, the 
vorticity created in the thin boundary layer is advected
into the bulk of the liquid. 
The assumption of a thin boundary layer is violated once this 
separation takes place, causing the boundary layer calculations 
using the method described above to break down. 
However, we can follow this 
advection of vorticity from the boundary layer using the vorticity equation,
$${D\{\bd{u}\}\ub{\omega}\over Dt}\approx
{d\ub{\omega}\over dt}=
\ub{\omega}\cdot\nabla\bd{v}+Re^{-1}\nabla^2
\ub{\omega}.\eqno(5.4.23)$$
Since $\ub{\omega}=\omega\uh{\theta}$, (5.4.23) can be written as
$${D\{\bd{u}\}\omega\over Dt}\approx\omega{\bd{u}\cdot\bh{r}\over r}+
Re^{-1}\left(\nabla^2\omega-{\omega\over r^2}\right).\eqno(5.4.24)$$
We assume that the layer is initially thin enough so that derivatives
of the pertubation velocity across it are larger than those taken
along it, thus the initial vorticity may be approximated by
$$\omega={\partial U_t\over\partial n}-{\partial U_n\over\partial s}-
\kappa^{(t)}U_t\approx{\partial U_t\over\partial n}.\eqno(5.4.25)$$
As the boundary layer separates, viscous terms will become less important,
and we may assume that the vorticity development can be described 
by pure convection
together with changes brought about by the stretching of vortex rings moving
towards or away from the central axis.
Hence by following material points, in the sense defined by the
potential flow field,
we can calculate an estimate of the vorticity evolution in both the
upward and downward
jets. We examine {\it a posteriori} our assumption 
that the self-induced motion of the
vorticity can be neglected when compared 
to the advection due to the irrotational flow.
\vskip 15pt
\c{\it 5.4.2 Repositioning and smoothing.}
\vskip 5pt
In order to maintain a good resolution of surfaces,
it is necessary to reposition 
points periodically. Points on bubbles are moved 
so that they are once again evenly 
spaced. Points on the free-surface are spaced so that the
arc lengths follow a non-linear 
distribution similar to that in equation (5.3.5).
Regular smoothing with the Longuet-Higgins and Cokelet (1976)
smoothing formula was also required in order to prevent high frequency
surface oscillations. 

As was mentioned in section 5.4.1, finite difference points in the boundary layer
are repositioned with nodes on the surfaces. Since these points must always
lie on normals to the surface, this can be done as follows. The standard
repositioning of the surface nodes decides between which two 
old nodes a new node
is to be placed. Values of $U_t$ on the new boundary layer normal are chosen
at the same heights as those on the old normal immediately to its left
by a simple linear interpolation scheme, based on the assumption that
the surface curvature will not change much from one node to the next, 
so that the 
three normals meet at a single point. 
Hence changes in $U_t$ at nodes equi-distant from the surface on
the normals will be in the same proportion as the 
arc lengths along the surface. To do this the value of $U_t$ on the 
normal immediately to the right of the new normal is also calculated
at the same height by quadratic interpolation. This is repeated for all 
nodes on each normal.
\vskip 15pt
\hbox{\bf 5.5 Results and discussion.}
\vskip 5pt
The methods described in the previous sections allow numerical calculation of
bubble and interface shapes at various times during bubble rise or burst.
A number of cases of bubbles bursting 
are shown in figures 5.3(a)-(f) for different bubble sizes.
It is clear that the smaller bubbles, as their internal pressures are higher,
burst from lower in the fluid and thus release a proportionately 
greater amount of energy in the 
form of the high-speed liquid jets observed
experimentally (see for example, MacIntyre, 1972; Kientzler et al, 1954).
The larger bubbles form wider jets and this is observed in Garner et al (1954).
The velocity of the
central node on the free-surface increases sharply as the jet is formed.
In the figures 5.3(a)-(d), there is a noticeable drop in velocity 
immediately before the jet begins to rise. This can be seen more clearly
in a magnification 
of the motion of figure 5.3(b), shown in figure 5.4. As the cavity walls straighten
early on in the burst, the bottom rises slowly and flattens off,
creating a ring of high curvature at the `corner' between the 
near vertical wall and the horizontal base. The effect of surface tension then pulls
this ring inwards. Thus the liquid --- which flows from the region of opposite
curvature near the highest point of the surface --- now flows predominantly
towards this ring rather than towards the very lowest part of the cavity. The 
node on the axis of symmetry is then observed to slow down. 
As the walls close in further,
an axisymmetric cusp almost occurs and the liquid is forced upwards. 
This high surface curvature and the subsequent large velocities and
accelerations suggest possible physical similarities between jet
formation and the
classes of free-surface flows considered by Longuet-Higgins (1980, 1983).

The jet accelerates for only a short time, and then slows as it rises. For the first
three calculations of figure 5.3, the speed of the uppermost point
increases very slightly, once more due to the thinning of the jet and the breaking off
of a drop, at which point the calculations must stop.
As velocities scale with respect to $(\sigma/\rho a)^{1/2}$,
it is clear that smaller bubbles result in faster jets. 

Compare these calculations with figure 5.5, where a 
$3mm$ radius bubble is burst from a
completely submerged position. Here, unlike in figure 5.3(f),
a high-speed jet is formed. 
Due to the greater influence of gravity, this jet is wider and
not as fast as those produced by smaller bubbles,
which burst with similar initial configurations.

The calculated free-surface shapes before jet production
are repeated for the $0.75mm$ bubble in figure 5.6 in order to show a 
comparison with the experimental results of Kientzler et al (1954). 
Slight differences in the initial motion can be attributed to the fact that the 
experimental profile starts off slightly lower in the fluid.
The jet produced in the experiments is asymmetric and
breaks up significantly earlier than calculated, so that the comparison
for later times is not so good. This is not shown, due to the difficulty
of identifying the surface position of the lower part of the jet which,
in the photographs of Kientzler et al (1954), is obscured by the surrounding
bubble crater. In addition to the inevitable neglect of asymmetric instabilities,
node repositioning and smoothing may cause discrepancy
in the jet shapes.
Enhancements to the numerical
procedure, to the surface representation or so that smoothing is no longer required, 
may improve this. The effect of a boundary layer on the
jet is discussed below.

The existence of a downward jet in the calculation is confirmed by placing 
a second bubble directly below the bursting bubble. 
It can be seen in figure 5.7 that
the top of the second bubble is firstly pulled up 
by the effect of the low pressure
around the highly curved bursting bubble cavity. 
Then, just as the jet forms, the 
top of the lower bubble develops a dimple. 
The lower bubble then begins to rise
and soon starts to form a jet from below, characteristic
of bubbles of this size (see Chapter 4).
 The jet velocity for the bursting bubble reaches a slightly higher
peak, but is otherwise largely unaffected by the following bubble.
Likewise, the centroid velocity of the following bubble, indicated by the 
lower trace of the velocity plot for figure 5.7, undergoes a small dip as the 
jet of the bursting bubble forms.

Using the boundary layer calculations of section 5.4, we find that
the surface shapes, up until jet formation, are 
indistinguishable from those in the figures above. Unfortunately boundary layer separation
over a significant region occurs
just before the jet forms (figure 5.8) so that the 
boundary layer calculation must stop. The dissipative effect of viscosity
on the jet can be seen by allowing a boundary layer to develop just
after the jet has begun to form (at
$t=0.5$ for the $0.75mm$ case). Calculations run as far as the time
when the droplet at the end of the jet begins to develop. 
Figure 5.9 shows the resulting jet to be very slightly shorter than in the 
corresponding potential flow case at the same time.

The pressure in the fluid is calculated using the Bernoulli formula (4.3.1) 
together with
the boundary layer modification (5.4.7). To improve resolution near to the 
surface, where the boundary integral formula for the potential in the fluid is
badly behaved due to the weakly singular kernel, the pressure there is given by
$$p^\star-p_\infty={dk_m\over dt}+\kappa_m-\kappa+{\gamma_mE_o\over 4}+
{2\over Re}\bh{n}\cdot\bd{e}\cdot\bh{n},\eqno(5.5.1)$$
where $\bd{e}=e_{ij}$ is the rate-of-strain tensor.
Since the fluid is incompressible,
$$\bh{n}\cdot\bd{e}\cdot\bh{n}=-{\partial u_t\over\partial s}+u_n\kappa^{(t)}
-{\bd{u}\cdot\bh{r}\over r}+O(\delta).\eqno(5.5.2)$$
As cells are more likely to be ripped apart by straining flows than by 
a strong velocity gradient in one particular direction, 
when damage can be reduced by the cells rotating with
the fluid, the energy dissipation rate, $\Phi=2\mu e_{ij}e_{ij}$, gives a
good indication of the stresses placed on a cell in the fluid, thereby
providing a possible measure of the damaging nature of a particular flow
field.

In the fluid, outside the
boundary layer, the non-zero rates-of-strain are given by
$$e_{rr}={\partial u_r\over\partial r};\quad e_{\theta\theta}={u_r\over r};
\quad e_{zz}={\partial u_z\over\partial z}
\quad\hbox{and}\quad
e_{rz}={1\over 2}\left\{{\partial u_r\over\partial z}+
{\partial u_z\over\partial r}\right\}.\eqno(5.5.3)$$
At the surface, by virtue of the stress-free boundary 
condition, the rates-of-strain are
$$\eqalign{
e_{nn}=-{\partial u_t\over\partial s}+u_n\kappa^{(t)}
-{\bd{u}\cdot\bh{r}\over r}+O(\delta);\quad e_{tt}=
{\partial u_t\over\partial s}
-u_n\kappa^{(t)}+O(\delta)
\quad\hbox{and}\quad
e_{\theta\theta}={\bd{u}\cdot\bh{r}\over r}+O(\delta),}\eqno(5.5.4)$$
with all off-diagonal terms zero.
By exploiting incompressibility,
the corresponding dimensional energy dissipation rate at the surface
may be written as
$$\Phi=4\mu(e_{nn}^2+e_{tt}^2+e_{nn}e_{tt}).\eqno(5.5.5)$$

The pressures above atmospheric pressure ($dyn\,cm^{-2}$) for the $0.75mm$
and $3mm$ bubbles are plotted in
figures 5.10 and 5.11. For the smaller bubble, the initial motion is driven
by the very high pressure around the rim of the cavity. This is less
dramatic in the $3mm$ case where gravity also plays a significant role
in shaping the motion.
In frames 11(a) and (b),
the low pressure around the underside of the bubble intensifies
as a small wave
that was initially just below the neck of the bubble crater
moves down the bubble increasing in curvature. This wave can also be seen
clearly on the second frames of figures 5.3(a)-(c), and moving on 
figure 5.4.
MacIntyre (1972) observed another ripple with a smaller wavelength
immediately in front of this as a possible Crapper capillary wave.
This is only resolved by the code if a high point density is
used, for example when $R_{max}$ is reduced to about $5$.
However, this results in excessively fast, thin jets.

As the walls of the cavity collapse inwards (figure 5.10(b))
a ring of high pressure can be seen to 
develop. This then moves downwards until it finally becomes a
point of high pressure directly beneath the bubble (figure 5.10(c)).
This pushes fluid upward and downward to produce the
jets evident in previous figures. The high pressure remains during the 
early stages of
jet formation, further accelerating the jet (figure 5.10(d)).
The pressure in the jet is relatively high due to surface tension which
acts so as to thin it until one or more drops break off.
For the large bubble, the pressure rises only slightly as the much smaller jet is
about to form. This can be seen through the slight upward bending of the
contour lines of the highly visible hydrostatic pressure.

Figure 5.12 shows that the largest maximum pressures occur for the smallest
bubbles. 
This was expected from the experimental studies on cell damage. However, 
it seems that the pressures themselves
are insufficient to kill cells, being equivalent to the hydrostatic 
pressure due to only a few centimetres of water.

Figures 5.13 and 5.14 show the energy dissipation rates ($dyn\,cm^{-2} s^{-1}$)
for the same times as above. It is 
seen that the peak energy dissipation rate moves round with the wave of 
fluid, where the pressure is low,
increasing in magnitude until the jet forms. 
MacIntyre (1972) indicated that this ripple was a site where
high rates of strain were likely.
Indeed, high rates of strain are to be expected when the jet is forming due to
the extensional flow around a stagnation point near to the pressure maximum.
Fluid must be drawn in from the sides towards the axis of symmetry and then rapidly
accelerated upwards or downwards into one of the jets.
Such flows are potentially lethal to cells which may be stretched and ruptured
by the high strain rates.
Figure 5.15 shows that the maximum energy dissipation rates, which
occur beneath the
bubble, immediately before jet formation, increase rapidly with decrease in bubble
radius. Again, this is in keeping with the finding that cell damage is 
greater for smaller bubbles. There is no complete agreement
on the strength of cells. Orton and Wang (1990) suggest that
stresses of the order of $10^3dyn\,cm^{-2}$ are sufficient to cause
death, whereas Zhang et al (1991) give cell bursting pressures of
about $5\times 10^4 dyn\,cm^{-2}$. 
The calculated maximum energy 
dissipation rates for the smallest bubbles 
are equivalent to stresses of the order
of $10^4dyn\,cm^{-2}$, indicating that bubble bursting can
create an hydrodynamic environment which may be deadly for cells.

The boundary layer makes almost no difference
to the calculated values of energy dissipation rate. This
is mainly because the effect of the boundary layer in ensuring
zero tangential stress at the surface, is exploited even when no boundary layer 
is being used in the calculations, in order to improve resolution
in the contouring program.

It was proposed by Orton and Wang (1990) that cell death rates are 
closely related to the rate of liquid entrainment from bursting bubbles.
As it seems reasonable to assume that the underlying cause of cell 
damage must be some form of hydrodynamic stress, there should be a 
corresponding relationship between this stress and the entrainment rate.
To investigate this, we calculated estimates for the 
volume of the first drop released. 
The calculated entrainment rates are 
compared with the jet drop sizes of Garner et al (1954) in figure 5.16.  
Note that only the size of the first released drop is calculated
and no allowance is made for the volume of fluid entrained by the rupture of the
film above the bubble, which is significant for only the largest bubbles
(see Garner et al, 1954).
The droplet sizes for the smaller bubbles agree to an order of magnitude.
Differences are partly because of 
the difficulty in predicting the precise time and place of disengagement,
but are also
due to discrepancies in the jet formation itself pointed out above.
Smoothing and repositioning of nodal points in the jets may smooth out
small scale instabilities that would otherwise affect drop formation.
An interesting point to note is that the calculated entrainment rates seem
to be at a maximum for bubbles of radii around $2mm$. Even though the
bubble of figure 5.3(d) does not form a long, thin jet, the relative volume of the 
drop released is much larger
than that of the three smallest bubbles studied.
When this calculation was allowed to continue past the point where
the drop almost disengaged,
the code did not break down, but the jet fell back into the fluid. 
In reality, it would be surely be pinched off by surface tension, before
falling back into the fluid.
We can therefore conclude that 
$2mm$ must be close to the bubble radius beyond which no drop is released. 
Garner et al (1954) suggest that this
cut-off point for jet drop formation is at about $2.5mm$.  
The velocity plot of figure 5.3(d) shows that, at the time of droplet 
formation, the top of the jet is moving slowly downwards.
If this is the 
case, such large droplets near to the cut-off point
may not have been identified as such by the 
experimentalists whose methods hinged on observations and detection of drops
thrown up above the jet.
That larger droplets are formed for larger bubbles is noted by
Garner et al (1954), but the sizes are still small in comparison with 
our calculations.

The approximate vorticity distribution ($s^{-1}$) is shown at three stages of
jet formation in figure 5.17. This vorticity was created 
in a boundary layer by the
initial collapse of the crater for the $0.75mm$ bubble and shed 
prior to jet formation (figure 5.17(a)).
Here, it is shown being convected into the bulk of the fluid by
the formation of the jets.
The sign of the vorticity in figure 5.17(a) can be explained in terms of the dissipative 
effect of viscosity, which reduces tangential velocity gradients at stress-free
surfaces to zero.
As the vorticity is dominated by $\partial U_t/\partial n$, it
will be negative in sign when $\partial u_t/\partial n$
is positive. This was verified by examining the values of
$\partial u_t/\partial n$ at the time of figure 5.17(a).
(In fact $\partial u_n/\partial s+\kappa^{(t)}u_t$
was used as it is easier to calculate accurately, due to the singular nature of
the boundary integral formulation. See (C6).)
In figure 5.17(a), the 
sign of the vorticity was observed to be
generally opposite to that of the tangential velocity, $u_t$.
Bearing in mind that normals are directed
out of the fluid, this implies that at this time
the tangential flow speed at the surface is 
generally faster than nearby in the bulk of the fluid.

It should be emphasised that although the vorticity carried into the jet 
is positive, large negative curvatures and tangential velocities near
the top of the fully developed jet in figure 5.17(c) and at subsequently times
ensure that the tangential speed near to the surface is
greater than at the surface. There is consequently a net 
negative vorticity at the surface due to additional 
vorticity created in the boundary layer 
of the jet which is not accounted for in these figures. 
In the jet of figure 5.17(b) and in the lower parts of the jet of figure 5.17(c),
the tangential velocity of the surface is higher so that the
direction of fluid rotation is as depicted.
As a droplet begins to develop, a region of positive curvature
is formed. Thus negative surface values of $\partial u_t/\partial n$
and positive vorticities are found in this region. 
If we move further down the jet, the thinning becomes less noticeable and 
a further change of the sign of the surface vorticity takes place.
More sign changes are observed to occur around the base of the jet.
The magnitude of the vorticity suggests that
viscous effects make only a very small contribution to
the stresses in the downward jet region.

