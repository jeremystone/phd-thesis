\c {\bigrm Appendix C: Identities for chapter 5.}
\vskip 15pt
\hbox{(i)}

We may easily show that a material curve that is locally normal to a stress-free
surface will remain so. First, consider a general material curve indexed by the
Lagrangian parameter $\lambda$. If the unit tangent
to the curve is $\bh{m}$ then
$$\eqalign{
\bh{m}\cdot\nabla\bd{v}&={\partial\over\partial s}
\left({d\bd{x}\over dt}\right)\cr
&={d\lambda\over ds}{\partial\over\partial\lambda}\left(
{d\bd{x}\over dt}\right).}\eqno(C1)$$
As both derivatives are taken in a Lagrangian sense, they may
be interchanged to give
$$\eqalign{
\bh{m}\cdot\nabla\bd{v}&={d\lambda\over ds}{d\over dt}\left(
{ds\over d\lambda}{\partial\bd{x}\over\partial s}\right)\cr
&={d\bh{m}\over dt}
+\bh{m}{d\lambda\over ds}{d\over dt}\left({ds\over d\lambda}\right).}
\eqno(C2)$$
Thus if $\bh{m}$ is chosen to be the normal or tangent to the 
surface, then
$$\bh{n}\cdot\nabla\bd{v}\cdot\bh{t}={d\bh{n}\over dt}\cdot\bh{t}
\quad\hbox{or}\quad
\bh{t}\cdot\nabla\bd{v}\cdot\bh{n}={d\bh{t}\over dt}\cdot\bh{n}.
\eqno(C3)$$
By using (5.4.2) and adding the equations of (C3), we 
see that for a stress-free surface,
$${d\bh{n}\cdot\bh{t}\over dt}=0.\eqno(C4)$$

\hbox{(ii)}

In order to calculate normal derivatives of the 
tangential velocity, $u_t$, we use the fact that $\bd{u}$
is irrotational to re-write it in terms of tangential derivatives
of the normal velocity, $u_n$, which can be calculated more
accurately. As the tangent vector
is fixed with respect to small changes in the normal direction,
$$\eqalign{
{\partial u_t\over\partial n}
&=\bh{n}\cdot\nabla\bd{u}\cdot\bh{t}.}\eqno(C5)$$
Since $\bd{u}$ is irrotational, the
rank-2 tensor $\nabla\bd{u}$ is symmetric, so that the
right-hand side of (C5) can be re-ordered, giving
$$\eqalign{
{\partial u_t\over\partial n}
&=\bh{t}\cdot\nabla\bd{u}\cdot\bh{n}\cr
&={\partial u_n\over\partial s}+\kappa^{(t)}u_t.}
\eqno(C6)$$
