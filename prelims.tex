%
%Title page
\baselineskip 24pt
\vskip 3in
\bigrmb
\c{Applications of}
\c{Boundary Integral Methods to}
\c{Rising and Bursting Bubbles}
\vskip 5pt
\bigrm
\c{By}
\vskip 5pt
\c{Jeremy Michael Boulton-Stone}
\vskip 2in
\bf
\baselineskip 18pt
\c{A thesis submitted to the}
\c{Faculty of Science}
\c{of the}
\c{University of Birmingham}
\c{for the degree of}
\c{DOCTOR OF PHILOSOPHY}
\timesrm
\vskip 2in
\hskip 4in
\vbox{
\hbox{School of Mathematics}
\hbox{and Statistics}
\hbox{University of Birmingham}
\hbox{B15 2TT}
\hbox{England}
\vskip 5pt
\hbox{March 1993}
}
\pg
%
%Synopsis
\c {\bf Synopsis.}
\vskip 5pt
Direct and indirect boundary integral methods are introduced and compared
for the problem of a two-dimensional, constant volume bubble rising 
under the action of buoyancy in
an inviscid liquid. It is found that the vortex, dipole and Green's formula
approaches produce good results
in agreement with previous theoretical studies and
with experimental evidence, with the source distribution 
method breaking down after only a short time. The Green's formula method is
found to be less stable to perturbed initial conditions. An analytic
stability analysis indicates  that  a 
bubble rising with zero surface tension is unstable, but that the one inch
bubble of Walters and Davidson (1962) is stable to small, short
wavelength perturbations.
By using a conformal map, the theory is extended to the case of a 
two-dimensional bubble near to a free surface, the numerical results 
of which compare 
well with an analytic expression for the surface  elevation  due  to  a 
deeply submerged dipole.

The motions of three-dimensional gas bubbles,
modelled using the equivalent
boundary element method for the axisymmetric geometry,  are found to be
similar to those of their two-dimensional counterparts. 
Important differences in behaviour, mainly of a quantitative nature, 
are highlighted, and discussed. 
Bubble size, though the action of surface tension, is found to have a marked
effect on the motion, determining whether the bubble will eventually split up
into one or two smaller bubbles.
The interaction of two bubbles is also considered, again the differences
from the equivalent two-dimensional case are explored.

Finally, using numerical techniques developed in the
previous chapters, bubble bursting is considered in detail. 
By assuming suitable initial conditions for the interface shape immediately
after film rupture, the effect of bubble volume on the following motion is
examined. The findings, that the smallest bubbles
burst the most violently resulting in greater pressures and energy 
dissipation rates, are broadly in agreement with the experimental evidence
for cell damage by bubbles in bioreactors.
A method for including a boundary layer into the calculations is
developed. This allows an approximation of the vorticity distribution
in the region below the bubble.
The effects of including a boundary layer are found to be small, and the
magnitude of the vorticity in the downward jet region suggests
shear rates to be insignificant compared to strain rates predicted by the irrotational flow model.
\vskip 10pt
Material based on Chapter 2 has been published
in {\sl Computer Methods in Applied Mechanics
and Engineering}, {\bf 102} pp~213-234, 1993;
and that based on Chapter 3 in the {\sl Journal
of Engineering Mathematics}, {\bf 27} pp~73-87, 1993.
\pg
%
%Dedication
\vbox{
\vskip 4in
\c {To}
\vskip 15pt
\c {My wife, Susan Bernadette.}
}
\pg
%
%Acknowledgements.
\vbox{
\vskip 1in
\c {\bf Acknowledgments.}
\vskip 15pt
I wish to take the opportunity to thank my supervisor, Professor
John Blake for his enthusiastic encouragement and guidance, together with 
Mr. Nick Emery, Professor Alvin Nienow
and Dr. Colin Thomas of the School of 
Chemical Engineering;
Dr. Paul Harris
and Dr. John Best for their assistance in certain aspects of
the computational side of the work, and Mr. Peter Robinson
for his friendship and advice.
\vskip 5pt
I am also indebted to the University of Birmingham Computing Service
for use of their facilities
and to the Science and Engineering Research Council for
financial support.
\vskip 5pt
Finally, I would like to express my sincere gratitude to my wife for 
her patience and support throughout the past few years.
}
\pg
%
%Contents.
\vbox{
\obeylines
\vskip 5pt
\c {\bf Contents.}
\vskip 5pt
\hfill Page.
\vskip 5pt
{\bf Chapter 1. Introduction.}
1.1 Aims. \dotfill 1
1.2 Motivation. \dotfill 2
1.3 Summary. \dotfill 5
\vskip 5pt
{\bf Chapter 2. A two-dimensional bubble in an infinite fluid.}
2.1 Introduction. \dotfill 7
2.2 Mathematical preliminaries.
\itemitem{} 2.2.1 Formulation of the problem. \dotfill 8
\itemitem{} 2.2.2 Green's theorem. \dotfill 9
\itemitem{} 2.2.3 Indirect formulations. \dotfill 12
2.3 Source distribution method.
\itemitem{} 2.3.1 Formulation. \dotfill 13
\itemitem{} 2.3.2 Discretisation. \dotfill 14
\itemitem{} 2.3.3 Time-stepping and repositioning. \dotfill 16
2.4 Dipole distribution method. \dotfill 18
2.5 Green's formula method. \dotfill 21
2.6 Vorticity distribution method. \dotfill 22
2.7 Results and discussion. \dotfill 23
2.8 Stability analysis.
\itemitem{} 2.8.1 Formulation of the equations. \dotfill 27
\itemitem{} 2.8.2 Solutions and discussion. \dotfill 32
\vskip 5pt
{\bf Chapter 3. A two-dimensional bubble near a free surface.}
3.1 Introduction. \dotfill 41
3.2 Numerical approach.
\itemitem{} 3.2.1 Problem formulation. \dotfill 43
\itemitem{} 3.2.1 Solution by boundary integral method. \dotfill 45
3.3 Perturbation expansion approximation. \dotfill 49
3.4 Results and discussion. \dotfill 54
}
\pg
\vbox{
\obeylines
\vskip 33pt
\hfill Page.
\vskip 5pt
{\bf Chapter 4. Rising axisymmetric gas bubbles.}
4.1 Introduction.
\itemitem {} 4.1.1 Analytical results. \dotfill 66
\itemitem {} 4.1.2 Numerical methods. \dotfill 70
4.2 The axisymmetric boundary integral method. \dotfill 71
4.3 The rise of constant volume bubbles. \dotfill 73
4.4 Steady rise of a bubble through a viscous fluid. \dotfill 76
4.5 Results. \dotfill 80
\vskip 5pt
{\bf Chapter 5. Bursting bubbles.}
5.1 Introduction.
\itemitem{} 5.1.1 Summary. \dotfill 97
\itemitem{} 5.1.2 Background. \dotfill 97
\itemitem{} 5.1.3 Aims. \dotfill 101
5.2 Problem statement.
\itemitem{} 5.2.1 Inviscid formulation. \dotfill 102
\itemitem{} 5.2.2 Initial configuration. \dotfill 104
5.3 Solution by boundary integral method. \dotfill 109
5.4 Viscous effects.
\itemitem{} 5.4.1 Boundary layer approximation. \dotfill 112
\itemitem{} 5.4.2 Repositioning and smoothing. \dotfill 120
5.5 Results and discussion. \dotfill 121
\vskip 5pt
{\bf Chapter 6. Concluding remarks and future developments.}
6.1 Conclusions. \dotfill 148
6.2 Future directions.
\itemitem{} 6.2.1 Bubble/free-surface interactions. \dotfill 152
\itemitem{} 6.2.2 Film rupture and foam production. \dotfill 154
\itemitem{} 6.2.3 Cell motion. \dotfill 154
\vskip 5pt
{\bf Appendix A: Derivation of integral equations for chapter 2.}\dotfill 156
\vskip 5pt
{\bf Appendix B: Details of interpolation scheme for chapter 2.}\dotfill 159
\vskip 5pt
{\bf Appendix C: Identities for chapter 5.}\dotfill 161
\vskip 5pt
{\bf References.}\dotfill 162
}

