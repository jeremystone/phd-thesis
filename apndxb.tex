\c {\bigrm Appendix B: Details of interpolation scheme}
\c {\bigrm for chapter 2.}
\vskip 15pt
This appendix gives the quadratic interpolation used to integrate over
elements neighbouring the observation node on the bubble, in the cases
indicated in the text. The same interpolation was used to give the
directions of normals and tangents at nodes and to give surface
derivatives of potentials and dipole strengths.

Since the distance between points is not, in general, uniform, a
Lagrangian interpolation polynomial is used, parameterised
with respect to a linear arc-length, $s$. We require the polynomial, $X_i(s)$,
to go through $(-d_{i-1},x_{i-1})$, $(0,x_i)$ 
and $(d_i,x_{i+1})$, where
$d_j=|\bd{x}_{j+1}-\bd{x}_j|$ is the distance between nodes.
To avoid lengthy expressions, define the functions
${\cal A}_a$, $(a=1,2)$, acting on three consecutive members of an
ordered set of scalars or vectors, by
$${\cal A}_1(\{x_{i-1},x_i,x_{i+1}\})=d_{i-1}^2x_{i+1}-(d_{i-1}^2-d_i^2)
x_i-d_i^2x_{i-1},$$
and
$${\cal A}_2(\{x_{i-1},x_i,x_{i+1}\})=d_ix_{i-1}-(d_{i-1}+d_i)x_i+
d_{i-1}x_{i+1}.$$
For convenience, write these as ${\cal A}_1(x_i)$ and ${\cal A}_2(x_i)$
respectively.
It turns out that $X_i$ is given by
$$X_i(s)={1\over D_i}\left(s^2{\cal A}_2(x_i)+s{\cal A}_1(x_i)\right)+x_i,$$
where $D_i=d_id_{i-1}(d_i+d_{i-1})$.

In particular, we have that
$$\bd{q}_i(s)={1\over D_i}\left(s^2{\cal A}_2(\bd{x}_i)+
s{\cal A}_1(\bd{x}_i)\right)+\bd{x}_i.$$
From this, we immediately see that the tangent at node $i$ is in the
direction ${\cal A}_1(\bd{x}_i)$ and hence the normal,
which we denote by ${\cal A}_1^\bot(\bd{x}_i)$, may be found.
We are now in a position to express the integrals in terms of the
interpolation polynomials.
\pg
\hbox{(i) Integrals of the form -}
$$\int_C\bigl(f(\bd{x}')\bh{n}(\bd{x})-
f(\bd{x})\bh{n}(\bd{x}')\bigr)\cdot
{\bd{x}'-\bd{x}\over|\bd{x}'-\bd{x}|^2}ds',$$
where $f$ represents some quantity such as the dipole strength.
For the interval of integration where
$\bd{x}'\in[\bd{x}_{i-1},\bd{x}_{i+1}]$,
we get
$$\eqalign{\int_{-d_{i-1}}^{d_i}
&\left\{\left[
{s^2{\cal A}_2(f_i)\over D_i}+{s{\cal A}_1(f_i)\over D_i}+f_i
\right]
{{\cal A}_1^\bot(\bd{x}_i)\over|{\cal A}_1(\bd{x}_i)|}-
f_i
\left[
{2s{\cal A}_2^\bot(\bd{x}_i)+{\cal A}_1^\bot(\bd{x}_i)\over
|2s{\cal A}_2(\bd{x}_i)+{\cal A}_1(\bd{x}_i)|}
\right]\right\}\cr
\cdot &D_i\left[
{s^2{\cal A}_2(\bd{x}_i)+s{\cal A}_1(\bd{x}_i)\over
|s^2{\cal A}_2(\bd{x}_i)+s{\cal A}_1(\bd{x}_i)|^2}
\right]
\left|{d\bd{q}\over ds}\right|ds.}$$
Noting that ${\cal A}_a(\bd{x}_i)\cdot{\cal A}_a^\bot(\bd{x}_i)=0$,
$(a=1,2)$, a factor of $s^2$ cancels and we are left with
$$\eqalign{\int_{-d_{i-1}}^{d_i}
&{|2s{\cal A}_2(\bd{x}_i)+{\cal A}_1(\bd{x}_i)|
\over D_i|s{\cal A}_2(\bd{x}_i)+{\cal A}_1(\bd{x}_i)|^2}
\left[
\left(s^2{\cal A}_2(f_i)+s{\cal A}_1(f_i)+D_if_i\right)
{{\cal A}_1^\bot(\bd{x}_i)\cdot{\cal A}_2(\bd{x}_i)
\over|{\cal A}_1(\bd{x}_i)|}
\right]\cr
&-f_i
\left[
{2{\cal A}_2^\bot(\bd{x}_i)\cdot{\cal A}_1(\bd{x}_i)+
{\cal A}_1^\bot(\bd{x}_i)\cdot{\cal A}_2(\bd{x}_i)
\over|s{\cal A}_2(\bd{x}_i)+{\cal A}_1(\bd{x}_i)|^2}
\right]ds.}$$
Substituting the values for the ${\cal A}$'s into this, the integrand
and hence the integral's value may be calculated.
\vskip 15pt
\hbox{(ii) Integrals of the form -}
$$\int_C\bigl(f(\bd{x}')\bh{t}(\bd{x})-
f(\bd{x})\bh{t}(\bd{x}')\bigr)\cdot
{\bd{x}'-\bd{x}\over|\bd{x}'-\bd{x}|^2}ds'.$$
This integral is slightly more difficult in that the factor of $s^2$
does not explicitly cancel, though the integrand is still bounded
as $s\to0$. For the interval of integration
where $\bd{x}'\in[\bd{x}_{i-1},\bd{x}_{i+1}]$, we write it as
$$\eqalign{\int_{-d_{i-1}}^{d_i}
&\left\{\left[
{s^2{\cal A}_2(f_i)\over D_i}+{s{\cal A}_1(f_i)\over D_i}+f_i
\right]
{{\cal A}_1(\bd{x}_i)\over|{\cal A}_1(\bd{x}_i)|}-
f_i
\left[
{2s{\cal A}_2(\bd{x}_i)+{\cal A}_1(\bd{x}_i)\over
|2s{\cal A}_2(\bd{x}_i)+{\cal A}_1(\bd{x}_i)|}
\right]\right\}\cr
&\cdot D_i
\left[
{s^2{\cal A}_2(\bd{x}_i)+s{\cal A}_1(\bd{x}_i)\over
|s^2{\cal A}_2(\bd{x}_i)+s{\cal A}_1(\bd{x}_i)|^2}
\right]
\left|{d\bd{q}\over ds}\right|ds.}$$
The term in the first factor
that does not contain an $s$ in the numerator is equal to
$${f_i\over D_i}{\cal A}_1(\bd{x}_i)\left|{d\bd{q}\over ds}\right|^{-1}
\left[{|2s{\cal A}_2(\bd{x}_i)+{\cal A}_1(\bd{x}_i)|\over
|{\cal A}_1(\bd{x}_i)|}-1\right],$$
and is clearly of order $s$ as $s\to0$, since the tangents at the observation
point and the integration point become equal at $s=0$. If the quadrature
routine evaluates the integrand at very small values of $s$, it would be
worthwhile expanding this difference in powers of $s$ in order to
avoid division by small quantities.

