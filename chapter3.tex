\vbox{
\c{\bigrmb Chapter 3.}
\vskip 1cm
\c{\bigrm A TWO-DIMENSIONAL BUBBLE}
\c{\bigrm NEAR A FREE-SURFACE.}
\vskip 15pt
\hbox{\bf 3.1 Introduction.}
\vskip 5pt
}
This chapter considers the motion of a two-dimensional bubble below 
a free gas/liquid interface. Free-boundary problems of this nature may be
dealt with using boundary integral  techniques  similar  to  those
described in the previous chapter. The situation is, however,
complicated slightly 
by the need to evaluate line integrals over the infinite free-surface.

There are many examples of numerical solutions to free-surface problems
in the literature. An early use of the Green's formula boundary integral 
method was by Longuet-Higgins and Cokelet (1976). They made a significant
advance in the problem of the computation of breaking waves by being
able, for the first time, to calculate the motion well after the 
interface had become vertical. Later, Baker, Meiron and Orszag (1982)
developed vortex and dipole distribution methods which they used
to tackle the same problem.
Using a vortex method similar to that of Baker et al (1982), Oguz
and Prosperetti (1990b) were able to obtain some beautiful pictures
of the jet formed after a drop had impacted on a free-surface.

Vortex methods can be employed to calculate
the rolling up of a vortex sheet
(Moore, 1973; Fink and Soh, 1978). These ideas
can then be extended to study the response of a free-surface
to a vortex sheet below it (Tryggvason, 1988).

There have also been a number of studies in the field of
cavitation bubble dynamics. Blake and Gibson (1981) used a 
discrete ring-source distribution to approximately represent the
cavitation  bubble  and  adjacent  free-surface  and  follow  the
bubble's growth and collapse. They were able to
predict motions that compare well with the experimental results 
given in their paper. More recently, Blake, Taib and Doherty (1987) used
a full boundary integral technique to solve the same problem.

In order to solve for the motion of the bubble using a boundary
integral method, it is necessary to discretise the integral equations
into a set of algebraic equations. To do this, the relevant 
quantities such as position and potential have to be known at a 
finite number of points along the bounding surfaces of the solution
domain. Thus it can be appreciated that, as mentioned above,
difficulties arise when one wishes to integrate over the free-surface.
If the 
bubble is situated in a finite container, it is possible
to integrate over the container walls 
and finite free-surface, thus simplifying the 
task significantly.
One way of overcoming the problem of an infinite free-surface
was used for the axisymmetric case
by Oguz and Prosperetti (1989) by integrating out for a finite 
radius and then matching on to a simple first-order expansion for
the potential and surface elevation. 
If the velocities at the 
free-surface are small then a linearised boundary
condition, namely that the potential vanishes there,
may be used. This was one of the approaches taken by Blake and 
Gibson (1981), and was achieved by using the appropriate 
image of the free-space Green's function in the $z=0$ plane.
For problems with periodic boundary conditions on the free-surface,
the integration can be simplified analytically by 
summing over all identical parts (Baker et al, 1982).
Alternatively a conformal map may be used as in Longuet-Higgins
and Cokelet (1976), to transform all periodic parts onto 
a closed contour.

The method described here is a conformal mapping technique 
that maps the bubble beneath the 
flat free-surface onto a bubble within the unit circle.
The results obtained are compared with a first order perturbation
expansion for the problem of a submerged dipole.
\pg
\hbox{\bf 3.2 Numerical approach.}
\nobreak
\vskip 15pt
\c{\it 3.2.1 Problem formulation.}
\nobreak
\vskip 5pt
Unlike the study of the growth and collapse of cavitation bubbles
where a bubble can just appear in the fluid near to the surface, 
in the case of bioreactors described in the introductory chapter, the
bubbles rise a large distance from a sparger before reacting 
with the interface.
Since the two-dimensional bubble considered in the previous chapter
breaks up after a time of the order of
$t=4(a/g)^{1/2}$ seconds
(where $a$ is the initial radius of the bubble) and takes about
$(a/g)^{1/2}$ seconds to rise one bubble radius
(Baker and Moore, 1989), 
a bubble that starts much further away from the free-surface than about
$4$ bubble radii will split up before getting close. Bearing this 
in mind, some of the example results given at the end of the 
chapter for larger bubbles are somewhat unphysical.
However for the purposes of 
studying the behaviour of bubbles in bioreactors where 
the bubble sizes are of the order of a millimeter, surface 
tension will act so as to keep the bubble shape roughly elliptical
and will prevent jetting and breakup; thus the example for a
bubble with a radius of one millimeter, is more realistic.
It is for this reason that it was deemed necessary to include 
surface tension into the model in this chapter.

As for the bubble in an infinite fluid, we wish to solve for the 
velocity field as the gradient of a potential, $\phi$. The kinematic
condition is as before, namely that particles on the bubble and 
free-surface remain there throughout the motion. The dynamic condition
is again from Bernoulli's theorem
$${p_\infty}=p+{1\over2}\rho|\bd{u}|^{2}+
\rho{\partial\phi\over\partial t}+\rho gy,\eqno(3.2.1)$$
where $p_\infty$ is the fluid pressure at infinity just below the 
gas/liquid interface. 
The surface tension introduces a discontinuity into the pressure
field, dependent on the curvature of an interface. For the 
case considered here, a two-dimensional bubble may be thought of
as being between two sheets of glass, thus 
at a point on the plane midway between the sheets there is 
curvature in two directions: $\kappa$ tangential to the 2-D 
representation and $\kappa^\star$ normal to the glass. The
latter value is assumed constant.
Both curvatures
are taken as positive if the corresponding surface is concave
on the gas side.
So the pressure just below the free-surface is given by
$$p=p_{atm}-\sigma(\kappa+\kappa^\star),\eqno (3.2.2)$$
where $\sigma$ is the surface tension coefficient of the gas/liquid
interface, and $p_{atm}$ is atmospheric pressure. 
Since $\kappa=0$ at infinity, (3.2.2) evaluated at the free-surface gives
$$p_\infty=p_{atm}-\sigma\kappa^\star.\eqno (3.2.3)$$
Using (3.2.2) and (3.2.3) in (3.2.1),
the Bernoulli pressure condition at the free-surface 
can be written as
$${1\over2}\rho|\bd{u}|^{2}+
\rho{\partial\phi\over\partial t}+\rho gy-\sigma\kappa=0.\eqno(3.2.4)$$
The pressure balance at the bubble surface is
$$p=p_b(t)-\sigma(\kappa+\kappa^\star),\eqno (3.2.5)$$
where $p_b(t)$ is the pressure inside the bubble.
Assume that the bubble is initially circular with radius $a$,
at rest and at a distance $h$ below the liquid surface.
Evaluating (3.2.5) and (3.2.1) at $t=0, y=-h$ 
(where use is made of the Walters and Davidson (1962) expression for the 
potential for small times, (2.8.10)) and eliminating $p$
we find that the pressure inside the bubble is initially given by
$$p_b(0)=p_\infty+\rho gh+\sigma\left(
{1\over a}+\kappa^\star\right).\eqno(3.2.6)$$
Substituting for $p$ and $p_\infty$ in (3.2.1) using (3.2.5) and 
(3.2.6) gives
the pressure condition at the bubble surface,
$${1\over 2}\rho|\bd{u}|^2+
\rho{\partial\phi\over\partial t}+\rho g(y+h)+
\sigma\left({1\over a}-\kappa\right)+p_b(t)-p_b(0)=0.
\eqno(3.2.7)$$

It is convenient to scale lengths with respect to $a$,
times with respect to $(a/g)^{1/2}$ and pressures with respect to
$\rho ga$.
On applying the substantial derivative to (3.2.4) and (3.2.7), we get
for the free-surface,
$${D\phi\over Dt}={1\over 2}|\bd{u}|^{2}-y+
{4\kappa\over E_o},
\eqno (3.2.8)$$
and for the bubble,
$${D\phi\over Dt}={1\over 2}|\bd{u}|^{2}-(y+\gamma_1)+
{4\over E_o}(\kappa-1)+p_b(0)-p_b(t),\eqno (3.2.9)$$
where $\gamma_1=h/a$ is the initial dimensionless distance
beneath the surface and we introduce the E\"otvos number,
$E_o={4\rho ga^2/\sigma}$.
Note that $\kappa$ is now the
non-dimensional curvature in the plane of the bubble.
Writing the dynamic boundary condition in terms of the initial
conditions, ensures that changes of the potential for each surface will be
initially centred on zero, thus reducing round-off
errors, especially for very small or deeply submerged bubbles.

The solution domain of the problem is the semi-infinite region 
of the complex $z$-plane below the free-surface, $S$, that is initially the
real axis. The domain is bounded internally by the closed curve, $B$,
representing the bubble. This region represents the area covered by the 
liquid and is denoted by $\Omega_-$. Its complement in the complex plane
is the gas of the atmosphere and bubble and is denoted $\Omega_+$.
For convenience, write $C=S\cup B$. Normals to $C$ are taken into $\Omega_+$.
\vskip 15pt
\c{\it 3.2.2 Solution by boundary integral method.}
\vskip 5pt
The conformal mapping
$$\zeta=-\left({z+i\gamma_1\over z-i\gamma_1}\right),\eqno(3.2.10)$$
transforms the unbounded region, $\Omega_-$, of the $z$-plane onto
a finite region, $\tilde\Omega_-$, in the \hbox{$\zeta$-plane}, bounded
externally by the closed contour, $\tilde S$, initially the unit circle
centred on the origin, $\tilde O$, and internally by the image of the 
bubble, $\tilde B$.

If $z=x+iy$ and $\zeta=\xi+i\eta$, then (3.2.10) gives 
the following expressions transforming co-ordinates:
$$y=\gamma_1{\xi^2-1+\eta^2\over(\xi+1)^2+\eta^2};\qquad
x={-2\gamma_1\eta\over(\xi+1)^2+\eta^2}$$
and
$$\xi=-{y^2-\gamma_1^2+x^2\over(y-\gamma_1)^2+x^2};
\qquad\eta={-2\gamma_1 x\over(y-\gamma_1)^2+x^2}.$$

If we denote the complex potential in the $\zeta$-plane by $\tilde W$,
then we have the relationship $\tilde W(\zeta,t)=W\bigl(z(\zeta),t\bigr)$,
and as the map is conformal in the solution domain, $\nabla^2\tilde\phi=0$, where 
$\tilde\phi=\Re[\tilde W]$.
Hence we may apply Green's formula to $\tilde\phi$.

In order to determine the value of the $p_b(0)-p_b(t)$ term of (3.2.9), we
introduce the quantity,
$k(t)=\int_0^t(p_b(t')-p_b(0))dt'$, and write the boundary 
values of the potential as
$$\tilde f(\zeta,t)=\tilde\phi(\zeta,t)+\cases{k(t),
&$\zeta\in \tilde B,$\cr 0,&$\zeta\in \tilde S.$}$$
Once we insist that the bubble is to remain of constant volume,
the problem becomes equivalent to a modified Dirichlet boundary 
value problem. This is one in which
the potential on each disconnected surface is known only
up to a constant which may depend on the particular 
surface; we are free to choose one of the constants to be zero
(see for example Muskhelishvili, 1953). The constants are determined 
by the side condition that the potential is the real part of a
complex function analytic in the solution domain. In a 
fluid dynamics context, this implies
that the flow field has no circulation or 
expansion due to a multivalued potential or stream function 
respectively.

Evaluating Green's formula (2.2.6),
where the integration is carried out over the surface 
$\tilde C=\tilde S\cup\tilde B$,
for $\zeta^\star\in\tilde\Omega_-$, $\zeta\in \tilde C$ gives
$$\tilde\phi(\zeta^\star)=\int_{\tilde C}\left(G(\zeta^\star,\zeta')
{\partial\tilde\phi\over\partial n}(\zeta')-
\bigl(\tilde\phi(\zeta')-\tilde\phi(\zeta)\bigr)
{\partial G\over\partial n'}(\zeta^\star,\zeta')\right)ds'-
\tilde\phi(\zeta)\int_{\tilde C}{\partial G\over\partial n'}
(\zeta^\star,\zeta')ds',\eqno(3.2.11)$$
where, as in the previous chapter, $G$ is the free-space Green's
function for Laplace's equation.
The last integral of (3.2.11) is equal to 
the angle subtended at $\zeta^\star$,
by $\tilde S$ divided by $2\pi$, and is thus equal to $-1$.

In practice, we use the boundary values of the potential
in terms of the function $\tilde f$, defined above, and work 
with $\tilde f$ entirely. Taking the limit of (3.2.11) as 
$\zeta^\star\rightarrow\zeta$ and
writing in terms of $\tilde f$ gives
$$\int_{\tilde C}\left(G{\partial\tilde\phi\over\partial n}(\zeta')-
\bigl(\tilde f(\zeta')-\tilde f(\zeta)\bigr)
{\partial G\over\partial n'}\right)ds'
+\cases{k,&$\zeta\in \tilde B,$\cr 0,&$\zeta\in \tilde S$}=0.\eqno(3.2.12)$$
The side condition that the complex potential is analytic
manifests itself, as suggested above, in the form of an integral 
requiring that the total flow out of the bubble vanishes
$$\int_{\tilde B}{\partial\phi\over\partial n}ds=0.\eqno(3.2.13)$$
The dynamic conditions for $\tilde f$, from (3.2.8) 
and (3.2.9) are:
for the free-surface
$$\eqalignno{
{D\tilde f\over Dt}=&{1\over 2}|\bd{u}|^{2}-y+
{4\kappa\over E_o},&(3.2.14)\cr
\noalign{\hbox{and for the bubble}}
{D\tilde f\over Dt}=&{1\over 2}|\bd{u}|^{2}-(y+\gamma_1)+
{4\over E_o}(\kappa-1).&(3.2.15)}$$

In order to update the position of a discrete set of
Lagrangian points taken on $\tilde C$,
the velocities are required in the $\zeta$-plane. Consider the 
identity
$${D\zeta\over Dt}={d\zeta\over dz}{Dz\over Dt}.$$
As the second factor on the right-hand side
is just the Lagrangian velocity of points in the
$z$-plane, which is simply the gradient of the potential, we have
$$\eqalign{{D\zeta\over Dt}=&{d\zeta\over dz}\left({dW\over dz}\right)^*
\cr =&\left|{d\zeta\over dz}\right|^2\left({d\tilde W\over 
d\zeta}\right)^*,}\eqno(3.2.16)$$
where an asterisk denotes a complex conjugate.
Hence using (3.2.10) to evaluate the first factor of (3.2.16)
and taking
components, it is easily seen that the normal and tangential particle
velocities in the $\zeta$-plane are
$${1\over 4\gamma_1^2}|\zeta+1|^4{\partial\tilde\phi\over\partial n}
\eqno(3.2.17)$$
and
$${1\over 4\gamma_1^2}|\zeta+1|^4{\partial\tilde\phi\over\partial s}
\eqno (3.2.18)$$
respectively, where here $\partial/\partial n$ and
$\partial/\partial s$ represent derivatives
taken normally and tangentially in the $\zeta$-plane. The
normal, $n$, and tangent, $\tau$, are oriented relative to each other 
in the sense $n=i\tau$.

In order to evaluate the Bernoulli pressure condition, it is
necessary to determine the value of the $|\bd{u}|^2/2$ term.
Here we make use of the fact that
$$\left|{dW\over dz}\right|^2=\left|{d\zeta\over dz}\right|^2
\left|{d\tilde W\over d\zeta}\right|^2,$$
and so from (3.2.10) above,
$$|\bd{u}|^2={4\gamma_1^2\over |\zeta+1|^4}\left|
{D\zeta\over Dt}\right|^2.\eqno (3.2.19)$$

Computationally, the method used here is essentially the same as
that described in section 2.5. The surfaces in the $\zeta$-plane
are discretised using $N_B$ and $N_S$ nodes on the images of the bubble
and free-surface respectively. 
The only major difference is that when updating positions 
of nodal points and values of potentials
in the $\zeta$-plane, (3.2.17), (3.2.18) and 
(3.2.19) are used to ensure that the correct kinematic and dynamic 
conditions are obeyed by the $z$-plane values.
By fitting quadratics to three adjacent points
on the surfaces in the $z$-plane, arc-length derivatives
can be found which give in turn tangents
and hence the curvatures for use in the dynamic 
conditions (3.2.14) and (3.2.15).
This `5-point' method for finding curvatures gives
reasonable results provided that there are not surface 
oscillations on the scale of the segment lengths, which may be
avoided by using a sufficient number of nodes together with frequent 
application of the Longuet-Higgins and Cokelet (1976) smoothing formula.

Due to the conformal mapping,
repositioning is done in a slightly different way to that in 
Chapter 2. Here, points on the free-surface are spaced out evenly with
respect to the $\zeta$-plane, whereas the repositioning 
of the bubble surface is done in the $z$-plane. This ensures
that the resolution of the free-surface is not lost for the 
purposes of the calculations, but also has the effect that the output
bubble shapes have points evenly spaced. Similar comments apply to the
initial nodal spacings.
\pg
\hbox{\bf 3.3 Perturbation expansion approximation.}
\vskip 5pt
Analytical solutions have been developed for many free-surface problems.
One of the most well-known examples being the simple but successful 
solution of the linearised equations for small amplitude gravity waves.
Many more complicated problems have been studied, for instance
Havelock (1927) considered the problem of a circular 
cylinder submerged in a uniform stream and its effect on the 
surface elevation. He did this by first using a dipole
and its image in the flat, surface, together with a 
distribution of dipoles behind this image. For the case of a 
horizontal dipole, he reflected this image system back in the 
cylinder to yield a second approximation, reinforcing the boundary
condition at the cylinder itself.

Many free-stream problems in two-dimensional potential flow
can be greatly simplified by considering an inverse problem
formulation. That is, by writing the position in space as  
a function of a complex potential, $W=\phi+i\psi$, 
rather than the other way around. With this scheme, 
steady free-surfaces become the straight lines $\psi=const$.
Problems such as the free-stream flow past a finite, flat plate
or flow out of an orifice which would otherwise be very
difficult, may be tackled in a straightforward manner by
using appropriate conformal maps. For example the 
hodograph transformation, $w=\log(dz/dW)$, can be used  to  map 
steady,
free-surfaces onto lines $\Re[w]=const$ and straight, rigid surfaces onto
lines $\Im[w]=const$. Following this by a Schwartz-Cristoffel
transformation to map the resulting straight-sided figure onto 
the upper half-plane, yields a problem that can be easily solved.
By considering the steady 
Bernoulli equation, written in terms of inverse variables,
evaluated at $\psi=0$, Vanden Broeck, Schwartz and Tuck (1978),
developed a power series, in the square of a Froude number,
for position as a function of potential. In this way they were
able to generate terms in the asymptotic expansions 
far more easily than with more direct methods. 
They considered some examples of free-surface flows one of which
is the response of a gas/liquid interface to a submerged source or sink.
The results obtained show a stagnation point above the source. 
However for somewhat larger Froude numbers, such solutions do not 
exist. Tuck and Vanden Broeck (1984) and King and Bloor (1988)
found solutions with a downward pointing cusps for larger Froude numbers. 
Forbes and Hocking (1990) considered the same problem and compared 
the results from a regular expansion in terms of the square of 
a Froude number with a steady, boundary integral solution.

In order to approximate the effect of a rising bubble on a free-surface,
consider the problem of finding the surface elevation as
a function of time due to a  dipole  moving  far  below  the
surface. We may expect this to give reasonable agreement with the 
numerical result since for a bubble of constant volume,
the dipole term is dominant in the far field (see (2.2.13)).

With the same scalings as in section 3.2, assume that the dipole
accelerates from rest from a point a distance
$\gamma_1=1/\epsilon$ below the surface,
for $\epsilon\ll 1$ and let $\gamma(t)>0$ be the distance of the
dipole below the $x-$axis as a function of time.
Re-defining $\Omega_-$ to be the whole of the semi-infinite region below the 
surface, the 
problem can be formulated as follows. Laplace's equation must be 
satisfied by the potential everywhere except at the dipole, 
namely
$$\nabla^2\phi=0,\quad\hbox{in}\quad\Omega_-\setminus
\{(0,-\gamma(t))\},\eqno(3.3.1)$$
with 
$$\phi(\bd{r})\sim-U(t){y+\gamma(t)\over x^2+(y+\gamma(t))^2},
\quad\hbox{as}\quad\bd{r}\rightarrow 
(0,-\gamma(t)),\eqno(3.3.2)$$
where $U(t)=-d\gamma/dt$ is the dipole's rise speed and
$U(0)=0$. Initially, the surface, given by the curve 
$y=\eta(x,t)$, is flat and the fluid occupying $\Omega_-$ is stationary.

The first boundary condition is the usual kinematic condition, this 
time in the Eulerian form
$${D\eta\over Dt}=\bh{\j}\cdot\nabla\phi\Big|_{y=\eta},$$
where $\bh{\j}$ is the unit vector in the $y$ direction.
This can be written more conveniently as 
$${\partial\eta\over\partial t}+{\partial\phi\over\partial x}
\Big|_{y=\eta}{\partial\eta\over\partial x}
={\partial\phi\over\partial y}\Big|_{y=\eta}.\eqno(3.3.3)$$
If we assume that the curvatures on the surface
remain small, the pressures across this interface 
can be assumed to be continuous and the dynamic condition (3.2.4) gives, in 
dimensionless variables,
$$\left({1\over 2}|\nabla\phi|^2+{\partial\phi\over\partial t}
\right)\Big|_{y=\eta}+\eta=0.\eqno(3.3.4)$$

Since $\gamma=O(\epsilon^{-1})$, ${(y+\gamma)/
[x^2+(y+\gamma)^2]}=O(\epsilon)$, as $\epsilon\rightarrow 0$
for fixed $x$ and $y$, and so we expect to be able to expand
the potential and surface elevation in the form
$$\eqalignno{
\phi(x,y,t)=&\epsilon\phi_1(x,y,t)+\epsilon^2\phi_2(x,y,t)+
O(\epsilon^3),&(3.3.5)\cr
\eta(x,t)=&\epsilon\eta_1(x,t)+\epsilon^2\eta_2(x,t)+
O(\epsilon^3),&(3.3.6)}$$
with the first order potential given by the solution to
$$\nabla^2\phi_1=0,\quad\hbox{in}\quad\Omega_-
\setminus\{(0,-\gamma(t))\},\eqno(3.3.7)$$
and
$$\phi_1(\bd{r})\sim-{U(t)\over\epsilon}{y+\gamma(t)\over 
x^2+(y+\gamma(t))^2},
\quad\hbox{as}\quad\bd{r}\rightarrow (0,-\gamma(t)),
\eqno(3.3.8)$$
and higher orders satisfying
$$\nabla^2\phi_i=0,\quad\hbox{in}\quad\Omega_-,\quad i\ge 2.
\eqno(3.3.9)$$
Since this is a free-boundary perturbation problem, Taylor 
expand each of the terms in (3.3.5) about $y=0$, and substitute for 
$\eta$ as defined in (3.3.6). This gives
$$\phi(x,\eta,t)=\epsilon\phi_1\Big|_{y=0}+\epsilon^2\left(
\eta_1{\partial\phi_1\over\partial y}+\phi_2\right)\Big|_{y=0}
+O(\epsilon^3).\eqno(3.3.10)$$
By differentiating (3.3.5) it is clear that we can write down similar
expressions for $\partial\phi/\partial t$, $\partial\phi/\partial x$
and $\partial\phi/\partial y$ at the surface.
Substituting (3.3.6) and (3.3.10) into the kinematic and dynamic conditions 
(3.3.3) and (3.3.4), produces the first order, linearised boundary conditions
$${\partial\phi_1\over\partial t}\Big|_{y=0}+\eta_1=0,\eqno(3.3.11)$$
and 
$${\partial\eta_1\over\partial t}
={\partial\phi_1\over\partial y}\Big|_{y=0}.\eqno(3.3.12)$$
The second order boundary conditions are given by
$$\left\{\left({\partial\phi_1\over\partial x}\right)^2
+\left({\partial\phi_1\over\partial y}\right)^2
+2\left(\eta_1{\partial^2\phi_1\over\partial y\partial t}
+{\partial\phi_2\over\partial t}\right)\right\}\Bigg|_{y=0}
+2\eta_2=0,$$
and
$${\partial\eta_2\over\partial t}+
{\partial\phi_1\over\partial x}\Big|_{y=0}{\partial\eta_1\over\partial x}=
\eta_1{\partial^2\phi_1\over\partial y^2}\Big|_{y=0}+
{\partial\phi_2\over\partial y}\Big|_{y=0},$$
but here, we restrict ourselves to solving only the first order problem.

Combining equations (3.3.11) and (3.3.12) gives a condition for the first 
order potential
$$\left({\partial^2\phi_1\over\partial t^2}
+{\partial\phi_1\over\partial y}\right)\Big|_{y=0}=0.\eqno(3.3.13)$$
To solve the first order problem posed by equations (3.3.7), 
(3.3.8) and (3.3.13),
apply a technique analogous to that used by Havelock (1927) for a
similar problem by writing $\phi_1$ as the sum
of a singular part,
$$\phi_S(x,y,t)=-{U(t)\over\epsilon}\left( 
{y+\gamma(t)\over x^2+(y+\gamma(t))^2}+ 
{y-\gamma(t)\over x^2+(y-\gamma(t))^2}\right),\eqno(3.3.14)$$
and an unknown harmonic part, $\phi_H$.
The same approach can be employed to find 
the response of a free-surface to the fundamental singularity 
of the Laplace equation and hence determine the Green's function;
see for example Wehausen and Laitone (1960), where many surface wave
problems are examined in detail.
Note that the second term in (3.3.14) makes the 
analysis slightly simpler by
anticipating the form of the final solution: in fact, from
(3.3.11), $\phi_S$
does not contribute to the surface elevation since 
$\phi_S\big|_{y=0}=0$.

We now take the Fourier transform with respect to $x$, ${\cal F}_x$,
and the Laplace transform with respect to t, ${\cal L}_t$, of the singular
potential, $\phi_S$, given by (3.3.14).
By consulting tables of transforms or using a simple 
application of the calculus of residues for an appropriate complex
contour, it can be seen that for $|y|<\gamma$
$$\hat\phi_S(q,y,t)\equiv{\cal F}_x[\phi_S]=-\sqrt{\pi\over 2}
{U(t)\over\epsilon}
\bigg(\exp\big(-|q|(y+\gamma(t))\big)-\exp\big(-|q|(\gamma(t)-y)\big)
\bigg).
\eqno(3.3.15)$$
Thus 
$${\partial\hat\phi_S\over\partial y}\Big|_{y=0}=\sqrt{2\pi}
{U(t)\over\epsilon}|q|e^{-|q|\gamma(t)}.\eqno(3.3.16)$$
If $\bar\phi_S(q,y,s)\equiv{\cal L}_t[\hat\phi_S]$ and
we define
$\bar h(q,s)={\cal L}_t[U(t)\displaystyle{e^{-|q|\gamma(t)}}]$, then we have
$${\partial\bar\phi_S\over\partial y}\Big|_{y=0}=
{\sqrt{2\pi}\over\epsilon}|q|\bar h(q,s).\eqno(3.3.17)$$
Similarly, transforming Laplace's equation, satisfied by $\phi_H$
throughout $\Omega_-$, gives
$${\partial^2\bar\phi_H\over\partial y^2}
-q^2\bar\phi_H=0,\eqno(3.3.18)$$
from which we may deduce that 
$$\bar\phi_H=\bar A(q,s)e^{|q|y},\eqno(3.3.19)$$
so that $\bar\phi_H\rightarrow 0$ as 
$y\rightarrow -\infty$, for all $q\ne 0$.
Transform the boundary conditions (3.3.13) to give
$$\left(s^2\bar\phi_1+{\partial\bar\phi_1\over\partial y}\right)
\Big|_{y=0}=0.\eqno(3.3.20)$$
In calculating the time transform we have assumed that,
since the fluid starts at rest, the potential is initially 
zero and, on using (3.3.11), we have taken ${\partial\phi_1/\partial 
t}(x,0,0)=0$ since
the surface is initially flat.

Substituting $\bar\phi_1=\bar\phi_S+\bar\phi_H$ into (3.3.20) and 
using (3.3.17) and (3.3.19) immediately gives us
$$\bar A(q,s)=-{\sqrt{2\pi}\over\epsilon}{|q|\over s^2+|q|}
\bar h(q,s).\eqno(3.3.21)$$
We may now use the convolution theorem for Laplace transforms
which, together with
the definition of the function $\bar h$, allows us to invert
the time transform for the harmonic potential given
by (3.3.19), to yield
$$\hat\phi_H(q,y,t)=-{\sqrt{2|q|\pi}\over\epsilon}e^{|q|y}\int_0^t
U(t')e^{-|q|\gamma(t')}\sin(\sqrt{|q|}(t-t'))dt'.
\eqno(3.3.22)$$
Proceeding with a direct inversion of the Fourier transform gives
$$\phi_H(x,y,t)=-\int_{-\infty}^\infty e^{iqx}{\sqrt{|q|}\over\epsilon}
e^{|q|y}\left(\int_0^t
U(t')e^{-|q|\gamma(t')}\sin(\sqrt{|q|}(t-t'))dt'
\right)dq.\eqno(3.3.23)$$
In order to simplify the appearance of (3.3.23)
slightly, we separate the intervals
$(-\infty,0)$ and $[0,\infty)$ of the $q$ integral
and interchange the order of integration to give, finally,
$$\phi_H(x,y,t)=-2\int_0^t {U(t')\over\epsilon}\left(\int_0^\infty
\sqrt{q}\exp(-q(\gamma(t')-y))\sin(\sqrt{q}(t-t'))\cos(qx)dq\right)
dt'.\eqno(3.3.24)$$
With the potential written in this form, we can easily use (3.3.11) 
to get an expression for the first order surface elevation, namely
$$\eta(x,t)=2\int_0^tU(t')\left(\int_0^\infty q 
e^{-q\gamma(t')}\cos(\sqrt{q}(t-t'))\cos(qx)dq\right)dt'+O(\epsilon^2).
\eqno(3.3.25)$$

Once the dipole speed $U(t)$ and hence depth $\gamma(t)$ are known,
the integrals in (3.3.25) can be calculated accurately numerically and
the assumption that the bubble rises with unit acceleration so that
$U=t$ and $\gamma=\gamma_1-t^2/2$, will give good results for small times.
However a small bubble flattens as it rises (see section 3.4 below)
and so its added mass increases and its acceleration decreases
leading to inaccuracies in the surface elevation.
Bearing this in mind, it is instructive to consider the expansion of 
the inner integral of (3.3.25) for
large $\gamma$. Since, in this case, the integral will be dominated by 
the contribution from a small region around $q=0$, 
substitute $q'=\gamma(t')q$
and expand the cosine term on the assumption that $t\ll\gamma^{1/2}$.
Retaining only the first term,
and using the fact that $U(t')dt'=-d\gamma$, we find that
the surface elevation is
$$\eta(x,t)=2\left[{\gamma'\over x^2+\gamma'^2}\right]
_{\gamma'=1/\epsilon}^{\gamma(t)}+O(\epsilon^2).
\eqno(3.3.26)$$
We thus can use (3.3.26) to approximate the surface elevation
without knowing the history of the dipole's motion, simply by using
the initial and current bubble centroid positions.
Equation (3.3.26) is also the result of substituting $\phi_S$, given by
(3.3.14) into (3.3.12).
\vskip 15pt
\hbox{\bf 3.4 Results and discussion.}
\vskip 5pt
The boundary integral code described in section 3.2  was  used  to 
solve the problem for various different E\"otvos numbers and initial 
distances from the free-surface. The first two figures show results 
for a bubble with $E_o=53$, equivalent to a $1cm$
bubble in water. Figure 3.1 is a
bubble that starts at rest with $\gamma_1=2$ bubble radii from the surface,
whereas figure 3.2 is with $\gamma_1=10$. The latter figure is a test case
where
it is clear that for large distances from the interface, the bubble moves 
as if it were in an infinite fluid. This also shows the effect of 
surface tension for this bubble is very small (compare with figure 
2.1). The effect of the free-surface for $\gamma_1=2$ is not very 
strong, but the jet broadens out earlier since the flow around the 
bubble is slightly slower. This is due to gravity acting on the 
free-surface thus preventing, to a certain extent, flow from above 
the bubble which is unhindered in the infinite fluid case.
Figure 3.3 shows a time sequence for a bubble with 
$E_o=0.53$, $\gamma_1=10$. Here the surface tension effect is much
larger and acts so as to keep the bubble almost elliptical, without any jet 
formation. This figure is more relevant to the long-term 
studies of bubbles in bioreactors.

Pressures are calculated in the same manner as in Chapter 2. The potentials
for a grid of points in the $z$-plane are calculated for each equivalent 
$\zeta$-plane position using 
$$\tilde\phi(\zeta)=\int_{\tilde C}\left(G
{\partial\tilde\phi\over\partial n}(\zeta')-
\tilde f(\zeta'){\partial G\over\partial n'}\right)ds'.\eqno(3.4.1)$$
In order to improve resolution around the bubble 
and free-surface, the pressure at these surfaces is calculated
explicitly using 
$$p(\zeta,t)=p_\infty+\cases{dk/dt-4(\kappa-1)/E_o+
\gamma_1,&$\zeta\in\tilde B,$\cr
-4\kappa/E_o,&$\zeta\in\tilde S.$}\eqno(3.4.2)$$
Streamlines are calculated by using the stream function, given by
$$\tilde\psi(\zeta)=\int_{\tilde C}
\left(-\theta(\zeta'-\zeta){\partial\tilde\phi\over\partial n}(\zeta')
+\tilde f(\zeta'){\partial G\over\partial s'}
(\zeta,\zeta')\right)ds',\eqno(3.4.3)$$
where $2\pi\theta(\zeta'-\zeta)$ is the angle between $\zeta'-\zeta$
and the $\xi$-axis.

Figure 3.4 shows pressure and streamline plots corresponding to the bubble
in figure 3.3 at time $t=5.5$. The lower pressures around the sides of the 
bubble, as a result of faster flow than at the top, can be seen. To reach an
equilibrium, the bubble has moved into an elliptical shape so that the 
curvatures at the sides allow for the correct jump between the fluid pressure
and the uniform bubble pressure. It is also clear from the pressure plot that 
there is a higher pressure region directly above the bubble. This acts to push
the free-surface up as the bubble approaches. The streamlines indicate
that the free-surface is moving up above the bubble and down and outwards to 
the left and right of the bubble.

In order to investigate the convergence of the solution 
of the discretised form of the integral equation (3.2.12)
as the number of nodal points increases, we consider
an example problem where the exact solution is known. This problem
is
$$\nabla^2\phi=0,\quad\hbox{in}\quad \Omega_-,\eqno(3.4.4)$$
with Dirichlet boundary conditions,
$$\eqalign{\phi&=0,\quad\hbox{on}\quad S,\cr
\phi&=-\left\{y+\gamma + {y-\gamma\over 1-4y\gamma}\right\},
\quad\hbox{on}\quad B.}\eqno(3.4.5)$$
As before, $S$ is the free-surface, taken here to be the
$x$-axis, and $B$ is the bubble, taken to be the unit circle 
centred at $(0,-\gamma)$.
The exact solution is given by
$$\phi(x,y)=-\left\{{y+\gamma\over x^2+(y+\gamma)^2}+
{y-\gamma\over x^2 + (y-\gamma)^2}\right\},\eqno(3.4.6)$$
which is clearly just a dipole, representing the bubble's
translation, together with its image in $S$. 
On the basis of the analysis in Section 3.3, 
equation (3.3.14) in particular, we would expect this to be
typical of the type of boundary value problem
solved by the numerical code when calculating
the bubble motion, and hence it should give us a practical
estimate for the rate-of-convergence.
\vskip 5pt
To measure convergence, the normal derivative
estimated on the transformed free-surface by the 
boundary integral method with $N_S=N_B=N$ nodes, denoted
$\tilde\psi_N$, is compared with that from the exact solution.
The exact normal derivative in the $z$-plane is, from (3.4.6),
$$\psi\Big|_S\equiv{\partial\phi\over\partial n}\Big|_S=
{4\gamma^2\over
(x^2+\gamma^2)^2}-{2\over x^2+\gamma^2},\eqno(3.4.7)$$
and is related to that in the 
$\zeta$-plane, $\tilde\psi$, by the expression
$$\tilde\psi=\left|{dz\over d\zeta}\right|\psi.\eqno(3.4.8)$$
We define the error, $E_N$, by
$$E_N=\max_{i=1,\ldots,N}\{\tilde\psi_N(x_i)-\tilde\psi(x_i)\}.\eqno(3.4.9)$$

Inspection of figure 3.5 shows that the error falls off as $N^{-2}$, 
as is the case for a corresponding second-kind 
integral equation using a piecewise linear collocation method (see
for example Atkinson et al, 1983). It was found in Chapter 2 that the 
1-norm condition numbers for the discretised
first-kind formulation, such as the 
Green's formula method used here, are significantly larger
than for second-kind 
vortex or dipole distribution methods. Thus small errors 
introduced at each time-step may be amplified faster using
this method. A source of such errors is the integration of
(3.2.14) and (3.2.15), particularly because of the difficulty involved in
accurately differentiating the surface shape twice to find the 
curvatures. Indeed, it is the inclusion of surface tension
that necessitates the use of smoothing.

Problems can occur when solving integral
equations with weakly singular kernels 
if elements become very close to nodes on
other parts of the surface.
If this happens on parts of the surfaces where the distances between
adjacent nodes are
relatively large, so that few quadrature points are used,
the integrals along the elements closest to the observation node 
will not be calculated accurately.
In the context of problems considered here,
this typically occurs in the cases 
when a small, circular bubble gets very close to the free-surface
or when a large bubble has developed a jet which is about to pinch 
off smaller bubbles. We find that the problem tends to manifest
itself by causing excessively
large values for the normal derivatives as solutions to the
integral equation. 
The time-step is chosen in the manner described in Chapter 2,
so that the maximum change in potential is restricted for each step.
From the dynamic conditions
(3.2.14) and (3.2.15), we see that large velocities
result in correspondingly large rates-of-change of potential,
and thus a small time-step, eventually causing 
the calculation to grind to a halt.
It is, however, important to note that for this particular application,
these problems occur in situations where
one may expect viscous effects to be important, namely
in the drainage of thin films. 
Thus the method will give good results to physical problems
provided that small enough elements are used
to ensure accuracy up to the point 
where the underlying assumptions of the model cease to be valid.

Figure 3.6 shows the  results  of  the  asymptotic  expansion
described in section 3.3, for the case $\gamma_1=10$.
The solid lines represent the 
surface elevations according to the analytic solution (3.3.26)
and the broken lines the solution from the boundary integral method. 
As expected, the curves compare best for small times when the bubble is 
further from the free-surface and when the surface elevation 
and velocity are small. As the bubble moves closer, the comparison is less satisfactory.
There are several reasons for inaccuracy at later times.
One factor is the fact that in the
above analysis we assume that the bubble is represented by a dipole, and
make no other attempt to ensure that the kinematic boundary conditions at 
the bubble are satisfied. Thus as it deforms  from  circular  or 
becomes much closer to the free-surface the simple dipole model starts to
break down. To get a better approximation, one could use 
the method of images to reinforce the boundary condition at the bubble on 
the assumption that it is a rigid cylinder (Havelock, 1927).
One may also consider taking more terms in the asymptotic expansion.
This should improve the comparison, however it should be emphasised
that this would only yield a better approximation to the solution of the 
problem of a {\it dipole's} effect on a free-surface and so could not
predict accurately the situation when the bubble becomes much
closer to the surface, for the reason pointed out above.

Figure 3.7 shows an example of the possible motion when a $1mm$ 
two-dimensional gas bubble bursts at a free-surface.
Unlike the corresponding 
calculation of Chapter 5, where a high speed jet is seen to
form, only a small bump can be observed here.
Much of the energy from the burst seems to propagate outwards in
the form of surface waves. Part of the reason for the lack of jet is
that, in two-dimensions, there is no component of the surface tension
force that in the axisymmetric case acts radially to
thin and lengthen a straight jet. For the same reason, the cavity formed by
the bubble collapses in a completely different way to the 
three-dimensional case where it is rapidly pulled inwards near to the
base by surface tension.


