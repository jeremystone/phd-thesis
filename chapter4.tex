\vbox{
\c{\bigrmb Chapter 4.}
\vskip 1cm
\c{\bigrm RISING AXISYMMETRIC GAS BUBBLES.}
\vskip 15pt
\hbox{\bf 4.1 Introduction.}
\vskip 5pt
}
The rise of air bubbles in water is one of the most common examples
of two-phase flow experienced in everyday life. It is not surprising,
therefore that a significant effort has been focused on trying
to understand some of the associated phenomena. From a mathematical
point of view, the complexity of the problem is due to the fact that
the bubble shape is itself an unknown. Even for the steady state
problem of a bubble rising at its terminal velocity
it is difficult to get accurate solutions to the general case 
except by using sophisticated numerical techniques. There are,
however, several known results and approximations for certain special 
cases. These and other topics
are covered in detail in, for instance, Clift, Grace and Weber (1978). As
an introduction, we give some of the more fundamental models here,
before going on to consider the numerical techniques used in this chapter.
\vskip 15pt
\c{\it 4.1.1 Analytical results.}
\vskip 5pt
For very small bubbles, with Reynolds number, 
$Re=2Ua/\nu$ ($U$ being the rise speed, $a$ the bubble radius and $\nu$
the kinematic viscosity),
much smaller than unity, viscous forces will dominate inertial forces and 
we can get a good picture of the flow field by assuming that it can be
represented by the Stokes flow equations. 
In the case when inertia is completely neglected, the 
gas bubble (or liquid drop) takes on a spherical shape
(see for example Taylor and Acrivos, 1964).

We may easily solve for the flow field around a spherical 
bubble. As the density and dynamic viscosity of the gas are
substantially less than those of
surrounding liquid, we may ignore the internal dynamics of the bubble
itself. 
For boundary conditions, assume that there is no flow across the bubble
surface, that
normal stresses are continuous, apart from a contribution due to
surface tension, and that tangential stresses vanish.
Were we dealing with a drop,
we would have to take into consideration
the effect of its internal circulation, driven by a non-zero
tangential stress at the interface.
In fact, it is straightforward to 
generalise the Stokes flow solution to a drop of arbitrary
density and viscosity for which the rigid sphere and bubble are the
extreme cases. 

To solve the problem, introduce a Stokes' stream function, $\psi$, and 
consider the equation $D^4\psi=0$ where 
$$D^2={\partial^2\over\partial r^2}+{\sin\theta\over 
r^2}{\partial\over\partial
\theta}\left({1\over\sin\theta}{\partial\over\partial\theta}\right),
\eqno(4.1.1)$$
subject to the conditions above and the far-field 
condition $\psi\sim-(Ur^2/2)\sin^2\theta$ as
$r\rightarrow\infty$. The solution obtained, it should be noted,
is not valid in the far field, where it can be shown that
if $r=O(aRe^{-1})$, the neglected inertia terms in the equations of motion
contribute to the same order as the retained viscous terms.

The drag, $D$, on the bubble can be determined directly by integrating
the stress over the bubble surface, namely
$$D=k_i\int_C\sigma_{ij}n_jdS,\eqno(4.1.2)$$
where $\sigma_{ij}$ is the stress tensor, $n_j$ is the inward normal
to $C$, the bubble surface, and $k_i$ is the unit vector in the
direction of motion. This immediately
gives the result $D=4\pi\mu Ua$. Often,
the drag coefficient, $C_D$, is quoted in text books on the subject. This is 
defined as the ratio of the drag to the dynamic pressure multiplied by
the projected area of the bubble in the direction of motion.
In practice, for non-spherical objects, 
this projected area is taken to be that of a sphere 
of equal volume. Thus
$$C_D={4\pi\mu Ua\over {1\over 2}\rho U^2 \pi a^2}={16\over Re}.
\eqno(4.1.3)$$
For an early account see, for example, Levich (1962).

The drag can also be used to estimate the terminal velocity, 
$U_0$,
by equating it to the buoyancy force $4\pi a^3\rho g/3$,
giving a value of $U_0=a^2 g/3\nu$. These results
agree well with experimental data for bubbles in
water which are of less than about $0.01cm$ in diameter.
The tendency for bubbles to deform slightly from spherical
is due to the effect of inertia (see for example Taylor and Acrivos, 1964).
By matching a low Reynolds number inner expansion to an Oseen outer
solution, for the general case of a drop,
Taylor and Acrivos (1964) derived expressions for the
drag and the shape deformation. 
They found an expression for the ellipticity, the ratio of cross-stream
to with-stream axes, $\chi$. This turns out to be an expression
in terms of the Weber number, $W=2a\rho U^2/\sigma$, which is based on
the ratio of inertial pressure variations to the pressure jump due
to surface tension.
This is surprising since inertial effects are only small, and one
would expect the shape to depend on a ratio of the viscous stress to the
surface tension pressure jump, in terms of a capillary number,
$Ca=\mu U/\sigma$. If, however, as pointed out by Harper (1972),
the ellipticity 
of a bubble or drop with fore and aft symmetry depended on the 
capillary number at first order, the axis ratio would not be invariant
under replacement of $U$ by $-U$.
Taylor and Acrivos' (1964) expression for the ellipticity is
given by $\chi=1+5W/32+O(W^2/Re)+O(WRe)$.

For large Reynolds numbers, Levich (1949) estimated the 
boundary layer thickness, $\delta$, around a spherical bubble to be of
order $(\nu a/U)^{1/2}=a/Re^{1/2}$ --- as is
the case for a solid sphere. Thus
it is appropriate to model such high Reynolds number flows in the usual
way, by considering all vorticity to be confined to a thin boundary layer
at the bubble's surface. 
Levich (1949) also showed that as the surface is stress free, the
normal derivative of the velocity must change by $O(U/a)$, so that the
velocity change due to the boundary layer is $O(\delta U/a)$, as
opposed to $O(U)$ in the solid sphere case. As $\delta/a\ll 1$ the
boundary layer perturbation to the velocity is much smaller than
the irrotational velocity, whereas velocity gradients are of the same order.
As the volume of the boundary layer is much smaller than the volume of
the region surrounding the bubble where these gradients are of the
same order, the first order energy dissipation rates for the 
potential flow and the exact flow for large Reynolds number are 
equal. The drag can thus be calculated by considering the 
energy dissipation rate for the potential flow case.

Solving Laplace's equation for
a sphere moving at speed $U$ through a fluid at rest at infinity,
we get the well-known result $\phi=-(a^3/2r^2)U\cos\theta$.
Since for irrotational flow $\nabla\bd{u}$ is already symmetric,
this is just the rate-of-strain tensor and the
energy dissipation rate, integrated over an infinite volume, $V$,
bounded internally by the bubble, can be written as
$$\eqalign{\Phi&=2\mu\int_V{\partial u_i\over\partial x_j}
{\partial u_i\over\partial x_j}dV,\cr
&=\mu\int_{\partial V}{\partial q^2\over\partial n}dS,}
\eqno(4.1.4)$$
on using the divergence theorem, where $q=|\bd{u}|$.
Equating $\Phi$, given by (4.1.4), to $UD$,
where $D$ is the drag, and substituting
$\bd{u}=\nabla\phi$ gives 
$$D=12\pi\mu Ua,\eqno(4.1.5)$$
so that, rather than (4.1.3), in this case we get
$$C_D={48\over Re},\eqno(4.1.6)$$
a result first calculated by Levich (1949).
It is interesting to note that integrating the normal stress
for the potential flow over the bubble, gives a drag coefficient of
$32/Re$. The reason for this discrepancy, as was pointed out by Moore (1963),
is that the neglected pressure perturbation due to the 
boundary layer also contributes to the drag at the same order.

By taking into account energy dissipated in the boundary layer
and wake, Moore (1963)
found the first order correction to the drag coefficient (4.1.6)
for spherical bubbles. His value is
$$C_D={48\over Re}\left(1-{2.21\over Re^{1/2}}+\ldots\right).\eqno(4.1.7)$$
For larger bubbles, the Weber number will not be so small and so the 
bubble shape will deviate from that of a sphere.
For large Reynolds numbers, the ellipticity is given by
\hbox{$\chi=1+9W/64+O(W^2)$}.
It is possible to form the corresponding expression for the
drag coefficient for $W\ll 1$ (see Moore, 1965),
when the bubble is assumed to be ellipsoidal.
\vskip 15pt
\c{\it 4.1.2 Numerical methods.}
\vskip 5pt
Many fundamental
free-boundary problems, particularly those in bubble dynamics,
exhibit a symmetry about a central axis. It is therefore not surprising
that boundary element techniques similar to those described in earlier
chapters have been developed to exploit this feature
of axisymmetry. Indeed,
there are many difficulties associated with a fully three-dimensional 
method, not the least being the increased computational times
involved with the number of nodes required to maintain a reasonable
resolution on a two-dimensional surface mesh. Examples of
3D codes have however been successfully used by, for instance,
Chahine (1991) and Harris (1992).
This highlights the beauty of 
two-dimensional problems where symmetry has very little effect 
on the method that has to be used.
In effect, the axisymmetric method reduces a three-dimensional
problem to a two-dimensional one. This is done in the obvious
way by representing the bubble as a curve in a half-plane containing
the axis of symmetry. The azimuthal integration in the Green's
formula can then be carried out analytically and written in terms
of complete elliptic integrals.

There are many examples of uses of axisymmetric boundary element 
methods in the literature. The analogue of the techniques introduced
in Chapter 2 were used for cavitation bubbles by Guerri et al (1981),
Blake et al (1986, 1987) and Taib (1985). An alternative integral 
equation approach was used by Miksis, Vanden Broeck and Keller (1981)
to find steady-state solutions for a bubble in a constant velocity 
stream.
In this article they also calculate the energy
dissipation and hence the drag coefficient and terminal
velocity, which compare well with the values
of Moore (1965). In a later paper --- Miksis et al (1982) --- 
the same authors considered a bubble rising under gravity and
included the effects of normal viscous stresses in their
pressure balance (see \S 4.4). More sophisticated, finite difference
techniques have also been developed. Ryskin and Leal (1984) used an orthogonal
mapping technique to solve the steadily 
rising bubble problem. Unsteady viscous flows
with a free-surface can also be treated accurately using front tracking finite 
difference
codes (see Unverdi and Tryggvason, 1992)

\vskip 15pt
\hbox{\bf 4.2 The axisymmetric boundary integral method.}
\nobreak
\vskip 5pt
We now give a brief overview of the differences between the 
two-dimensional Green's formula method of Section 2.5 and the method
used here to solve axisymmetric problems. Green's formula is given
by (2.2.12) where in this case
$$G(\bd{x},\bd{x}')={1\over 4\pi |\bd{x}-\bd{x}'|},
\eqno (4.2.1)$$
is the free-space Green's function for the three-dimensional Laplace 
equation and $C$ is the bubble surface.
In the examples given here, modified versions of the
code of Best and Kucera (1992) were used. 
The routines use cubic splines to interpolate
$N+1$ nodes, $\bd{x}_i$, $i=0,\ldots,N$, which are  
given in cylindrical polar coordinates by $(r_i,0,z_i)$.
Note that nodes $0$ and $N$ lie on the
central axis at the bottom and top respectively. The splines take the form
$$\eqalignno{\bd{q}(s,\theta)\big|_{\theta=0}=&
(r(s),0,z(s))=\bd{q}_{0i}+\bd{q}_{1i}(s-s_i)+
\bd{q}_{2i}(s-s_i)^2+\bd{q}_{3i}(s-s_i)^3,&(4.2.2)\cr
\noalign{\hbox{and}}
\phi(s)=&\phi_{0i}+\phi_{1i}(s-s_i)+
\phi_{2i}(s-s_i)^2+\phi_{3i}(s-s_i)^3,&(4.2.3)}$$
for $s_i\le s\le s_{i+1}$, $i=0,\ldots,N-1$.
Here, $\phi$ represents the potential along the surface and
$s_i$ is the cumulative arc-length from node $0$ to node $i$.

In order to obtain an accurate arc-length,
Best and Kucera (1992) use an iterative scheme.
Define $\delta s_i\equiv~s_i~-~s_{i-1}$ and set
$$\eqalignno{\delta s_i^{(0)}=&|\bd{x}_i-\bd{x}_{i-1}|,&(4.2.4)\cr
\noalign{\hbox{and}}
\delta s_i^{(j+1)}=&\int_{s_{i-1}^{(j)}}^{s_i^{(j)}}
\left |{\partial\bd{q}\over \partial s}\right |ds,
\quad j=0,1,\ldots .&(4.2.5)}$$
After each approximation, the spline coefficients 
for $\hbox{\bf q}$ are calculated to ensure that it runs through the 
$N+1$ nodes with continuous first and second derivatives. The end conditions
used are that the top and bottom of the bubble are  horizontal  as is
required physically.
The method converges very quickly, at which point the 
corresponding spline for the potentials can be constructed, with
end conditions dictated by symmetry, that $\partial\phi/\partial 
s=0$ at $s=0$ and $s=s_N$.

The unknown potential derivatives, $\psi_i$, are parametrised linearly
by arc-length, namely
$$\psi(s)=\psi_{i-1}{(s_i-s)\over \delta s_i}+
\psi_i{(s-s_{i-1})\over \delta s_i},\eqno (4.2.6)$$
for $s_{i-1}\le s\le s_i, i=1,\ldots,N$.

In the axisymmetric case, it is necessary to deal with the 
azimuthal integrations separately.
We may write the three dimensional
form of Green's formula, analogous to (2.2.12), evaluated on the bubble as
$$\eqalign{{1\over 2}\phi(\bd{x}_i)=k+
&\int_0^{s_N}{\partial\phi\over\partial n}(\bd{q}(s,0))\left\{
\int_0^{2\pi}G(\bd{x}_i,\bd{q}(s,\theta))r(s)d\theta\right\}ds
\cr
-&\int_0^{s_N}\phi(\bd{q}(s,0))\left\{
\int_0^{2\pi}{\partial G\over\partial n_q}(\bd{x}_i,
\bd{q}(s,\theta))r(s)d\theta\right\}ds.}\eqno(4.2.7)$$
Here the functions $\bd{q}(s,\theta)$ are given in terms
of the splines by $(r(s),\theta,z(s))$,
in cylindrical polar co-ordinates. As the Green's function vanishes
at infinity, the constant, $k$, is the limiting value of the potential there.

The inner integrals of (4.2.7) over the angle $\theta$ can be 
expressed in closed form (see Taib (1985)) as follows
$$\eqalign{\int_0^{2\pi}{\partial G\over\partial n_q}
(\bd{x}_i,\bd{q}(s,\theta))r(s)d\theta
=&{1\over\pi}{r(s)\over \bigl((r(s)+r_i)^2+(z(s)-z_i)^2\bigr)^{3/2}}
\Bigg\{{2\over k^2(s)}{dz\over ds}r_iK(k(s))\cr
&+\left[{dz\over ds}(r(s)+r_i)-{dr\over ds}(z(s)-z_i)
-{2\over k^2(s)}{dz\over ds}r_i\right]
{E(k(s))\over 1-k^2(s)}\Bigg\},
}\eqno(4.2.8)$$
and
$$\int_0^{2\pi}G(\bd{x}_i,\bd{q}(s,\theta))r(s)d\theta
={1\over\pi}{K(k(s))r(s)\over\bigl((r(s)+r_i)^2+(z(s)-z_i)^2)\bigr)^{1/2}},
\eqno (4.2.9)$$
Here
$$k^2(s)={4r(s)r_i\over (r(s)+r_i)^2+(z(s)-z_i)^2},\eqno (4.2.10)$$
and $K(k)$ and $E(k)$ are complete elliptic integrals of the first
and second kind respectively. These are calculated from the approximate 
formulae
$$\eqalign{K(k)=P(1-k^2)-Q(1-k^2)\log(1-k^2),\cr
E(k)=R(1-k^2)-S(1-k^2)\log(1-k^2),}\eqno(4.2.11)$$
where $P$, $Q$, $R$ and $S$ are polynomials tabulated in Hastings (1955). 

In an analogous manner to that explained in Chapter 2, we 
can substitute (4.2.6) into (4.2.7) to
obtain a system of \hbox{$N+1$} algebraic equations,
$${1\over 2}\phi_i+A_i=k+\sum_{j=1}^N(B_{ij}\psi_{j-1}+C_{ij}\psi_j),
\quad i=0,\ldots,N,\eqno(4.2.12)$$
where,
if we denote the azimuthal integrals (4.2.8) and (4.2.9) 
by $\alpha_i(s)$ and $\beta_i(s)$ respectively,
$$\eqalignno{
A_i=&\int_0^{s_N}\phi(s)\alpha_i(s)ds,&(4.2.13)\cr
B_{ij}=&\int_{s_{j-1}}^{s_j}
\left({{s_j-s}\over\delta s_j}\right)\beta_i(s)ds,&(4.2.14)\cr
\noalign{\hbox{and}}
C_{ij}=&\int_{s_{j-1}}^{s_j}
\left({s-s_{j-1}\over\delta s_j}\right)\beta_i(s)ds.&(4.2.15)}$$
Numerical integration is done via Gauss quadrature, and the 
weak singularity is dealt with by using a logarithmic Gauss-type scheme.

Two problems are considered here. The first is a simple extension
of Chapter 2,
that is modelling a constant volume bubble rising through an
infinite fluid under gravity. The second problem is that of finding 
the steady state of a rising bubble --- the problem considered in detail
by Miksis et al (1982).
\vskip 15pt
\hbox{\bf 4.3 The rise of constant volume bubbles.}
\vskip 5pt
This section is the axisymmetric analogue of the Chapter 2
problem. The only difference being the inclusion of surface tension 
effects. It is a straightforward extension to allow more than one bubble
positioned on the central axis, so we formulate the problem for the more
general case of $M$ bubbles. In this way, the
interaction of several bubbles can be examined. 

The solution domain, $\Omega_-$, is defined to be the 
infinite region bounded internally by each of the bubbles $C_m$, 
$(m=1,\ldots,M)$.
Normals, as previously, are taken as pointing outwards
from $\Omega_-$, into the gaseous phase.
Initially, bubbles are 
spherical with non-dimensional radii $r_m$, lengths being scaled
with respect to the radius of bubble $1$,
which we denote by $a$.
Bubble $m$ is initially
positioned on the $z-$axis at a distance $\gamma_m(>0)$ 
below the centre of bubble $1$ which is situated at the origin.
At the start of the calculations, the fluid is assumed at rest
so we take $\phi\equiv 0$ at $t=0$ and the constant, $k$, of equations
(4.2.7) and (4.2.12) is chosen to be zero.
As before we scale times with respect to 
$(a/g)^{1/2}$
and pressures by the factor $\rho ga$.

For air bubbles larger than about $1mm$ in radius, rising at terminal
velocity in water, the 
corresponding Reynolds numbers are of the order of $700$
(see Levich, 1962; Clift, Grace and Weber, 1978).
We can therefore assume that the potential flow
model of the previous chapters will give realistic
results when applied to the problem of axisymmetric
bubble rise.

With the above scalings, the Bernoulli equation becomes
$${p_\infty}=p+{1\over2}|\bd{u}|^{2}+
{\partial\phi\over\partial t}+z,\eqno(4.3.1)$$
where $p_\infty$ is the fluid pressure at infinity
at $z=0$.

If viscous stresses in the thin boundary layers
are ignored, a pressure balance may be used 
across each interface.
The pressure just outside the surface $C_m$ $(m=1,\ldots,M)$ is given by 
$$p=p_m(t)-{4\kappa\over E_o},\eqno (4.3.2)$$
where $\kappa$ is twice the mean curvature
and again $E_o={4\rho ga^2/\sigma}$ is the E\"otvos number.
Here, $p_m(t)$ $(m=1,\ldots,M)$ is the pressure inside the $m$th bubble.

Combining (4.3.1) and (4.3.2), evaluated initially and at a general time, and 
employing the substantial derivative
gives an equation for the time evolution of the 
potential on $C_m$,
$${D\phi\over Dt}\Big|_{C_m}={1\over 2}|\bd{u}|^{2}-z-\gamma_m+
{4\over E_o}(\kappa-\kappa_m)+p_m(0)-p_m(t).\eqno (4.3.3)$$
In (4.3.3), $\kappa_m(=2/r_m)$ is the initial
curvature for bubble $m$. Note that, by definition,
$\kappa_1=2$ and $\gamma_1=0$.

In order to eliminate the unknown pressure terms in (4.3.3) we introduce, 
as in the previous chapter, a function $f$ defined on $\partial\Omega$
by
$$f\Big|_{C_m}=\phi\Big|_{C_m}+k_m,\eqno(4.3.4)$$
where
$$k_m(t)=\int_0^t(p_m(t')-p_m(0))dt',\eqno(4.3.5)$$
for $m=1,\ldots,M$, and again we work with $f$ rather than $\phi$.
With this definition, the dynamic condition (4.3.3) becomes
$${Df\over Dt}\Big|_{C_m}={1\over 2}|\bd{u}|^{2}-z-\gamma_m+
{4\over E_o}(\kappa-\kappa_m), \quad m=1,\ldots,M.\eqno (4.3.6)$$

Each surface $C_m$ is now represented parametrically 
by the function $\bd{q}_m(s,\theta,t)$.
The kinematic condition expresses the assumption that there is
no mass transfer across interfaces,
so that particles that lie on a surface
will remain there and move with its local velocity. Hence
for the time dependent surface parametrisation, we have
$${\partial \bd{q}_m\over\partial t}(s,\theta,t)=
\nabla\phi(\bd{q}_m(s,\theta,t),t),\quad m=0,\ldots,M.\eqno(4.3.7)$$
The surface $B_m$, the projection of $C_m$ into the half-plane $\theta=0$,
is discretised using $N_m+1$ nodes
interpolated by cubic splines as described in section 4.2.
The curvature of $B_m$,
required in the boundary condition (4.3.6), is
obtained directly from the splines.

To keep the bubble volumes constant, constraint equations
in the form of the line integrals
$$\int_{B_m}{\partial\phi\over\partial n}rds=0,\quad m=1,\ldots,M
\eqno (4.3.8)$$
are used.

The calculation of the integrals that make up the boundary integral method
is essentially as described in section 4.2 except that all of the surfaces
must now be integrated over for each observation point.
At each time-step, equations analogous to (4.2.12) and 
the discrete form of (4.3.8) are solved to yield the
normal velocities and the constants, $k_m$. The  
bubble is then stepped through time using the boundary conditions
(4.3.6) and (4.3.7). A trapezium rule is used for the time integrations.
\vskip 15pt
\hbox{\bf 4.4 Steady rise of a bubble through a viscous fluid.}
\nobreak
\vskip 5pt
Consider a bubble rising in a viscous fluid at high Reynolds number.
Below, we describe a method
similar to that of Miksis, Vanden-Broeck and Keller (1982), to find
the steady-state shape reached. They assumed that the bubble 
remains at a fixed height, otherwise its volume would increase and no 
steady state could be reached. Viscous forces were included in their
model in as much as viscous stresses were incorporated into 
the normal stress balance across the bubble surface. There is also 
a pressure drop due to the thin boundary layer that exists in 
the real high Reynolds number flow, but Miksis et al (1982) showed
this to be negligible compared with the normal stress component,
$\sigma_{rr}$, on the upper part of the bubble surface. On the lower
part of the bubble, this pressure drop is not negligible and 
this, together with the turbulent wake, it was argued, reduces confidence
in the shape of this part of the bubble as predicted by their calculations.

The formulation of the model used here
is slightly different from that 
used by Miksis et al (1982). The method described below was used because it
was reasonably straightforward to adapt the existing
boundary integral program of Best and Kucera (1992), 
although unfortunately it uses more equations. In common with their method
however it is equally applicable to drops falling as to bubbles rising.

Assume that the fluid around the bubble is incompressible with 
density, $\rho$, and kinematic viscosity, $\nu$. The bubble is moving
upward through the fluid at a steady speed, $U$. 
As explained in section 4.1.1, the change in velocity due to a viscous 
free-surface boundary layer is $o(U)$. We may therefore assume that 
the flow can be represented by a velocity potential, $\phi$, so
that $\nabla^2\phi=0$ outside the bubble which, as usual, we
denote by $\Omega_-$. Denote the bubble surface itself by $C$.
At infinity, the fluid is taken to be at rest and the potential
assumed to vanish there.
On the bubble there is no normal component of velocity
relative to $C$, so we must have that, on $C$,
$${\partial\phi\over\partial n}=Un_z,\eqno (4.4.1)$$
where $n_z$ is the $z$-component of the unit normal to the bubble.

A normal stress balance on $C$, taking into account the density of gas
(or liquid), $\rho_b$, inside the bubble gives
$$p_b-\rho_bgz=p-2\rho\nu{\partial u_n\over\partial n}+\sigma\kappa,\eqno 
(4.4.2)$$
where $p_b$ is the ambient pressure within the bubble, $\kappa$ is
the sum of the curvatures, $\sigma$ is the surface tension and $p$ is
the pressure on the outside of the interface. If, as in Miksis et al (1982),
we take the pressure jump across the boundary layer as being
insignificant, then we may equate $p$ to the pressure outside
the boundary layer and use Bernoulli's theorem for the potential
flow region. First, we introduce a frame of reference where the bubble
is at rest by defining the potential $\psi=\phi-Uz$, so that $\psi\sim-Uz$
at infinity. 
This transformation alleviates the need to calculate
$\partial\phi/\partial t$ in the moving frame.
Hence we may write down Bernoulli's equation as
$${p_\infty\over\rho}+{1\over 2}U^2={p\over\rho}+
{1\over 2}|\nabla\phi-U\bh{k}|^2+gz.\eqno (4.4.3)$$
If we now eliminate the pressure, $p$, from (4.4.2) and (4.4.3) we get
$${p_\infty-p_b+{1\over 2}\rho 
U^2\over\rho}=gz\left(1-{\rho_b\over\rho}\right)
+2\nu{\partial u_n\over\partial n}-{\sigma\kappa\over\rho}+
{1\over 2}|\bd{u}|^2-U{\partial\phi\over\partial z}+{1\over 2}U^2.
\eqno (4.4.4)$$
Following Miksis et al (1982), scale velocities with respect to $U$ and 
lengths with respect to an equivalent radius, $r_e$, defined from
the volume, $V=4\pi r_e^3/3$. This reduces (4.4.4) to
$$1+\gamma+{3\over 4}C_Dz+{8\over Re}{\partial^2\phi\over\partial n^2}
-{4\over W}\kappa+|\bd{u}|^2-2{\partial\phi\over\partial z}=0,
\eqno (4.4.5)$$
where the parameters are defined by
$$\eqalign{Re&={2Ur_e\over\nu},\cr
W&={2U^2r_e\rho\over \sigma},}\qquad
\eqalign{C_D&={8\over 3}\left(1-{\rho_b\over\rho}\right){gr_e\over U^2},\cr
\gamma&={p_b-p_\infty-{1\over 2}\rho U^2\over{1\over 2}\rho U^2}.}
\eqno (4.4.6)$$

To solve for the shape of the bubble and the potential on the bubble, the
intersection of the bubble surface, $C$, with the half-plane containing 
the axis of symmetry can be 
discretised and represented using an even number of nodes $N=2K$
interpolated using  cubic splines as defined above
by (4.2.2) to (4.2.5). The linear interpolant (4.2.6) gives
the normal derivative of the potential in terms of its nodal values.
To describe the properties of the fluid itself, we use the
Morton number, 
$$M=\left(1-{\rho_b\over\rho}\right){g\rho^3\nu^4\over \sigma^3},
\eqno (4.4.7)$$
and to describe the flow,
the Miksis et al `cavitation' number, $\gamma$.
The unknown parameters are the Reynolds number, $Re$, the
Weber number, $W$, and the drag coefficient, $C_D$. 
As the problem is axisymmetric, the end nodes are
explicitly placed on the central axis but at unknown heights.
We fix the height of the bubble by insisting that it is at its widest at
$z=0$, and explicitly set $z_K=0$.
Bearing this in mind, we see that there are a total of $3N+3$ unknowns.
The effect of a different choice for the bubble
height, for example setting $z_0$ to zero, would be to introduce a constant into
equation (4.4.5) and so alter the effective value of $\gamma$.
The physical requirements that
$dz/ds=0$ and $\partial\phi/\partial s=0$ on the axis are also explicit
in view of the spline end conditions given in section 4.2.

It is more convenient to write the term in (4.4.5) involving a
second normal derivative of the potential in terms of tangential 
derivatives which are readily calculated numerically from the
splines. As $\phi$ is harmonic,
$${\partial^2\phi\over\partial n^2}=-\left({\partial u_t\over\partial s}-
\kappa^{(t)}u_n+{\bd{u}\cdot\bh{r}\over r}\right),\eqno(4.4.8)$$
where $\kappa^{(t)}$ is the curvature of the surface in the plane $\theta=0$.

In order to determine the unknowns, Green's formula (2.2.12)
with $G$ given by (4.2.1) and $k=0$,
is employed to relate the potential and its normal derivative, the latter of 
which is given by the condition (4.4.1).
This is discretised as described in section 4.2
and, written in the form of (4.2.12), it represents $N+1$ equations.

The Bernoulli condition (4.4.5), likewise gives $N+1$ equations.
The volume of the bubble is also known: in non-dimensional
variables, it is
$$V={4\over 3}\pi.\eqno (4.4.9)$$
Since $r$ and $z$ are discretised separately, whereas Miksis et al (1982),
in principle, use $z_i=z(r_i)$ for a known set of values
$r_i$, we need to 
fix which one of the infinite number of solutions to take.
To do this, we set the spacing between all the nodes 
on the upper surface equal and, separately, all spacings on the lower surface 
are set equal, namely
$$\delta s_1=\delta s_2=\ldots=\delta s_K,\quad\hbox{and}\quad 
\delta s_{K+1}=\ldots=\delta s_N.\eqno (4.4.10)$$
This gives another $N-2$ equations
and allows us to choose node $K$ to coincide with the maximum bubble
radius expressed by the equation 
$$\bh{n}_K\cdot\bh{k}=0.\eqno(4.4.11)$$

Finally, we use the fact that the Reynolds number may be 
calculated explicitly in terms of the other parameters
$$Re=\left({3C_DW^3\over 4M}\right)^{1\over 4}.\eqno (4.4.12)$$

We now have sufficiently many independent equations to solve the
problem.
However, as was pointed out above, the 
drag calculated from the inviscid flow, and therefore which would be
produced by a solution of these equations,
does not take into account
the pressure perturbation of the boundary layer which contributes
to the drag at the lowest order. In order to compensate for this
(Miksis et al, 1982)
we can replace the Reynolds number in (4.4.12) by the value given 
by Levich's (1949) expression for the drag (see \S 4.1.1).
This gives the Morton number as
$$M={3\pi^4C_D^5W^3\over 1024I^4},\eqno(4.4.13)$$
where $I$ is the non-dimensional form of the energy dissipation
rate
$$I=\int_C{\partial\over\partial n}|\nabla\phi|^2dS.\eqno(4.4.14)$$
Using (4.4.13) in place of (4.4.12) allows a more accurate
results in the light of the above. After the solution has been found,
(4.4.12) can then be used to determine a better value for the Reynolds
number.

For computational purposes the integrand of (4.4.14) is written as
$$2\left(u_n{\partial^2\phi\over\partial n^2} + 
u_t{\partial u_n\over\partial s}+u_t^2\kappa^{(t)}\right).\eqno(4.4.15)$$

The system of non-linear equations (4.4.5), (4.4.8), (4.4.9), (4.4.10),
(4.4.11) and (4.4.12) can then be solved using
a modified version of Newton's method, where derivatives for the 
Jacobian are approximated by a simple finite-difference scheme.
As a starting point, the asymptotic solutions for large $\gamma$,
given in Miksis et al (1982), are used. In order to get 
convergence for other values of $\gamma$ the procedure is run
repeatedly with smaller and smaller values.
As the Morton number is typically much less than unity, (4.4.13)
is divided by $M$ in order to allow unbiased determination of the error
for this equation.
\vskip 15pt
\hbox{\bf 4.5 Results.}
\vskip 5pt
The following pages show examples of output for the two problems 
considered in sections 4.3 and 4.4.
Figures 4.1(a)-(d) show the time evolution for single bubbles
accelerating from rest at $t=0$. In 4.1(a) the largest bubble,
with an E\"otvos number of $212$,
is considered, so that buoyancy forces are much stronger 
than surface tension. Consequently, a narrow jet of fluid forms below the 
bubble, eventually impacting on the upper surface, at which
point the calculations break down.
In reality, the bubble would evolve into a toroidal vortex ring bubble
similar to those modelled by Lundgren and Mansour (1991).
As the E\"otvos number decreases, corresponding 
to smaller bubbles and higher values of surface tension, the jet
becomes slower and less pointed. In figure 4.1(b), where $E_o=29.8$,
the jet only just reaches the far side of the bubble before widening and 
forming a jet directed radially outwards,
ultimately pinching off a toroidal bubble and leaving behind a spherical cap.

As the E\"otvos number decreases further, the size of the toroidal bubble 
that splits off decreases as the jet becomes shorter and wider. For the case 
$E_o=13.2$, shown in figure 4.1(c),
the bubble eventually evolves into a shape resembling a skirted bubble.
Experimentally observed skirted bubbles (Bhaga and Weber,
1981) rising steadily in viscous fluids are only observed 
for intermediate Reynolds numbers (between about $10$ and $100$),
with large E\"otvos numbers (between about $300$ and $1000$).
As the jet broadens, the skirt thins and the curvature at its rim increases
until surface tension forces pull it back up to the position shown in the
final frame. This particular bubble seems to be close to the limiting
size whereby bubbles larger than this develop a jet which pinches off
a thin ring and bubbles smaller than this form a jet so wide that the
bubble forms a crescent shape when viewed in profile.

Figure 4.1(d), a slightly smaller bubble, $E_o=4.8$ is shown. Here
the effect of surface tension is again exhibited pulling back the rim,
this time just as a very slow jet starts to form. As the bubble accelerates
and thins due to the increasing pressure difference across 
its surface, it begins to develop a secondary jet or dimple on its underside. 

As the bubble becomes thinner and the acceleration decreases,
the pressure difference between the upper and 
lower surfaces due to the hydrostatic gradient and the time rate
of change of the velocity potential becomes less.
The curvature of the upper and lower surfaces thus become equal
and the lower dimple is matched by an upper one. Eventually these
meet, transforming the bubble into a toroidal geometry.

In summary, there seem to be two clear cut-off points which, although
we have made no attempt to locate them accurately, seem to occur around
$E_o=30$ and $E_o=13$. The first of these corresponds to the change from the
jet impacting at a single
point --- or at least over some simply connected region ---
of the upper surface, to the jet impacting on a ring --- or doubly
connected region --- thus breaking the bubble up into a single
toroidal bubble or a toroidal bubble together with a spherical cap
respectively.
The other bifurcation point is when the impact ring reaches
the rim of the bubble, so that the bubble remains intact, although 
the resulting high curvature at the rim subsequently pulls it back towards the 
centre of the bubble.
We should point out that continuing these calculations indefinitely
to ascertain the long-term behaviour of the bubbles that remain in one piece
can be seen as somewhat unrealistic in that we are ignoring
the drag on the bubble which would ensure that they would
reach a terminal rise speed, rather than keep on accelerating, albeit
slower as the added mass increases.

The similarity between figure 4.1 and the results of 
Chapter 2 are clear, showing the value of the study
of two-dimensional bubbles as an indication of the
behaviour in the axisymmetric case.
The major difference between this and the two-dimensional case
is that due to the greater mobility of the fluid moving in a 
three dimensional geometry, liquid is more easily drawn in to form a 
jet. Consequently for large bubbles jets may penetrate the
bubble and impinge on its upper surface. For the two-dimensional
case, this is not so: even for effectively infinite E\"otvos numbers
the jet broadens out due to the downward pull of gravity before 
being able to reach the far side. The broadening jet in the axisymmetric
case is a result of surface tension acting so as to prevent 
the sharpening of the jet, rather than of gravity alone.

In figures 4.2(a) and (b),
the interaction of two identical bubbles is examined. 
In each case, $\gamma_1=0$ and $\gamma_2=-2.5$.
In 4.2(a), $E_o=212$ and we find that the jet of the upper bubble
is noticeably broader than that of the lower bubble, and 
slightly broader than the case of 
a single bubble of the same size (see figure 4.1(a)). 
However the increased pressure at the
top of the lower bubble, as compared to case when this second  
bubble is not there, (see figure 4.5(a))
has the effect of making the tip speed of 
the jet on the upper bubble slightly faster than for a single bubble of 
equal size. This can be seen by comparing figures 4.1(a) and 4.2(a)
at $t=0.91$ and $t=1.2$.
Although the upper part of the lower bubble
is pulled upwards slightly with the jet of fluid into the upper bubble,  
this effect is not as great as in 
the two-dimensional case (Robinson, 1992; Robinson et 
al, 1993). The volume flow rate across a normal to a
sphere, radius $a$, placed in a uniform stream, $\phi\sim Uz$ at infinity
is $U\pi a\sin^2\theta$, per unit length of normal.
The corresponding rate for a cylinder is $Ul\sin\theta$,
where $l$ is the length of the cylinder. 
Since for small $\theta$ the expression for the spherical case
is an order of magnitude smaller, we may expect that 
the fluid in the jet of the axisymmetric bubble to originate mainly
from the sides rather than from the underneath. Comparing
figures 2.1 and 4.1(a) --- both for large bubbles with buoyancy 
dominant --- seems to bear this out:
the initial jet is much narrower for the case of the cylindrical bubble.
The result of this when a second bubble obstructs the flow 
beneath the upper bubble is that the jet speed in the top one
is largely unaffected as the bulk of the volume is drawn from
the sides, so that the lower bubble is not drawn into the jet
as much as it is in the two-dimensional case.
The lower jet differs little from the jet of a single bubble except 
that careful comparison with figure 4.1(a) shows it to be
slightly faster. This can be seen as a result
of the lower bubble becoming thinner due to the flow around the upper
bubble, and therefore rising faster ---
a similar `slipstreaming' behaviour was reported for the 
two-dimensional case (Robinson, 1992; Robinson et al, 1993).

For smaller bubbles, $E_o=13.2$, shown in 
figure 4.2(b), the interaction is less strong
due to the bubbles widening as very broad jets form, thus
increasing their separation.
However ring bubbles at the bottom of both bubbles are
now seen to split off
rather than be pulled back by surface tension as in the case of the 
single bubble. This is likely to be as a result of slightly faster jets
as in figure 4.2(a).
In general, the effect of the following bubble is the same as for the
larger bubbles in the previous figure, for instance in the final frames
of both figures the lower bubble is taller and wider than the upper one.

Figure 4.3(a) shows how the high pressure due to surface tension
on the concave sections of the bubble of figure 4.1(c)
acts against further lengthening of the jet. The low pressure around the 
ring tip (figure 4.3(b)) of the bubble intensifies as the jet broadens.
Here unlike the two bubble case of figure 4.5 (corresponding to figure
4.2(b)), the high pressure below 
the bubble does not extend far into the broad jet, so that the low pressure 
is able to pull back the rim before the jet pinches it off.
Figure 4.4 shows the pressure distribution for the two bubbles shown in
figure 4.2(a)

The bubble shapes for the steady-state problem of section 4.4
are shown in figure 4.6(a) and (b)
for the cases $M=1.75\times 10^{-7}$, and $M=10^{-5}$
respectively. In each case $\gamma$
decreases in successive frames.
These compare well with the results of Miksis et al (1982).
As one may expect, as $\gamma$
is reduced due to the pressure difference between the points on the
axis and at the edge of the bubble increasing, the Weber number, $W$, 
which relates this pressure difference to the pressure drop due to surface
tension increases and the bubble thus becomes more elliptical
(see \S 4.1.1). A smaller 
Morton number has the effect of decreasing the buoyancy and viscosity and 
increasing surface tension. One effect of decreasing buoyancy is to 
lessen the pressure difference on the 
upper and lower surfaces thus increasing
fore and aft symmetry of the bubble. The case for the larger Morton number 
shows that
the lower surface has become slightly concave for $\gamma=0$, while the upper 
surface is strongly curved in the opposite direction.

Figure 4.7(a) shows the relationship between the drag coefficient and the 
Reynolds number. This agrees with the calculations of Miksis et al 
(1982). For much larger Morton numbers (above about $4\times 10^{-3}$),
the curve relating the drag to the Reynolds number does not
have a minimum value as it does here (Harper, 1972; Bhaga and Weber, 1981).

Figure 4.7(b) shows the effect of the bubble ellipticity,
$\chi=2r_K/(z_N-z_0)$, on the Weber number.
The case for $M=0$, which is calculated using the method described 
in section 4.4 by neglecting the viscous and the buoyancy terms
from equation (4.4.5), is also included.
This reduces the number of unknowns by $2$.
In addition, equation (4.4.13) was dropped as was (4.4.11) which
is redundant due to the symmetry of the new problem. The results show
a maximum value for the Weber number, as identified by Moore (1965),
at about $3.59$. Miksis et al (1981) found this value to be $3.23$.

