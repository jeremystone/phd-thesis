\vbox{
\c{\bigrmb Chapter 6.}
\vskip 1cm
\c{\bigrm CONCLUDING REMARKS AND}
\c{\bigrm FUTURE DEVELOPMENTS}
\vskip 15pt
\hbox{\bf 6.1 Conclusions}
\vskip 5pt
}
In this thesis we develop and utilise powerful boundary integral
techniques in order to study unsteady gas bubble rise and burst
at high Reynolds number.
This was motivated by the problem of cell destruction
in bubble aerated bioreactors. In particular, we have focused on the 
bubble disengagement which experiments highlight as an important
damaging factor. A consideration of the effect of a free surface on 
the motion of bubbles is also of interest. This, together with bubble bursting,
is a theme running through a large portion of this work.
An examination of the motions of two-dimensional bubbles provides an
important introduction both into the types of fluid behaviours that 
one may expect in three-dimensions and into the numerical methods used to 
study them. As the axisymmetric boundary integral method is effectively
a two-dimensional method with a modified Green's function, the step up
from two to three-dimensional bubbles is straightforward.
Although, through choice, we use splines for interpolation along the
interfaces in the latter case rather than linear elements,
the axisymmetric method still affords a simple, easily managed
surface representation, when compared to a fully three-dimensional 
code. The effectiveness of these techniques for solving  
often complicated, non-linear, unsteady fluid motions is apparent from the
previous chapters. At the very heart of the boundary integral methods used
is the ability to follow fluid interfaces through time and thus
gain an understanding of the physical processes involved which would not
have been possible through the use of purely analytical techniques.
Useful functions of the flow field such as streamlines, pressures and 
energy dissipation rates can also be calculated with relative ease.

Several boundary integral methods are compared for
solving the problem of a rising two-dimensional bubble.
The methods used are in many ways very similar. Unlike Baker et al (1982) and
Baker and Moore (1989) who form equations which can be used to directly update 
singularity strengths for the surface distributions, in each case we retain
the same form for the dynamic boundary condition and use this to update surface 
potentials. This comparison therefore depends mainly upon the relative merits
of the underlying integral equations upon which the methods rely.
In all of the methods used in Chapter 2,
linear elements are used to interpolate surface nodes for integration
purposes. The use of quadratic elements affords increased accuracy
for the determination of tangential derivatives but is also sometimes necessary
in the calculation of integrals over elements adjoining
the field point node in order to maintain the regularity of the integrand.
Superficially, the results indicate very little difference between the various 
approaches except for the case of a source distribution method which fails
to calculate the motion beyond the first stages of jet formation. Otherwise the
bubble profiles agree with independent calculations performed by Baker and 
Moore (1989). 
The vortex method seems to conserve energy best, particularly in the final
stages 
of the calculations. The numerical stability of the Green's formula method,
based on a first-kind Fredholm equation, is comparatively
poor. When using an initial bubble perturbed by just $1\%$, with a wave number of
$18$ the top of the bubble develops a severe
Rayleigh-Taylor instability within a very 
short time. The underside of the bubble, where the liquid is accelerated
towards the lighter gas phase, is largely unaffected by the initial 
perturbation as might be expected.
This instability seems to be as a result of ill-conditioning of the Green's
formula method, which greatly amplifies a slightly altered initial condition.

An investigation of the physical stability of a two-dimensional bubble
rising from rest for small times, when an analytical solution for the 
unperturbed motion is available, shows that a small amount of surface tension
can stabilise a bubble, at least for short wavelength perturbations, 
even though this surface tension has very little
effect on the motion of an unperturbed bubble, where the 
surface curvatures are much less. This suggests that the Green's formula
method may reflect reality better than the other methods.

The reason for the lack of success with the source and modified source methods
is not clear --- the condition numbers of the modified source method
which relies on a second-kind equation are not significantly higher
than those for the Green's formula method. 
It seems that there is an inherent problem with trying to accurately solve a 
problem with a dipole far-field potential based solely 
on a distribution of sources.

By employing a conformal map to transform a semi-infinite region onto a 
bounded domain, the boundary integral methods described in Chapter 2 are
extended to allow the modelling of a two-dimensional bubble beneath a free
gas/liquid interface. The bubble behaviour is very similar to that
for the infinite fluid case, especially when the 
bubble rises from a distance far below the interface; a fact that gives us 
confidence in the accuracy of our results. 
Nearer to the interface, the fluid is less mobile due to the stabilising effect
of gravity on the free-surface.
Consequently, the jet formed on a bubble rising in
close proximity to the surface is appreciably slower and thus broadens earlier.
The inclusion of surface tension allows us to model bubbles that do not
form jets, but that become flattened as they rise due to the lower pressures 
around the sides. The pressure at a point in the fluid above the bubble
increases as the bubble approaches so that the free-surface
rises noticeably when the bubble becomes close. This surface elevation
compares well with calculations based on the assumption 
that a deeply submerged, circular, constant volume bubble may 
be regarded as a dipole.

Several similarities in the bubble shapes for the two and three-dimensional cases
become apparent when we consider the problem of axisymmetric rising bubbles.
For low surface tension, a much faster jet is formed which,
although initially broader,
impacts on the far side of the bubble rather than spreading out.
The broadening of the jet tip, which subsequently pinches off a ring bubble
leaving a spherical cap, occurs only in the case of non-zero 
surface tension. Increasing surface
tension results in wider jets, until no secondary, toroidal bubble 
is released. 

The interaction of two axisymmetric bubbles
is also considered. The effect on the 
lower bubble is greater, its top being drawn up with the fluid 
flowing into the jet of other. This phenomenon is not as 
marked as in the corresponding 
situation for two-dimensional bubbles
(considered by Robinson, 1992; Robinson et al, 1993),
since a proportionately greater volume
of the fluid forming the jet originates further from the axis of symmetry
in three-dimensions.
As the top of the lower bubble is pulled upward, the bubble as a whole thins
reducing its added mass so that it rises faster than it 
would if it were in an infinite fluid. Its jet speed is correspondingly 
increased.

As an example of an alternative use of boundary integral methods, the problem
of a bubble rising steadily due to the balance of viscous drag and buoyancy is
considered. This problem is formulated as a large set of non-linear integral
and differential equations which when put into a discrete form 
are solved using Newton iteration. The results match closely those
of Miksis, Vanden Broeck and Keller (1982) who first took this approach
to this problem.

In the last chapter, we develop a numerical model of bursting bubbles 
for the time 
between immediately after the rupturing of the film up until jet and 
drop formation.
The formation of jets is the result of the
collapse of the bubble crater towards the axis of symmetry so that fluid is
eventually forced both upwards and downwards.
High speed jets do not feature in larger (over about $2.5mm$ radius) bubble
bursts as there is insufficient potential energy in the initial configuration.
This is due to the reduced equilibrium depth of a larger, more buoyant
bubble.

High energy dissipation rates prior to jet rise
have been identified as a possible indicator of cell damage.
The maximum values of energy dissipation rates are reduced 
exponentially as the bubble radius is increased.
With regard to the problem of cell damage 
in aerated bioreactors, this is very encouraging, particularly as 
experiments suggest that only the very small bubbles cause damage. However,
the full implications can only become apparent when we know
more details of the likely positions of cells in relation to
regions of high rates of strain, both around the crater as it collapses and
below the bubble as the jets form.
The survivability of cells in specific flow environments also needs
to be studied in more depth before any concrete conclusions are 
drawn in this respect.

A technique for including viscous forces in a stress-free boundary
layer, based on the method of Lundgren and Mansour, is introduced. 
This utilises the fact that a linearisation of the velocities in
the boundary layer results in material elements that remain straight
and perpendicular to the interface. This has the
effect of obviating the need to use a complicated boundary layer
mesh to calculate numerical values for the first order perturbation 
to tangential velocities.
The only noticeable effect of the boundary layer on the
motion is on the jet which is slowed slightly due to viscous dissipation.
The onset of boundary layer separation gives initial conditions for an estimate
of the vorticity development in the upward and downward jets.
This suggests only a small contribution to the total stress placed on cells,
and indicates that the potential flow model gives essentially accurate
information of the fluid motion subsequent to film rupture.
\vskip 15pt
\hbox{\bf 6.2 Future directions.}
\vskip 5pt
Although the work described in Chapter 5 provides an important basis
for future mathematical research into cell damage in bubble aerated bioreactors,
there remain many unexplored avenues of investigation.  In this section, we
indicate some of the main topics which need to be analysed further,
in order to arrive at a more definite set of damage mechanisms.
These topics may be divided into three main areas: bubble/free-surface
interactions; film rupture and foam production; cell motion. 
We consider each in turn below.
\vskip 15pt
\c{\it 6.2.1 Bubble/free-surface interactions.}
\nobreak
\vskip 5pt
The available evidence strongly suggests that
damage is linked with bubble/cell
interactions at the air/medium interface. Small bubbles, less than 5mm diameter,
appear to be the most lethal, but these often burst in clusters 
(Oh, Nienow, Al-Rubeai and Emery, 1992), 
rather than singly as was assumed in Chapter 5. This will clearly
affect the bursting process. Neighbouring bubbles are likely to 
inhibit much of the inflow of fluid required to produce the jet, and so one may
expect clustering bubbles to cause less damage.

Experimentation so far has been largely confined to 
studies with actual, scaled-down bioreactors and little
attention has been paid to observing the stresses placed on a cell
in the vicinity of a bursting bubble. A greater amount of experimental
information regarding the behaviour of a cell in a straining flow such
as that which is seen to occur at the base of the bursting bubble immediately
prior to jet formation, would be of great value in this respect.
Placing a cell at the stagnation point of a four-roll mill would be a 
useful experiment that could be performed as a means of 
determining the maximum strain which a cell can withstand.

In bioreactors, the medium is rarely a pure liquid.
Surface active compounds will often be present and
any good model of the flow must take account of surface 
rigidity and surface tension gradients. So far, qualitative agreement
with experiment has been obtained, but the results of 
the last chapter indicate that
it is of tremendous importance to determine accurately
the position of the bubble in the fluid when it bursts.
This will clearly affect greatly the
amount of energy released at burst, a factor critical
in deciding whether the burst will be deliterious to nearby cells.
It is clear that surface chemistry will play a role in
this, as well as altering the ensuing bursting motion.

Mathematically, the addition of surface active compounds to the reactor
medium, poses us a number of problems. The amount of surfactant
will vary from place to place:
there may be a different concentration on the free surface than on the 
rising bubbles that eventually burst there.
Consequently there will be a 
variation in surface tension so that we need to take into account the
associated contribution to tangential particle motions. 
The advection and diffusion of surfactant also needs to be included
in the calculations.
If there is sufficient surfactant, the liquid/gas interfaces will be effectively
rigid to tangential motion. 
The potential flow solution will not be able to take into account a 
no-slip boundary condition as well as the other boundary conditions.
The resulting boundary layer correction for the velocity at the boundary
is likely to have a greater effect on the flow 
than the stress-free case studied in the previous chapter.
In any case, as the surface tension gradients exert a force tangentially,
the surface will be able to support a non-zero stress and so
the techniques used in Chapter 5 will not be applicable.
\vskip 15pt
\c{\it 6.2.2 Film rupture and foam production.}
\vskip 5pt
There are two important aspects to this topic. The first of these is
based on the proposal of Chalmers and Bavarian (1991) that
film rupture occurring at bubble burst is a possible
cause of cell damage. This rests on the claim of Culick (1960) that 
the region of film in front of the rapidly expanding toroidal rim
of a rupturing film is almost stationary, thus any cells adsorbed on the
film will be struck by the rim with considerable force. 

The other important aspect of film rupture is its relationship to foams.
Experimentally it has been shown that the cell damage rates are 
very sensitive to the amount and stability of any foam produced 
and on the thickness of the lamellae separating bubbles.
The production of commercially viable bioreactors
depends critically on determining the precise physico-chemical properties of the
bioreactor medium that will support a stable foam
which, it has been observed (Handa, Emery and Spier, 1987), significantly
reduces the cell death rates. The mechanism for the protection afforded by
the presence of Pluronic F-68, which itself allows a stable foam to be formed,
is also a matter of great debate (Handa, Emery and Spier, 1987; Kowalski, 1991).
An understanding of the effects of the various
physical and chemical parameters on the bubble drainage times
would provide a vital cornerstone for the future success of this research.
\vskip 15pt
\c{\it 6.2.3 Cell motion.}
\vskip 5pt
In any complete study of the role of bubbles in damaging cells, there
must be a consideration of where the cells are in relation to the bubbles.
The work of Blanchard and Syzdek (1972) shows that certain solid particles
in this case bacteria, tend to become adsorbed onto bubble walls as they rise 
through the liquid.
Cells adsorbed onto bubbles are clearly in a position where they are 
vulnerable to damage from the stresses produced by bubble bursting.

Using similar techniques to those of Stoos and Leal (1990), a 
boundary integral method can be employed to calculate the motion
of a solid particle attached to a gas/liquid interface. In this way it 
is possible to gain important
information about the cell distribution on the bubble at the time of burst.
If they tend to congregate near to the base of the bubble as it rises then they
are likely to move into the jet during the burst, thus undergoing a period 
of high stress.
The possibility that cells may be dragged up through
the fluid in the wake of a rising bubble, leaving them at risk
of possible damage by the downward jet (Chalmers, 1992) is also of possible
interest.
\vskip 40pt
To conclude, we remark that we have developed boundary integral 
techniques which have allowed accurate calculations of the 
unsteady motions of rising and bursting gas bubbles in 
two and three-dimensional
flow domains. 
By devising a scheme whereby stress-free
interface conditions can easily be incorporated into a boundary integral method
through a viscous boundary layer, we have been able to approximate the 
advection of separated vorticity around a bursting bubble.
The presence of a free-surface or other nearby bubbles 
often has a significant effect on the fluid flow near to rising bubbles
and this is clearly reflected in the bubble motion.
The ratio of surface tension forces to buoyancy forces is
of great importance, both in terms of its effect on the deformation rate 
of rising bubbles and through
the ascertainment of initial conditions for bubble burst.
The latter being a critical factor in the determination of the
energy released during the burst itself.

