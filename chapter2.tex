\vbox{
\c{\bigrmb Chapter 2.}
\vskip 1cm
\c{\bigrm A TWO-DIMENSIONAL BUBBLE}
\c{\bigrm IN AN INFINITE FLUID.}
\vskip 15pt
\hbox {\bf 2.1 Introduction.}
\vskip 5pt
}
Two-dimensional bubbles are less physically realistic
than their three-dimensional counterparts. Nevertheless their 
study provides a useful introduction
to the mathematical concepts involved in potential theory and
boundary integral equations.

Two-dimensional bubbles have been the subject of the mathematical
attention of a  number  of researchers. Walters and
Davidson (1962) found an analytic expression for the initial motion.
They also conducted experimental studies and, in their paper, they include
photographs of a bubble rising in the narrow gap between two sheets of glass.

A complex-variable technique was used by Baumel et al (1982) to
compute the bubble motion.
They developed a series solution which they summed using 
a variant of Pad{\'e} approximants.
This approach successfully predicted
the position of the top and bottom of the bubble through time.
However because of singularities in the complex
plane, their series does not have a very large radius of convergence,
thus they could only use it to give information for small times.

Baker and Moore (1989) successfully used numerical techniques to model the
motion of the bubble.
Their methods hinge on finding equations describing the evolution of
the strengths of vortex or dipole singularities distributed over the 
gas/liquid interface.
The dipole method is given in detail in Baker, Meiron and Orszag (1982), and
works by deriving a second-kind Fredholm equation for the rate
of change of the dipole strength, which may be solved iteratively.
Vortex methods have been used extensively in other problems,
for example in the modelling of large 
amplitude surface waves (Baker, Meiron and Orszag, 1982),
and in studying the surface water waves induced by a submerged body (Soh, 1987).
\vskip 15pt
\hbox {\bf 2.2 Mathematical preliminaries.}
\vskip 5pt
\c{\it 2.2.1 Formulation of the problem.}
\vskip 5pt
Incompressible fluid fills the infinite
region of the \hbox{$(x,y)$-plane}, $\Omega_-$.
The fluid is bounded internally by the closed curve, $C$, representing
the bubble's surface. Denote the interior of $C$ by $\Omega_+$.
It is assumed that the fluid is inviscid 
and that the flow is irrotational since it is initially at rest.
Incompressibility leads to
$\nabla\cdot\bd{u}=0$ and irrotationality to
$\nabla\times\bd{u}=\bd{0}$ so that the velocity
field, $\bd{u}$, may be expressed as the gradient
of a potential, $\varphi$, which satisfies
the Laplace equation, $\nabla^2\varphi=0$.
Initially, the fluid is at rest so that the potential, $\varphi$,
is constant throughout the domain of solution, $\Omega_-$.
Particles on the boundary of the bubble, 
for this two-dimensional problem given by the
Lagrangian description $\bd{q}(\eta,t)$,
$\eta\in[0,1)$ say, must remain there.
This gives us the kinematic condition
$${\partial\bd{q}\over\partial t}(\eta,t) = \nabla\varphi(
\bd{q}(\eta,t),t),\quad\forall \eta\in[0,1),t\ge0.\eqno(2.2.1)$$
The dynamic boundary condition takes the form of the Bernoulli theorem,
namely
$${{p_\infty}\over\rho}={p\over\rho}+{1\over2}|\bd{u}|^{2}+
{\partial\varphi\over\partial t}+\Phi,\eqno(2.2.2)$$
where $p_\infty$ is the ambient pressure
and $\Phi$ is the gravitational potential ($= gy$).
Due to the relatively low density of the gas inside the
bubble, we assume that the pressure inside the bubble
is a function of time only. 

Lengths are scaled with respect to the radius of 
the initially circular bubble, $a$,
times with respect to $(a/g)^{1/2}$ and pressures with respect to
$\rho ga$.
Evaluating (2.2.2) at the bubble surface and using the
substantial derivative,
$${D\varphi\over Dt}={\partial\varphi\over\partial t}+
|\bd{u}|^2,$$
yields an equation describing the evolution of the potential 
of fluid particles on the bubble. In terms of dimensionless variables,
$${D\varphi\over Dt}=
{1\over 2}|\bd{u}|^2-y+p_\infty-p_b(t),$$
where $p_b(t)$ is the bubble's internal pressure.

The total volume change for a bubble containing an ideal gas at
fixed temperature, is given by
$${V\over V_0}\simeq \left(1-\rho ga\Delta y/p_\infty\right)^{-1}.$$
If $p_\infty$ is taken to be 
atmospheric pressure of the order of $10^6 dyn/cm^2$,
and we assume that the total translation of the bubble
before it splits up (see Walters and Davidson, 1962),
$\Delta y$, is only a few bubble radii,
then $\rho ga\Delta y/p_\infty\ll 1$
and we may assume that the bubble volume remains constant.
This agrees with the experimental findings of Walters and Davidson (1962).
To conserve bubble volume, we insist that the total flux of fluid 
through the bubble's surface vanishes, i.e.
$$\int_C{\partial\phi\over\partial n} ds = 0.$$
As a far-field condition for the potential, we may assume in this case that 
$\varphi(\bd{x})\rightarrow 0$ as $|\bd{x}|
\rightarrow\infty$.
As will be seen presently, this condition provides us with a 
means of calculating the unknown value of $p_\infty-p_b(t)$.

To remove the $p_\infty-p_b(t)$ term from the dynamic condition, we 
define a new potential, $\phi$, by 
$$\phi=\varphi+\int_0^t(p_b(t')-p_\infty)dt',\eqno (2.2.3)$$
(see also Lundgren and Mansour, 1991), thus yielding
$${D\phi\over Dt}={1\over 2}|\bd{u}|^2-y,\eqno (2.2.4)$$
for $\phi$ on the bubble.
\vskip 5pt
\c{\it 2.2.2 Green's Theorem.}
\nobreak
\vskip 5pt
In a similar manner to the three-dimensional case,
Green's formula may be used in the plane, namely
$$\int_A\left (G\nabla^2\phi(\bd{x}')-
\phi(\bd{x}'){\nabla'}^2G\right)dA'=\int_{C\cup\Sigma_R}
\left(G{\partial\phi\over\partial n}(\bd{x}')
-\phi(\bd{x}'){\partial G\over\partial n'}\right)ds',\eqno(2.2.5)$$
where here $A$ is a finite region of $\Omega_-$, bounded internally by $C$
and externally by $\Sigma_R$. Here,
$$G(\bd{x},\bd{x}') =
{1\over 2\pi}\log{1\over |\bd{x}-\bd{x}'|}$$
is the free-space Green's function for the
two-dimensional Laplace equation, i.e.
$${\nabla'}^2 G(\bd{x},\bd{x}')=
-\delta(\bd{x}-\bd{x}').$$
To resolve any ambiguity with functions of two space variables,
$\bd{x}$ and $\bd{x}'$,
differentiations and integrations with respect to $\bd{x}'$
are indicated by a prime: so that for example
$\{\nabla'\}_i\equiv \partial/\partial x_i'$.

If $\nabla^2\phi=0, \forall \bd{x}\in A$, we get
$$c(\bd{x})\phi(\bd{x})=\int_{C\cup\Sigma_R}
\left(G{\partial\phi\over\partial n}(\bd{x}')-
\phi(\bd{x}'){\partial G\over\partial n'}\right)ds',\eqno(2.2.6)$$
where
$$c(\bd{x})=\cases{1,&$\bd{x}\in A\subset\Omega_-,$\cr
{1\over 2},&$\bd{x} \in C\cup\Sigma_R,$\cr 0,&$\bd{x} \in \Omega_+.$}$$
These values for $c(\bd{x})$ come about as a result of the fact that in
two dimensions, Green's formula can be derived as the real part of 
a Cauchy integral. The factor of a half for the case where the 
field point is on the contour, represents the Cauchy principal value.

The integral over $\Sigma_R$ in (2.2.6) can be significantly simplified.
Following Batchelor (1967), this is perhaps
best seen if $\Sigma_R$ is taken to be a circle of radius $R$ around the
point $\bd{y}\in A$ which coincides with the field point, $\bd{x}$.
In this case the integral can be written as
$${1 \over 2\pi}\left(\log{1\over R}\int_{\Sigma_R}
{\partial\phi\over\partial R}ds+
{1\over R}\int_{\Sigma_R}\phi ds \right).\eqno(2.2.7)$$
The second integral in (2.2.7) over $\Sigma_R$ is just
$$\int_0^{2\pi}\phi d\theta.$$
In the general case, when
$${1\over 2\pi}\int_S{\partial\phi\over\partial n} ds = -m_S,
\eqno(2.2.8)$$
over any curve, $S$, enclosing the bubble, we have
by the divergence theorem that
$m_S=m$ is independent of the particular contour, $S$, since
$\nabla^2\phi=0$ in $A$. Thus $m$ is the rate of expansion of
the bubble. So taking $S$ to be $\Sigma_R$ and writing (2.2.8) in terms
of an angular variable, $\theta$, as above, we have
$${R\over 2\pi}\int_0^{2\pi} {\partial \phi\over\partial R}d\theta=-m.
\eqno (2.2.9)$$
Integrating (2.2.9) with respect to $R$ gives
$${1\over 2\pi}\int_0^{2\pi}\phi d\theta = m\log{1\over R}+k.\eqno (2.2.10)$$
Moreover $k$ is independent of $\bd{y}$, the centre of $\Sigma_R$,
since differentiating (2.2.10) with respect to $y_i$ gives
$${\partial k\over\partial y_i}={1\over 2\pi}\int_0^{2\pi}
{\partial\phi\over\partial y_i}d\theta,\eqno (2.2.11)$$
where this expression is independent of $R$ and, in the limit
$R\rightarrow\infty$, the right-hand-side is seen to be zero,
since the fluid is at rest at infinity.

Thus combining (2.2.6), (2.2.7), (2.2.9) and (2.2.10), we get the final form 
of Green's theorem
which can be used as a basis for boundary integral methods
in the unbounded domain, $\Omega_-$, viz
$$c(\bd{x})\phi(\bd{x})=k+\int_C
\left(G{\partial\phi\over\partial n}(\bd{x}')-
\phi(\bd{x}'){\partial G\over\partial n'}\right)ds'.\eqno(2.2.12)$$

Information as to the form of $\phi$ given by (2.2.12) can
be found by expanding in a Taylor
series (Batchelor,t 1967; Lighthill, 1986) for points
$\bd{x}\in \Omega_-$ with $r=|\bd{x}|\gg|\bd{x}'|$.
On doing this, we find
$$\phi(\bd{x})=k-{m\over 2\pi}\log{1\over r}-
{\bd{x}\over 2\pi r^2}\cdot\left({d(\bd{c}V)\over dt}+
{\bd{I}\over\rho}\right)+O(r^{-2}).\eqno(2.2.13)$$

Here $m$, the strength of the source term, is the expansion rate 
of (2.2.9). The second term represents a dipole and is made up of two
components. The first is related to the centroid velocity, where
$\bd{c}$ is the centroid position and $V$ is the volume 
of the bubble. The second part is related to the Kelvin
impulse defined by
$$\bd{I}=\int_C\rho\phi\bh{n}ds.$$
It was suggested by Lamb (1932) that
the Kelvin impulse can be thought of as an impulsive wrench which
induces the motion on a fluid initially at rest, analogous to
the notion of impulse in the theory of particle mechanics.

It is clear from (2.2.13) that, if we take $\phi$ to be the modified 
potential defined in section 2.2.1, $k$ is the 
limiting value of $\phi$ as $r\rightarrow\infty$ and so,
from the far-field condition on the original potential,
$\varphi$, we have that 
$$k=\int_0^t(p_b(t')-p_\infty)dt'.\eqno(2.2.14)$$ 
\vskip 5pt
\c{\it 2.2.3 Indirect formulations.}
\nobreak
\vskip 5pt
From Green's theorem, the potential may be written as a simple
source or dipole distribution which can then give rise to
two so-called `indirect' boundary integral methods
(see also Brebbia and Walker, 1980 and Jaswon and Symm, 1977).
If, in (2.2.5), we had chosen $A$ to be the internal
region, $\Omega_+$, we would have immediately had
$$\tilde c(\bd{x})\tilde\phi(\bd{x})=-\int_C
\left(G{\partial\tilde\phi\over\partial n}(\bd{x}')-
\tilde\phi(\bd{x}'){\partial G\over\partial n'}\right)ds',\eqno(2.2.15)$$
(the `$-$' sign is because we are taking normal derivatives in the
same direction as in (2.2.6), i.e. into $\Omega_+$)
where
$$\tilde c(\bd{x})=\cases{1,&$\bd{x} \in \Omega_+,$\cr
{1\over 2},&$\bd{x} \in C,$\cr 0,&$\bd{x} \in \Omega_-.$}$$

Hence adding (2.2.15) to (2.2.12) gives, for all $\bd{x}$,
$$\phi(\bd{x})=k+\int_C
\left\{\left({\partial\phi\over\partial n}(\bd{x}')-
{\partial\tilde\phi\over\partial n}(\bd{x}')\right)G-
\left(\phi(\bd{x}')-\tilde\phi(\bd{x}')\right)
{\partial G\over\partial n'}\right\}ds'.$$
If we assert that the potential is to be continuous across $C$
then
$$\phi(\bd{x})=k+{1\over 2\pi}\int_C\sigma(\bd{x}')
\log{1\over|\bd{x}'-\bd{x}|}ds',\eqno(2.2.16)$$
where $\sigma(\bd{x}')=\left({\partial\phi/\partial n}-
{\partial\tilde\phi/\partial n}\right)$
and if the normal derivative is continuous across $C$,
$$\phi(\bd{x})=k-{1\over 2\pi}\int_C
\mu(\bd{x}'){\partial\over\partial n'}\log{1\over|\bd{x}'-
\bd{x}|}ds',\eqno(2.2.17)$$
where $\mu(\bd{x}')= (\phi-\tilde\phi)$.
\vskip 15pt
\vbox{
\hbox {\bf 2.3 Source distribution method.}
\vskip 5pt
\c{\it 2.3.1 Formulation.}
\vskip 5pt
}
\nobreak
The source formulation, given by (2.2.16), may be derived from the
Green's formula as shown above or may be thought of as a Poisson
integral for the solution of
\hbox{$\nabla^2\phi=-\int_C\sigma(\bd{q}(s))\delta(\bd{x}-\bd{q}(s))ds$},
where $\bd{q}(s)$ is an arc-length parametrisation of the bubble surface.

Consider the integral of the normal derivative of the
potential (by re-writing (A3)) for the cases
where the field point $\bd{x}^\star$ approaches $\bd{x}$ from 
$\Omega_+$ and from $\Omega_-$,
$$\int_C{\partial\phi\over\partial n}ds=
\int_C\left({1\over 2\pi}\int_C\sigma(\bd{x}'){\partial\over\partial
n}\log{1\over|\bd{x}'-\bd{x}|}ds'
\mp{1\over 2}\sigma(\bd{x})\right)ds,\quad \bd{x}^\star
\rightarrow \bd{x}_\pm.\eqno(2.3.1)$$
Here $\bd{x}_\pm$ is shorthand for the two limiting cases.
Note that, for $\bd{x}'\in C$, the Cauchy-Riemann relations
with inward facing normals, 
applied to the complex logarithm, give us
$$\int_C{\partial\over\partial n}\log{1\over|\bd{x}'-
\bd{x}|}ds
=\int_C{\partial\theta\over\partial s}ds=\pi,\eqno (2.3.2)$$
where ${\partial/\partial s}$ represents a tangential derivative and
$\theta$ is the angle between $\bd{x}$ and $\bd{x}'$.
So we have, on reversing the order of integration in (2.3.1) and applying
(2.3.2), that
$$\int_C{\partial\phi\over\partial n}ds=
\cases{\int_C\sigma(\bd{x}')ds',
&$\bd{x}^\star\rightarrow \bd{x}_-,$\cr
0,&$\bd{x}^\star\rightarrow \bd{x}_+.$}\eqno(2.3.3)$$
So in the case considered here, we need to explicitly constrain the
integral of source strengths over the surface to be zero, in order
that the volume remains constant.
To show that this can be done, consider the
problem of finding a potential in an internal domain, $D$,
with constant Dirichlet boundary conditions. If we are given that
$\phi=1$ on $\partial D$, then we have the following Fredholm
equation to solve for $\lambda$ on $\partial D$:
$${1\over 2\pi}\int_{\partial D}\lambda(\bd{x}')\log{1\over|\bd{x}'-
\bd{x}|}ds'=1,\quad \forall\bd{x}\in\partial D,\eqno(2.3.4)$$
(see for example Jaswon and Symm, 1963).
By the Dirichlet existence theorems, $\phi=1$ throughout $D$,
and so taking the derivative of (2.3.4) normal to
$\partial D$, into $D$ yields (as in (2.3.1))
$${1\over 2\pi}\int_{\partial D}\lambda(\bd{x}'){\partial\over\partial 
n}
\log{1\over|\bd{x}'-\bd{x}|}ds'-{1\over 2}\lambda(\bd{x})=0,
\quad\forall\bd{x}\in \partial D.$$
This has non-trivial solutions since application of (2.3.2) shows that its 
adjoint,
$${1\over 2\pi}\int_{\partial D}\mu(\bd{x}'){\partial\over\partial n'}
\log{1\over|\bd{x}-\bd{x}'|}ds'-{1\over 2}\mu(\bd{x})=0,
\quad\forall\bd{x}\in \partial D,$$
certainly has non-trivial solutions for $\mu$ --- namely $\mu=const$.
Multiplying  (2.2.16) by $\lambda(\bd{x})$, integrating over $C$, interchanging
the order of integration and using (2.3.4),
for the case $\partial D=C$, it is evident that the
volume remains fixed, according to (2.3.3), when 
$$k={\int_C\lambda(\bd{x})\phi(\bd{x})ds\over
{\int_C\lambda(\bd{x})ds}}.$$
In practice we do not need to find $\lambda$ in order to determine
the correct constant, $k$, as we can use (2.2.16) in conjunction
with a volume constraint. Thus, from (2.3.3), we need to solve the system: 
$$\eqalignno{\phi(\bd{x})&=k+{1\over 2\pi}
\int_C\sigma(\bd{x}')
\log{1\over|\bd{x}'-\bd{x}|}ds',&(2.3.5)\cr
0&=\int_C\sigma(\bd{x}')ds'.&(2.3.6)}$$
\vskip 5pt
\c{\it 2.3.2 Discretisation.}
\nobreak
\vskip 5pt
In order to solve the problem of the two-dimensional bubble 
computationally, the surface shape and functions defined on the surface 
need to be represented; in this case, discrete forms are used.
The symmetry of the problem means that only half of the bubble need be
followed through time, but a more general code
not requiring symmetry was written.
Only the repositioning routine assumes symmetry (see section 2.3.3),
but modification for the general case is straightforward.
The program should therefore be able to handle other problems such
as finding the motion of many bubbles 
in  arbitrary  arrangements (see Robinson, 1992;
Robinson, Boulton-Stone and Blake, 1993)
or, by using an appropriate Green's function, 
a bubble rising under an inclined plane.
In this particular case, the whole of the bubble's surface was represented at 
$N$
discrete points. It was found that in order to maintain stability
for those methods which produced good results, it was necessary to
place points either side of the central axis at the top and bottom
of the bubble, rather than on this axis. Since a linear representation
for the bubble shape was used here, this is the same as insisting that the
tangents to the top and bottom of the bubble are horizontal as
required physically due to the symmetry of the
problem.

An isoparametric representation was used, for both the bubble
surface and the source strength, $\sigma$, namely that
on the $i^{th}$ element $(i=1,2,\ldots,N)$,
$$\eqalign{
\bd{q}(\epsilon)&=(1-\epsilon)\bd{x}_i+
\epsilon\bd{x}_{i+1},\cr
\sigma(\epsilon)&=(1-\epsilon)\sigma_i+\epsilon\sigma_{i+1},
}\eqno (2.3.7)$$
for $\epsilon\in [0,1]$, where $\bd{x}_{N+1}\equiv\bd{x}_1$ and
$\sigma_{N+1}\equiv\sigma_1$.
Thus equation (2.3.5) becomes
$$\phi_i\equiv\phi(\bd{x}_i)={1\over 2\pi}\sum_{j=1}^{N}\int_0^1
\left(\bigl((1-\epsilon)\sigma_j+\epsilon\sigma_{j+1}\bigr)
\log{1\over |\bd{q}(\epsilon)-\bd{x}_i|}\right)d_j d\epsilon+k,
\eqno (2.3.8)$$
where $d_j=|\bd{x}_{j+1}-\bd{x}_j|$, is the length of the 
$j^{th}$ element
The coefficients of the $\sigma_i$ in the integrals (2.3.8)
can be placed in a matrix,
\hbox{$G\in R_{N+1\times N+1}$},
with ones in the $N+1^{st}$ column to represent the constant, $k$.
The $N+1^{st}$ equation is the volume constraint (2.3.6) which discretises to
$$0=\sum_{j=1}^{N}{d_j\over 2}(\sigma_j+\sigma_{j+1}),$$
so that $G_{N+1,j}=(d_j + d_{j-1})/2$.

A 4-point Gauss quadrature scheme is used to calculate the integrals in (2.3.8)
except in the case where $\bd{q}$ lies on an element adjacent to the
node $\bd{x}_i$, in which case it is a simple matter to
calculate them analytically.
Once $G$ is found, the equation $G\ub{\sigma}=\ub{\phi}$
where $\ub{\sigma}=(\sigma_1,\sigma_2,\ldots,\sigma_N,k)^T$
and $\ub{\phi}=(\phi_1,\phi_2,\ldots,\phi_N,0)^T$ can be solved
using Gauss elimination.

In order to update the positions of the nodes on the bubble, the derivatives
of the potential in the normal and tangential directions are required.
Differentiating (2.2.16), the expression for the potential, gives
(see A3)
$${\partial\phi\over\partial n}(\bd{x})={1\over 2\pi}\int_C
\bigl(\sigma(\bd{x}')\bh{n}(\bd{x})-
\sigma(\bd{x})\bh{n}(\bd{x}')\bigr)\cdot
{\bd{x}'-\bd{x} \over |\bd{x}'-\bd{x}|^2}ds',
\quad\bd{x}\in C,\eqno (2.3.9)$$
where $\bh{n}$ is the normal to the bubble surface at $\bd{x}$
or $\bd{x}'$ as indicated.
The knowledge of $\sigma$ on $C$ can be employed to find
$\partial\phi/\partial n$ at all points on $C$.
This integral is  calculated using the discretisation scheme of
(2.3.7), thus yielding an expression similar to (2.3.8).
We take the normal at $\bd{x}'$ to be in the direction
perpendicular to the line joining the ends of  each segment of integration.
However since $\bd{x}$, the observation point, is situated on a node
rather than along an element, it is necessary to use a quadratic fit
to the curve through the point and its two neighbours to get a good
approximation for the direction of the normal.
The exception to this rule is in the case where $\bd{x}'$ lies on the
intervals adjacent to
the node at $\bd{x}$. Here, the normal at $\bd{x}'$ is also taken
to be the normal to a quadratic through three points, and we use
quadratic interpolation for the integration (see Appendix B). If this
is not done, then as $\bd{x}'$ approaches $\bd{x}$ the normals
do not become parallel and thus the expression
$\sigma(\bd{x}')\bh{n}(\bd{x})-\sigma(\bd{x})\bh{n}(\bd{x}')$ 
of (2.3.9) does not vanish as it clearly should.
A 4-point Gauss rule is again used for all the integrals.
Placing the coefficients of the $\sigma_j$ in the integrals
corresponding to (2.3.9) into a matrix, $H\in R_{N\times N}$, and defining
$$\psi_i \equiv {\partial\phi\over\partial n}\Bigl |_{\bd{x}_i},$$
we have $\ub{\psi}=H\ub{\sigma}$,
which can then be used to calculate the normal velocities.

The tangential derivatives of $\phi$ and hence the tangential velocities
are calculated by fitting a quadratic to each set of potentials at
three adjacent nodes (Taib, 1985).
The boundary conditions (2.2.1) and (2.2.4) allow us to update
the nodal positions and potentials respectively,
integrating with a simple iterative trapezium rule (see below).
Once a single step of the method has been completed, the initial
conditions for the next step --- the potentials at and positions of the nodes 
--- are known. Thus repeating this process, the motion of the bubble can be
followed through time.
\vskip 15pt
\c{\it 2.3.3 Time-stepping and repositioning.}
\nobreak
\vskip 5pt
To move the position of the bubble and the potential at its
surface through time, an iterative trapezium rule is used for
each $\bd{x}_i(t)$ and $\phi_i(t)$, $i=1,2,\ldots,N$.
Firstly, a new value is predicted using Euler's rule; for instance, for
$\phi_i(t)$
$$\phi_i^{(0)}(t+\delta t)=\phi_i (t)+\delta t {D\phi_i \over Dt}(t),$$
where ${D\phi/Dt}$ is given by
the right-hand side of (2.2.4).
The trapezium rule,
$$\phi_i^{(j+1)}(t+\delta t)=\phi_i (t)+
{\delta t\over 2}\left\{{D\phi_i \over Dt}(t)+
{D\phi_i^{(j)} \over Dt}(t+\delta t)\right\},$$
is then used repeatedly.
The same scheme is used, in parallel, for $\bd{x}_i (t)$, with
the velocities given by (2.2.1).
(For simplicity, the velocities are resolved into cartesian
coordinates, when updating the position vectors).
Notice that the integral equations need to be re-formulated and solved
for the velocities at time $t+\delta t$
after the first Euler and each trapezium step, until convergence, thus
making the time-stepping process computationally expensive.

The time-step is chosen so that
$$\delta t=\min_{i=1,\ldots,N}\left\{\left|{\Delta\phi\over{D\phi_i/Dt}}
\right|\right\},$$
where $\Delta\phi$ has a prescribed value. This time-step is calculated
immediately before the first Euler step, and fixed until the time-step
has been completed.
Convergence is assumed when the position of the bubble does not change
between iterations, according to the criterion
$${\max\limits_{i=1,\ldots,N}\left\{\left|\bd{x}_i^{(j+1)}(t+\delta t)-
\bd{x}_i^{(j)}(t+\delta t)\right|\right\}
\over \max\limits_{i=1,\ldots,N}\left\{\left|\bd{x}_i(t)\right|\right\}}
<\epsilon.$$
Typically, $\epsilon=10^{-4}$.
If the method fails to
converge after $j_{max}$ steps, or starts to diverge after $j_{min}$
steps the time-step, $\delta t$, is halved and the procedure repeated.

If, after any of the time-steps, the nodes become too close, i.e.
if for any $j$
$$kd_j<{1\over N}\sum_{i=1}^N d_i,$$
then the original nodes are repositioned. Typically, $k=2$ gives
good results.
Repositioning is carried out by stepping around the
bubble and placing new nodes equally spaced according to the
linear arc-lengths of the old positions. The new positions and
potentials can be found by quadratic interpolation to three points.
Note that there is an arbitrariness here, since the new
point is generally on an old element, and thus a choice needs to be made
as to which of the two points at the ends of the adjacent elements to take 
for the third interpolation point.
Since the problem is symmetrical, we fix the side from which
the final node is taken and only reposition half of the surface
after which we reflect the data in the $y$-axis.
This reflection is also required on the grounds that it is not
the exact arc-length being used above to find the new positions of the
nodes, so it is likely that there would, otherwise, be a different
distribution of points on either side of the bubble, after repositioning.
The reason for using quadratic interpolation is
that with linear interpolation, the bubble tends to flatten
off and corners begin to appear after many repositionings.
\vskip 15pt
\hbox {\bf 2.4 Dipole distribution method.}
\nobreak
\vskip 5pt
Returning to Green's theorem and noticing that, for a fixed volume bubble,
the source term is identically zero in the far-field expansion (2.2.13),
it is reasonable to expect a double-layer distribution to
represent  well the behaviour of the bubble. 
If we know the potential initially then, in principle, we may use (2.2.17) to
find the
dipole strength, $\mu$. However as $\bd{x}\rightarrow C_-$, we get
the boundary problem
$$\phi(\bd{x})-k=-{1\over 2\pi}\int_{C}
\mu(\bd{x}'){\partial\over\partial n'}\log{1\over|\bd{x}'-
\bd{x}|}ds'+{1\over 2}\mu(\bd{x}),
\quad \bd{x}\in C,\eqno(2.4.1)$$
and from (2.3.2), the right-hand side vanishes if $\mu=const$.
One way to get round this problem is to add a vortex term to the integrand.
This was studied by Kre$\beta$ and Spassov (1983) for the Helmholtz and 
Laplace
equations, with a view to minimising the condition number of the
corresponding integral operators. However in order to keep this method as a
pure dipole method, the following resolution, detailed in Mikhlin (1957),
is used here.

Consider the complex form of (2.2.17), namely
$$W(z)=k-{1\over 2\pi i}\int_C {\mu (z')\over z'-z}dz',
\quad z\in\Omega_-,$$
and set $k$ to be the real constant,
$$k=-{1\over 2\pi}\int_C \mu(z')ds'.$$
As $z$ approaches $C$, the real part of the potential is given by (2.4.1).
If the homogeneous problem, i.e. $\phi\equiv\Re[W]=0$ on $C$, has solution
$\mu_0(z)$, then for $z\in\Omega_-$, define the function
$$W_0(z)=-{1\over 2\pi i}\int_C {\mu_0(z')\over z'-z}dz'-{1\over 2\pi}
\int_C \mu_0(z')ds'.\eqno (2.4.2)$$
Since $W_0$ is analytic with $\Re[W_0]=0$ on $C$,
it must be an imaginary constant, i.e. $W_0=ia$.
As $z\rightarrow\infty$ in (2.4.2), the first integral vanishes and so we
have
$$ia=-{1\over 2\pi}\int_C \mu_0(z')ds'.\eqno (2.4.3)$$
As $\mu_0$ is real, $a=0$ and so $W_0(z)=0,\forall z\in\Omega_-$, so that
(see Muskhelishvili, 1953), $\mu_0$ is the boundary value of a
function, $\Psi$, continuous on $C\cup\Omega_-$, analytic on $\Omega_-$.
Again as $\mu_0$ is real, $\Im\bigl[\Psi(z)\bigr]=0$ and as $\Psi$ is
analytic, it must be a real constant, $c$, i.e. $\mu_0=c$, but as
$a=0$, (by virtue of 2.4.3), $c=0$ and thus there are no non-trivial
solutions to the boundary problem.

Hence the dipole method used is based on the integral equation
$$\phi(\bd{x})={1\over 2\pi}\int_C\bigl(\mu(\bd{x}')-
\mu(\bd{x})\bigr)\bh{n}(\bd{x}')\cdot
{\bd{x}'-\bd{x}\over|\bd{x}'-\bd{x}|^2}ds'-
{1\over 2\pi}\int_C \mu(\bd{x}')ds',
\quad\bd{x}\in C,\eqno (2.4.4)$$
(see A5).
The method is, from now on, very similar to the source method, except
that no volume constraint is required: this is implicit.

Discretising the first integral of (2.4.4) according to (2.3.7) gives
$${1\over 2\pi}\sum_{j=1}^N\int_0^1\bigl((1-\epsilon)\mu_j
+\epsilon\mu_{j+1}-\mu_i\bigr)
\bh{n}\bigl(\bd{q}(\epsilon)\bigr)\cdot
{\bd{q}(\epsilon)-\bd{x}_i\over
|\bd{q}(\epsilon)-\bd{x}_i|^2}
d_j d\epsilon.\eqno (2.4.5)$$
If $G$ is the matrix of coefficients of the
$\mu$'s of (2.4.4), of which a contribution $\tilde G$ is from (2.4.5), with
$\tilde G^1$ and $\tilde G^2$ coming from the first and second terms
of the integrals of (2.4.5), then
$$\tilde G_{ij}=\cases{\tilde G^1_{ij}+\tilde G^2_{ij},&$j\ne i,$
\cr \tilde G^1_{ii}+\tilde G^2_{ii}+\tilde G^3_i,&$j=i,$}$$
where $\tilde G^3$ is the coefficient of the dipole
strength at the observation node  in  each  of  the  integrals  of 
(2.4.5).
It is clear that for the linear, isoparametric interpolation used,
$$\eqalign{\tilde G^3_i&=-\sum_{j=1}^N(\tilde G^1_{ij}+\tilde G^2_{ij+1})
\cr &=-\left(\sum_{j\ne i}\tilde G_{ij}+\tilde G^1_{ii}+\tilde G^2_{ii}
\right).}$$
This gives
$$\tilde G_{ii}=-\sum_{j\not = i}\tilde G_{ij},\eqno (2.4.6)$$
thus removing the need to explicitly calculate the integrals for the
diagonal terms of $G$.

Taking the normal at $\bd{q}$ as the linear normal, the
integrand of (2.4.5) vanishes when 
$\bd{q}$ is adjacent to $\bd{x}_i$.
The constant term in the second integral of (2.4.4) gives a contribution
to $G_{ij}$ of \hbox{$-(d_j+d_{j-1})/4\pi$}.

As before, we get a linear algebraic system, $\ub{\phi}=G\ub{\mu}$,
to solve, this time for the dipole strength at each node.
The normal derivative of (2.2.17) (see A6) is
$${\partial\phi\over\partial n}(\bd{x})=-{1\over 2\pi}\int_C
\left({d\mu\over ds}(\bd{x}')\bh{t}(\bd{x})-
{d\mu\over ds}(\bd{x})\bh{t}(\bd{x}')\right)\cdot
{\bd{x}'-\bd{x}\over|\bd{x}'-\bd{x}|^2}ds',
\quad\bd{x}\in C,\eqno(2.4.7)$$
where $\bh{t}$ is the tangent to the bubble.

The tangent at $\bd{x}'$ is taken to be the
linear tangent, whereas the
tangent at $\bd{x}$ is taken to be the quadratic-fit tangent (compare
with source method). Again, the exception is when $\bd{x}$ and
$\bd{x}'$ are adjacent, when the quadratic integration has to be used,
so that the integrand is non-singular as
$\bd{x}' \rightarrow \bd{x}$.
The tangential derivatives of the dipole strength are also found by a
quadratic fit to three points. Thus the coefficient matrix
for the discretisation of (2.4.7) allows us to calculate
the normal velocities on the bubble surface. The tangential
velocities can be calculated, as before, from the tangential derivatives
of the potentials.
The positions and potentials are then updated (see \S 2.3.3), thus
completing a single time step.
\pg
\hbox {\bf 2.5 Green's formula method.}
\vskip 5pt
A very common approach to solving potential problems by boundary integral
methods is to use Green's formula (2.2.12) directly.
The Green's formula method may be thought of as a distribution of both
single and double layer potentials on the surface of the bubble.
However the strengths of these distributions are the physically significant
potential and its normal derivative, rather than the jumps of these
values across the surface of the bubble, as is the case in
the indirect methods described in the last two sections.

Analogous to (A5) we may express the second integral of (2.2.12) in a regular
form so that Green's formula may be written as
$$\phi(\bd{x})={1\over 2\pi}\int_C{\partial\phi\over\partial n}(\bd{x}')
\log{1\over|\bd{x}'-\bd{x}|}ds'+{1\over 2\pi}
\int_C\bigl(\phi(\bd{x}')-\phi(\bd{x})\bigr)\bh{n}
(\bd{x}')\cdot{\bd{x}'-\bd{x}\over|\bd{x}'-\bd{x}|^2}
ds'+k,\eqno (2.5.1)$$
for $\bd{x}\in C$.
As, initially at least, we know the potential, $\phi$, at each of the points
on the bubble's surface, we can in principle solve (2.5.1) as a
Fredholm integral equation of the first kind for
${\partial\phi/\partial n}$ on $C$.

Since the method allows for a source-like behaviour, it is necessary to
apply the constraint
$$\int_C{\partial\phi\over\partial n}ds=0,\eqno (2.5.2)$$
to ensure a constant volume and determine the unknown constant, $k$.
Proceeding in a similar manner to previously, we discretise the
integrals using linear representations for the surface, the potential
and its normal derivative.

Define $G\in R_{N+1\times N+1}$ and $H\in R_{N+1\times N}$ to be
the coefficient matrices of the normal derivative and potential
respectively. Placing the second integral on the left-hand
side and using (2.4.6) we have
$$H_{ii}=1-\sum_{j\ne i}H_{ij}.$$
The $N+1^{st}$ row of $H$ contains zeros
and the volume constraint equation (2.5.2) gives
$G_{N+1,j}=(d_j+d_{j-1})/2$.
The $N+1^{st}$ column of $G$ contains all ones --- the coefficients of
the constant, $k$.

The linear system $G\ub{\psi}=H\ub{\phi}$,
can then be solved for $\ub{\psi}$, the vector of normal velocities.
Note that the integrals are exactly those which appear for the potentials in 
the source and dipole formulations and are calculated in the same way.
Once the normal derivatives and the tangential derivatives are found,
the potential and positions can be updated, thus completing a step
of the method.
\vskip 15pt
\hbox {\bf 2.6 Vorticity distribution method.}
\vskip 5pt
It is possible to solve the problem using a distribution of point vortices
around the bubble's surface. As the fluid is incompressible, we may
write $\bd{u}={\nabla}\times\bd{B}$ for a vector
field, $\bd{B}$.  If $\ub{\omega}$ is a vorticity distribution,
we have that
$$\ub{\omega}=\nabla\times\bd{u}=
\nabla(\nabla\cdot\bd{B})-\nabla^2\bd{B}.\eqno(2.6.1)$$
If (2.6.1) were simply
$\nabla^2\bd{B}=-\ub{\omega}$, it would
have the solution given by the Poisson integral,
$$\bd{B}(\bd{x})
={1\over 2\pi}\int_A \ub{\omega}(\bd{x}')
\log{1\over |\bd{x}'-\bd{x}|}dA'.\eqno(2.6.2)$$
It can be shown, by an application of the divergence theorem (Batchelor, 1967),
that $\nabla\cdot\bd{B}=0$ if the region, $A$, extends to infinity
where the fluid is at rest or if $A$ is extended to a region with a boundary
that is everywhere tangent to the vorticity, $\ub{\omega}$.

For a distribution of vorticity $\ub{\Gamma}(\bd{x}')=\ub{\omega}(\bd{x}')dn$
on the bubble, where $dn$ is an element normal to the surface, 
the vector potential (2.6.2) may be written
$$\bd{B}(\bd{x})
={1\over 2\pi}\int_C \ub{\Gamma}(\bd{x}')
\log{1\over |\bd{x}'-\bd{x}|}ds'.\eqno(2.6.3)$$
We may be able to invert this integral equation to find the vorticity
distribution,
given the vector potential. In order to find the velocity with
which the bubble's surface moves, we consider the complex potentials. As
$\bd{u}=\nabla\times\bd{B}$, for a
two-dimensional velocity field, $\bd{u}$, it must be that
$\bd{B}=(0,0,\psi)^T$, where $\psi$ is the stream function of the
flow, and $\ub{\Gamma}=(0,0,\Gamma)^T$. Thus from (2.6.3)
the complex potential, $W=\phi+i\psi$, is given by
$$W(z)=-{1\over 2\pi i}\int_C 
{\Gamma(z')\over\tau(z')}\log{1\over z'-z}dz',\eqno(2.6.4)$$
with $\tau(z)=dz/ds$ the tangent to $C$.
The derivative of (2.6.4) forms the basis of the vortex 
method used by Soh (1987).
Here, we use the normal and tangential velocity components
which, for the external problem, (see (A8-9)) are given by
$${\partial\phi\over\partial n}=-{1\over 2\pi}\int_C
\bigl(\Gamma(\bd{x}')\bh{t}(\bd{x})-
\Gamma(\bd{x})\bh{t}(\bd{x}')\bigr)\cdot
{\bd{x}'-\bd{x}\over|\bd{x}'-\bd{x}|^2}ds',\eqno(2.6.5)$$
and
$${\partial\phi\over\partial s}={1\over 2\pi}\int_C
\bigl(\Gamma(\bd{x}')\bh{n}(\bd{x})-
\Gamma(\bd{x})\bh{n}(\bd{x}')\bigr)\cdot
{\bd{x}'-\bd{x}\over|\bd{x}'-\bd{x}|^2}ds',\eqno(2.6.6)$$
for $\bd{x} \in C$.

The computational scheme here is slightly different from previously.
Once the integrals (2.6.5) and (2.6.6) 
have been calculated, in the same manner as the normal derivatives of the
source and dipole integrals, two matrix equations are obtained
$$\ub{\phi}_n=H\ub{\gamma},\eqno (2.6.7)$$
and
$$\ub{\phi}_s=G\ub{\gamma},\eqno (2.6.8)$$
respectively, with $\ub{\phi}_n$, $\ub{\phi}_s$ and $\ub{\gamma}$ denoting
vectors of $R^N$ containing nodal values of normal velocities, tangential 
velocities and vortex strengths respectively; $G,H\in R_{N\times N}$ are 
coefficient matrices.
Thus given a potential distribution on the bubble (initially constant),
its tangential derivatives are first calculated by a quadratic fit.
Equation (2.6.8) is then solved for the vortex strengths $\ub{\gamma}$
and this substituted into (2.6.7) to find the normal velocities. In
this way the bubble can be stepped through time exactly as before.
\vskip 15pt
\hbox {\bf 2.7 Results and discussion.}
\nobreak
\vskip 5pt
The following pages show examples of the output from the four codes described
above.
Calculations shown are based on a standard set-up with $N=50$ 
points,
the repositioning parameter $k=2$, the time-step parameter $\Delta\phi=0.1$
and the convergence parameter $\epsilon=10^{-4}$ (see \S 2.3.3). In order
to get insight into the effect of changing some of the parameters, other
runs were also carried out.
For the dipole, vortex and Green's methods the results are visually identical
and sample output for the vortex method only is included in figure 2.1. 
The number
of time-steps required to reach $t=4.0$ was $82$ for these three 
methods, in addition to which $8$, $7$ and $7$ time-steps 
respectively were abandoned in order to reposition points.
An estimate of the
average convergence rates in the time-stepping routine can be calculated
from the ratio of the number of iterations required in the trapezium rule
routine to the number of time-steps. The dipole, vortex and Green's
methods gave values of $2.86$, $2.82$ and $2.83$ iterations per
time-step respectively.
Thus, based
on this data so far there is nothing to distinguish these three methods.

The results in figure 2.1 closely match those of Baker and Moore (1989),
who went on to examine the discrepancies between calculation and experiment. 
The main difference, which can be clearly seen in the photographs
of Walters and Davidson (1962), is that the jet in the experiment widens out when
it is further away from the top of the bubble. 
This was shown to be as a result of a  greater computed jet 
speed. This discrepancy was not removed by including surface tension
forces into the calculations.
Possible reasons for differences were put forward. 
These relate to the fact that in the experiments, a two-dimensional bubble is 
approximated by injecting gas into the
narrow gap between two sheets of glass which is filled with liquid.
Thus, three-dimensional effects
such as the formation of a meniscus or a thin film 
separating the bubble from the glass may be important.

Figure 2.2 shows the poor performance of the source method. 
This was as far as
we could manage to get it to run. There seems to be an inherent instability
on the lower part of the surface, near the indent. We attempted to
smooth this out using the Longuet-Higgins and Cokelet (1976) 5-point 
formula, but this
made little difference.     
We also used a modified source distribution technique (Muskhelishvili, 1953),
by integrating (2.2.16) by parts to give, for $\bd{x}\in C$,
$$\phi(\bd{x})={1\over 2\pi}\int_C\bigl(\nu(\bd{x}')
-\nu(\bd{x})\bigr)\bh{t}(\bd{x}')\cdot
{\bd{x}'-\bd{x}\over |\bd{x}'-\bd{x}|^2}ds'+k,$$
where
$${d\nu\over ds}\Bigl|_{\bd{x}}\equiv\sigma(\bd{x}).$$
As in the dipole method, $k$ was chosen to be
$$k={1\over 2\pi}\int_C\nu(\bd{x}')ds',$$
to ensure solvability of the corresponding Dirichlet problem and thus
preserve bubble volume.
The normal velocities are again given by equation (2.3.9).
Results from this method were also very disappointing,
with only a small improvement over the original source method.

A routine to calculate the total energy in the fluid was included in all 
codes.
As in Baker and Moore (1989) the quantity $|T+V|/V$ was output at several
times throughout the runs. However, for all methods these values range
from about $10^{-3}$ up to $10^{-2}$ with no difference between
the dipole and Green's methods of 
magnitude significantly larger than the variation
with time. However for the vortex method the energy
ratios remain of the order  of  $10^{-3}$  throughout
most of the
latter part of the run, through to $t=4.0$, whereas the  other  two
methods that completed the run have figures higher than this by a 
factor of $10$.

To investigate the stability of the various codes, an initial perturbation
was applied to the bubble. Specifically, the initial radius was taken as
$$r=1+e\cos(f\theta).$$
For the case of 
a perturbation amplitude of $0.01$ and a wave number of $18$, a clear initial
amplification of the perturbation is apparent for the Green's method
(figure 2.3),
suggesting that it is significantly less stable than the other methods
which ran to  completion  by  sweeping  the  perturbation  to  the
underside of the bubble (figure 2.4).
For a lower wave number of $5$, both the dipole and vortex methods are
relatively untouched whereas the  Green's method reaches  $t=4.0$ 
visibly perturbed.
For a wave number of $9$,
a jagged region develops on the jet
in the vortex method causing the code to break down. It is 
however clear that the results for initially
large wave numbers are 
somewhat restricted by the point resolution on the bubble
so that the actual wave number 
imposed numerically may be much less than this and irregular
in apparent amplitude.

The 1-norm condition numbers for the various methods were also examined.
The average values for a complete run starting with the unperturbed 
bubble were
$1.2\times 10^2$, $6.1\times 10^1$ and $4.6\times 10^2$
for the dipole, vortex and Green's methods were respectively.
This may explain the
initial amplification of the perturbation for the Green's method as
opposed to the other methods, but work needs to be done on how stable
a two dimensional bubble really is
(see section 2.8), as in all of the methods
the repositioning will damp out perturbations to a certain extent
(Moore, 1981).
The greater success of the vortex and dipole methods may have been
anticipated on the grounds that they hinge on solving second-kind
rather  than  first-kind  equations.

It is interesting to note that the average condition numbers for the
source and modified source methods were $4.2\times 10^4$ and $5.8\times 10^2$
respectively, which suggests that it is not the stability of the
modified source method that hinders its progress, although this 
may contribute to the lack of success with the source method. 

The coloured plots of figures 2.5 and 2.6 show the pressure in 
the fluid at times $t=1.5$ and $t = 3.5$ respectively.
The pressures are calculated from (2.2.2) by working out,
at two successive time-steps, the potential, $\varphi$, at
a large number of points in the fluid using (2.2.12),
viz for $\bd{x}\in \Omega_-$,
$$\varphi(\bd{x})=\int_C
\left(G{\partial\phi\over\partial n}(\bd{x}')-
\phi(\bd{x}'){\partial G\over\partial n'}\right)ds'.$$
In  this  case,  a 
$30\times 30$ grid of  points is used.
Derivatives of the potential are calculated
by a finite difference scheme.
Figures 2.5 and 2.6 
also show the streamlines at the corresponding times. These
are calculated by taking the conjugate harmonic function to 
(2.2.12) which gives the stream function in the fluid in 
terms of the boundary values of the potential and its normal 
derivative, namely for $\bd{x}\in \Omega_-$,
$$\psi(\bd{x})={1\over 2\pi}\int_C\left(-\theta(\bd{x}'-
\bd{x}){\partial\phi\over\partial n}(\bd{x}')+
\phi(\bd{x}')\bh{t}(\bd{x}')\cdot
{\bd{x}'-\bd{x}\over|\bd{x}'-\bd{x}|^2}
\right)ds',$$
where $\theta(\bd{x}'-\bd{x})$ is the angle between 
the vector $\bd{x}'-\bd{x}$ and the $x$-axis,in the 
sense defined in Jaswon and Symm (1977, p 155), so that 
the function is continuous as $\bd{x}'$ traverses the bubble
surface for any fixed $\bd{x}\in\Omega_-$.
The pressure plots (figures 2.5 (a) and (c))
show the pressure near the centre of the underside
of the bubble to be slightly greater  than  the pressure beneath 
the outer parts; this helps to accelerate the jet through the 
bubble. 
At the later time, this high pressure is seen to extend into the 
middle of the jet causing the subsequent broadening.
The streamlines (figures 2.5 (b) and (d)) 
show the bubble to exhibit the essential behaviour
of a dipole, as predicted by equation (2.2.13).

If no repositioning is performed, all of the successful
methods  break  down  just 
before the jet starts to broaden out.
Due to the translation of the bubble,
the points move rapidly around to the lower part of the bubble
as it moves up through the fluid,
thus resolution is soon lost on the upper part. This can be clearly seen
in figure 2.6(b) where the paths taken by particles on the bubble's surface
are followed through time. Even points that start near the top of the bubble
end up well into the jet by $t=4.0$. 
Repositioning 
was not used in some of the calculations that have been done for
cavitation bubbles, for example Blake et al (1986),
where translating velocities are  much smaller than jet 
velocities. 

The programs were also run with reduced time-step convergence
parameters, so that the stepping routine converged with 
just one pass through the trapezium rule. All three working
methods ran through to t=4.0, thus indicating that the 
higher accuracy in the time-stepping routine may be superfluous.
\vskip 15pt
\hbox {\bf 2.8 Stability analysis.}
\nobreak
\vskip 5pt
\c{\it 2.8.1 Formulation of the equations.}
\nobreak
\vskip 5pt
In this section, we examine the  growth of 
small perturbations to a two-dimensional bubble.
This is done by utilising the analytic solution
of Walters and Davidson (1962) which is valid for small times.
Initially, the bubble has acceleration $g$ and thus the 
analysis is slightly different from that of Batchelor (1987)
who examined the stability of a spherical cap bubble which
had reached a steady state. It may be possible to adapt
Batchelor's results for a two-dimensional bubble that
has almost reached a terminal velocity and, on the upper surface
at least, is changing shape very slowly. This could
be used in conjunction with this analysis to judge its long-term
stability.

For convenience, place the co-ordinate origin at the centre of 
the accelerating bubble. If this centre rises at a speed $U(t)$
then, as in Walters and Davidson (1962), Bernoulli's theorem 
becomes 
$${p_\infty\over\rho}={p\over\rho}+{1\over 2}|\bd{u}|^2
+{\partial\phi\over\partial 
t}+g\left(\int_0^tU(t')dt'+r\cos\theta\right),\eqno (2.8.1)$$
where $\bd{u}$ and $\phi$ are the velocity and 
corresponding potential, taken with respect to the fluid
which is at rest at infinity, and $(r,\theta)$ represent 
polar co-ordinates from the moving centre with
$\theta=0$ in the direction of travel.
If the bubble has internal pressure $p_b(t)$, we have the 
pressure balance at the bubble surface
$$p=p_b-\sigma\kappa,\eqno (2.8.2)$$
where $\sigma$ is the surface tension, and $\kappa$ is the curvature.
Thus using (2.8.1) and (2.8.2) gives  
$${1\over 2}|\bd{u}|^2+{\partial\phi\over\partial t}
\Big|_{\hbox{\smrm fixed in}\atop\hbox{\smrm space}}+
g\left(\int_0^tU(t')dt'+R\cos\theta\right)
-\sigma\kappa=P(t),\eqno (2.8.3)$$
on the bubble surface, given by $r=R(\theta,t)$, where
$P(t)={(p_\infty-p_b(t))/\rho}$.

For the corresponding kinematic condition, as before,
use the fact that particles of fluid on the bubble surface
must remain there. In order to determine this condition, first define
the potential for the bubble
at rest with the fluid moving around it by $f=\phi-Ur\cos\theta$.
Thus the kinematic condition is 
$${\partial R\over\partial t}+{1\over R^2}
{\partial f\over\partial\theta}\Big|_{r=R}
{\partial R\over\partial\theta}=
{\partial f\over\partial r}\Big|_{r=R}.\eqno(2.8.4)$$
In terms of $\phi$, (2.8.4) becomes
$${\partial R\over\partial t}+{1\over R}
\left({1\over R}{\partial\phi\over\partial\theta}\Big |_{r=R}
+U\sin\theta\right)
{\partial R\over\partial\theta}=
{\partial\phi\over\partial r}\Big|_{r=R}-U\cos\theta.
\eqno(2.8.5)$$

Now introduce perturbations $\eta(\theta,t)$ to the bubble
surface and $\psi$ to the potential.
Substitute these into (2.8.3) and (2.8.5), expand about $r=R$,
neglecting squares of small terms, and subtract the corresponding
unperturbed equations. This gives, for the dynamic condition,
$$\eta u_\theta{\partial u_\theta\over\partial r}+
{u_\theta\over R}{\partial\psi\over\partial\theta}+
\eta u_r{\partial u_r\over\partial\theta}+
u_r{\partial\psi\over\partial r}+
\eta{\partial^2\phi\over\partial r\partial t}+
{\partial\psi\over\partial t}+g\eta\cos\theta-\sigma\kappa'=0,
\eqno (2.8.6)$$
and, for the kinematic condition,
$${\partial\eta\over\partial t}+{1\over R^2}\left\{
{\partial R\over\partial\theta}\left(
\eta{\partial^2\phi\over\partial r\partial\theta}+
{\partial\psi\over\partial\theta}-
U\eta\sin\theta-
2{\eta\over R}{\partial\phi\over\partial\theta}\right)+
{\partial\eta\over\partial\theta}\left(
{\partial\phi\over\partial\theta}+
UR\sin\theta\right)\right\}=
\eta{\partial^2\phi\over\partial r^2}+
{\partial\psi\over\partial r}.\eqno(2.8.7)$$
Here $u_\theta=\partial\phi/R\partial\theta$,
$u_r=\partial\phi/\partial r$ and $\kappa'$ is the 
small change in curvature.

In order to solve these equations and determine the 
evolution of the perturbation, it is necessary to
know the unperturbed quantities appearing in
(2.8.6) and (2.8.7). For this, we refer to the 
work of Walters and Davidson (1962). They derived
an approximate solution for the motion of the 
bubble, correct for small times. Here we use the
highest order terms in their expansion for the 
undisturbed potential. They assumed a general
potential of the form
$$\phi=\sum_{n=1}^\infty\beta_n(t){\cos n\theta\over r^n},
\eqno(2.8.8)$$
which they substituted into the Bernoulli equation (2.8.1),
ignoring the effects of surface tension and assumed the pressure
to be uniform throughout the bubble, since the density of the
gas contents is small compared to that of the surrounding fluid.
This gave them a first approximation to the unknown $\beta$'s,
ignoring non-linear terms, of
$$\eqalign{\beta_1^{(1)}=&-U^{(1)}a^2=-gta^2,\cr
\beta_n^{(1)}=&{(-1)^n(n-1)!n!2^na^2\over (2n)!t}(gt^2)^n.}\eqno 
(2.8.9)$$
The first two terms of the potential give
$$\phi\sim -gt{a^2\over r}\cos\theta+
{1\over 3}g^2t^3{a^2\over r^2}\cos2\theta,\quad t\rightarrow 0.
\eqno (2.8.10)$$
The unperturbed surface elevation can be obtained from
the kinematic condition (2.8.5). If the 
surface elevation is $R=a(1+\zeta)$, $\zeta\ll 1$, then
since $R\sim a$ as $t\rightarrow 0$, the first term of the 
power series in $t$ for $R$ must be $a$, and thus independent
of $\theta$. Hence by considering the second term in the
series, we see that 
$\partial R/\partial\theta=O(t\partial R/\partial t)$
as $t\rightarrow 0$,
for almost all $\theta$: the coefficient of the second
term may vanish for some values of $\theta$, with its
derivative non-zero. Thus the terms of (2.8.5) in 
$\partial R/\partial\theta$ are of order
$t^2\partial R/\partial t$ as $t\rightarrow 0$,
and so may be neglected for small times. This leaves us with
a first order ordinary differential equation for $\zeta$,
$$a{\partial\zeta\over\partial t}=
{\partial\phi\over\partial r}-U\cos\theta.
\eqno (2.8.11)$$
On substituting (2.8.10) into (2.8.11), using $U=gt$ and linearising
with respect to $\zeta$, we find the
solution, retaining just the first term in
$t$, for $\zeta$ is
$$R\sim a\left(1-{g^2t^4\over 6a^2}\cos 2\theta\right),
\quad t\rightarrow 0.\eqno (2.8.12)$$
Now with (2.8.10) and (2.8.12) and their respective derivatives
substituted into (2.8.6), we find
$$2\eta g\cos\theta+{\partial\psi\over\partial\theta}{gt\over a}
\sin\theta+{\partial\psi\over\partial r}gt\cos\theta+
{\sigma\over\rho a^2}{\partial^2\eta\over\partial\theta^2}+
{\partial\psi\over\partial t}=0,\eqno (2.8.13)$$
where a term proportional to $\eta g^2t^2/a$ has been
neglected in comparison to the $\eta g\cos\theta$
term. This is true for $t\ll (a/g\cos\theta)^{1/2}$, or if
$\theta\ll 1$ for $t\ll (a/g)^{1/2}$.
Similarly, we get for (2.8.7), 
$${\partial\eta\over\partial t}+2\eta{gt\over a}\cos\theta+
{1\over 3}{\partial\psi\over\partial\theta}{g^2t^4\over a^3}
\sin 2\theta - {\partial\psi\over \partial r}+
2{\partial\eta\over\partial\theta}{gt\over a}\sin\theta=0.
\eqno (2.8.14)$$
Again terms of order $t^5$ have been neglected in comparison
to terms of order $t$, when $t\ll (a/g)^{1/2}$.

Next, follow Batchelor (1987) by confining ourselves to
$\theta\ll 1$. Near the top, the interface
is almost flat, so introduce non-dimensional, cartesian
co-ordinates defined by $x=\theta, z=r/a-1, \tau=(g/a)^{1/2}t$.
Thus at $z=0$, (2.8.13) and (2.8.14) become
$$2\eta+{\partial\psi\over\partial x}\tau x+
{\partial\psi\over\partial z}\tau+
k{\partial^2\eta\over\partial x^2}+
{\partial\psi\over\partial\tau}=0,\eqno (2.8.15)$$
and
$${\partial\eta\over\partial\tau}+2\eta\tau+
{2\over 3}{\partial\psi\over\partial x}\tau^4x-
{\partial\psi\over\partial z}+
2{\partial\eta\over\partial x}\tau x=0,\eqno (2.8.16)$$
where $k=\sigma/\rho ga^2$.

In order to solve these equations, try a solution of the 
form
$$\eqalign{\psi=&A(\tau)\exp[n(\tau)(ix-z)],\cr
\eta=&B(\tau)\exp[in(\tau)x].}\eqno(2.8.17)$$
These are substituted into (2.8.15) and (2.8.16),
bearing in mind that 
$${\partial\phi\over\partial t}
\Big|_{\hbox{\smrm fixed in}\atop\hbox{\smrm space}}
={\partial\phi\over\partial t}\Big|_{\hbox{\smrm fixed}\atop r,\,\theta}+
{\partial\phi\over\partial r}{dr\over dt}+
{\partial\phi\over\partial\theta}{d\theta\over dt},$$
where the first term on the right is taken in the accelerating
frame of reference.

Neglecting the $\tau^4$ term as $\tau\ll 1$, 
the imaginary parts of (2.8.15) and (2.8.16) give $dn/d\tau+2n\tau=0$,
thus 
$$n=n_0\exp(-\tau^2),\eqno (2.8.18)$$
whereas the real parts give
$$\eqalign{{dA\over d\tau}=&B(kn^2-2),\cr
{dB\over d\tau}=&-2B\tau-An.}\eqno (2.8.19)$$
The form of equations (2.8.18) and (2.8.19) are in many ways similar 
to those found by Batchelor (1987). A couple of differences are worth
mentioning. In both this problem and that considered by Batchelor
as one moves around the bubble from 
a position above the upper stagnation point to the  equatorial region,
the flow speed relative to a frame where the bubble is stationary
increases and so the corresponding streamlines  become closer together.
Consequently any waves propagated along the bubble surface will
become elongated and attenuated. The effect of this
straining flow around the bubble is seen here for 
a single Fourier component of a travelling wave. In the case of an 
accelerating bubble the decay of the wave number is also dependent
upon time. For small times,  when the bubble is not rising
very fast, (2.8.18) shows that the
increase in wavelength is not as rapid as for the
steadily rising bubble. The other notable feature of these equations
is that, from (2.8.19),
the critical nondimensional wavelength, above which disturbances would be
expected to grow, is $2\pi/(\sqrt{2/k})=2\pi/\sqrt{2g\rho a^2/\sigma}$. 
This is due to the 
destabilising effect of the upward acceleration in addition to that of gravity;
hence $2g$ rather than $g$ as is the case of steady rise.
The meaning of the terms in equations (2.8.19)
becomes slightly more transparent if we eliminate $A$ to give
$${d^2B\over d\tau^2}+4\tau{dB\over d\tau}+B\left\{2+4\tau^2-
2n\left(1-{n^2\over n_c^2}\right)\right\}=0,\eqno(2.8.20)$$
where, as in Batchelor (1987), $n_c$ is the critical wave number, in this case
in non-dimensional form and equal to $\sqrt{2/k}$.
\vskip 15pt
\c{\it 2.8.2 Solutions and discussion.}
\vskip 5pt
Equations (2.8.19) are  solved  numerically, using  NAG routine
D02BBF, and  the 
results are given in figures 2.7(a)-(c).
For initial conditions it is assumed that the bubble  shape  is
perfectly circular, but that the initial velocity on the bubble
is perturbed slightly from zero. Thus we take $B(0)=0$ and
$dB/d\tau(0)=\delta$, and hence $A(0)=-\delta/n_0$.
As the equations are linear, $\delta$ is taken to be 
$0.01$, in all cases. These initial conditions are similar to those 
used by Batchelor (1987) for a spherical cap bubble, rising 
steadily. 

Figure 2.7(a) shows the effect of surface tension on the amplitude
of the perturbation, $B$, on the bubble, as a function of time,
for $n_0=6$ and $k=0$, $0.076$, $0.15$, $0.23$ and $0.30$.
This value of $n_0$ is the critical wavenumber if $k=0.056$, so that only 
the smallest two values for $k$ exhibit continued growth.
The other curves, which are beginning to show oscillatory behaviour
are higher values of the surface tension parameter, $k$,
thus confirming the stabilising effect of surface tension.
Note that as the physical validity of the results is confined to
$\tau\ll 1$ the attenuation due to the vertical contraction of the
fluid moving around the bubble is not apparent in these calculations,
although (2.8.20) shows that the amplitude would eventually approach zero
were they continued beyond $\tau=1$.
Figure 2.7(b) shows the effect of  changing  the 
initial wave number, for a value of $k=74/(980\times 2.54^2)$ corresponding to
a one inch bubble in water (this value was chosen to allow
a comparison with the Walters and Davidson (1962) and Baker and Moore (1989)).
Here we take $n_0=5$, $15$, $25$ and $35$.
A higher initial  wavenumber  results  in
higher frequency waves, which, as the initial velocity
is fixed at $0.01$, therefore have lower peak amplitudes.
Again growth is only observed when $n_0$ is (approximately)
less than $n_c$.
The lowest wave number plotted does not reach a
high enough amplitude for small times for us to say that
such a case is unstable.
This leads us to comment that
the analysis as given is limited by the 
fact that it can only predict instability if it occurs very
early on (i.e. for $\tau\ll 1$), and cannot really predict long-term
stability at all. For this, further, more detailed calculations
are required.

The analysis given becomes more interesting if we
examine figure 2.7(c) where the results of different initial wave 
numbers for zero surface tension are given,
with the same set of values of $n_0$.  The  fastest  growing 
curves are those with  highest  wave  numbers
as the coefficient of the $B$ term in (2.8.20) is now
initially negative for $n_0>1$.  For  
the largest wave  number, provided that the results still
hold close to $\tau=1$, the amplitude of the  surface  perturbation
may be sufficient to induce bubble break-up. This indicates that 
the results given in figure 2.3, where the Green's method  is  applied 
to a perturbed bubble are what would happen in reality
with zero surface tension.
However if we look back at figure 2.7(b) for a one
inch bubble with a short wave length perturbation, and thus 
correspondingly high curvatures, surface tension induces large
restoring forces to produce
small amplitude oscillations. Thus on the basis of this  analysis,
bearing in mind the above caveat, one may expect such a bubble
to be stable to small surface perturbations.

To conclude, we mention that for
a one inch bubble, where according to Baker and Moore (1989) surface tension
has little effect on the unperturbed motion,  figures  2.7(b)  and 
(c) show that this small amount of  surface  tension  affects  the 
perturbed motion tremendously.
The initial growth of the amplitude is unchecked in the case of  zero 
surface tension with the result that such bubbles are unstable.
