\vbox{
\c{\bigrmb Chapter 1.}
\vskip 1cm
\c{\bigrmb INTRODUCTION.}
\vskip 15pt
\hbox{\bf 1.1 Aims.}
\vskip 15pt
}
Bubbles present us with a startling variety of
fluid dynamical problems. Although many aspects of
bubble dynamics can be examined analytically, it is only
recently, with the advent of sophisticated numerical methods
and high speed computers, that we have been able to inspect in
detail complicated unsteady moving boundary problems. 
The highly non-linear
nature of the equations of motion make futile most attempts to find
neat
mathematical solutions in closed form. Having said this, there are
methods --- based on a global conservation 
principle such as energy integrals or virial theorems 
(Longuet-Higgins, 1983; Oguz and Prosperetti, 1990a)
or the Kelvin impulse (Blake, 1988) ---
that one can often use in order to obtain an idea of the
motions involved. 

Boundary integral methods rely on the
writing a partial differential equation, in this case
Laplace's equation, in terms of an integral equation. 
In the problems dealt with here, the
integration is carried out over the gas/liquid interface so that
the dimension of the problem is effectively reduced by one:
finite element and finite difference techniques
require discretisation of the whole flow domain. In many 
`three-dimensional' situations one can further simplify calculations
by assuming axisymmetry. 
In terms of computational effort, such an 
approach is thus relatively cheap.
For problems where the accurate determination of the
position of an interface is an
essential requirement,
boundary integral methods can be very effective.
Using such techniques, the motion of points
along the interfaces can be explicitly
tracked. Piecewise continuous elements are often employed
to interpolate between these nodes in order to calculate
the surface integrations. High order surface representations
such as cubic splines (Best and Kucera, 1992) parametrised by arc-length
can afford greater accuracy.

Boundary integral methods have been employed to solve a wide variety
of free boundary problems in fluid mechanics, including a number in
bubble dynamics.
One of the most successful applications is to the field of
cavitation bubble dynamics.
A Green's formula approach was used 
by Guerri, Lucca and Prosperetti (1981),
Blake, Taib and Doherty
(1986, 1987) and Taib (1985) to study cavitation bubbles
near boundaries.  
Blake and Kucera (1988) tackled
the problem of water-coning, a phenomenon occurring in oil reservoirs,
using a steady boundary integral technique. Longuet-Higgins
and Cokelet (1976) used a boundary integral scheme in conjunction
with a conformal map to study the 
overturning of large amplitude surface waves.

The aim of this thesis is to present various aspects of
bubble dynamics from a largely numerical viewpoint with the
aid of boundary integral methods. 
The first chapters deal exclusively with the
modelling, using a variety of boundary
integral techniques, of two-dimensional bubbles
both rising in an infinite fluid and then near to a free-surface. 
Although two-dimensional bubbles are unrealistic,
they have similar properties to `real' bubbles. This
work also provides an introduction to the most commonly used 
boundary integral techniques for problems in potential flow theory,
paving the way for the consideration of axisymmetric problems
in later chapters.
\vskip 15pt
\hbox{\bf 1.2 Motivation.}
\vskip 5pt
The long-term aims of the project are to study and gain an
understanding
of the physics involved in mammalian cell damage in aerated 
bioreactors. 
The production of mammalian cells in culture vessels is important 
for several reasons. The biochemicals produced by cells such as 
monoclonal antibodies, hormones and enzymes have many uses in medicine.
The cells themselves may be used in future for transplantation
and diagnostic purposes.

Bioreactor induced cell damage has been studied experimentally 
for several years by many researchers: 
indeed it was a team working in this field in the School
of Chemical Engineering at the University of Birmingham,
who suggested this mathematical study. Through the data collected,
a number of possible damage mechanisms have 
been proposed. It has been claimed
(e.g. Tramper et al, 1986) that the high agitation rates,  
used to circulate nutrients through the
medium, are detrimental to cells. 
However, many now believe an overwhelming proportion of
the damage observed to be
due to the cells' interactions with bubbles sparged
through the medium to provide oxygen to the cell culture.

Experiments have often involved actual, scaled down
bioreactors (figure 1.1, below), but there have been some where cells have 
been placed in controlled hydrodynamic environments. McQueen and Bailey (1989)
pumped cells in a medium through capillary tubes which had either smooth
or sudden constrictions to produce laminar or turbulent flows respectively.
They found that 
the viscosity of the fluid did not effect the cell death rates, 
even though the associated Kolmogorov eddy lengths changed.  By using 
different
sized tubes with viscosities adjusted so that the wall shear stresses
remained constant, they concluded that there are other factors affecting
cell death as well as just wall shear stresses. 
However, both they and
Kunas and Papoutsakis (1989) who used an agitated bioreactor, found 
that an increase in serum concentration decreased the death rate, these
latter authors claiming the change to be due to some
physical factor other than viscosity.
\vskip 10pt
\input fig11
\vskip 10pt
Using simple bubble columns, with no agitation, 
Handa, Emery and Spier (1987) and Handa (1986) showed a direct relationship
between the presence of bubbles and cell death rate. Death rates, per bubble, 
were found to be higher for small bubbles than for large ones.
It was also found that Pluronic (a co-polymer of propylene oxide
and ethylene oxide; a product of BASF, UK) which allows a stable, slowly 
draining foam, with the bursting of bubbles mainly restricted to its upper
surface, resulted in lower death rates. They claimed that the presence of an
anti-foam causes  cell damage as a consequence of shock waves produced by 
many bubbles bursting at the free-surface, and that without anti-foam,
rapid draining in unstable foams causes high shear stresses that can 
cause damage. By showing cell
death to be independent of the air sparger position and hence bubble
residence time,
Kioukia (1990) was able to conclude that, at least for the bubble 
column case, 
this damage occurs in a region near to the free-surface.

Al-Rubeai, Chalder and Emery (1990) used laser flow cytometry to study 
the morphology of damaged cells. They discovered that damage 
caused by agitation alone was less severe than that occurring when air bubbles
were being sparged through in which case cells were seen to have been torn 
apart.
In another recent paper, Oh, Nienow, Al-Rubeai and
Emery (1989), it was shown that, under solely surface aeration, cell growth
and viability does not depend on hydrodynamic conditions, even at very high
agitation rates. However with sparging, the growth rates become lower and 
reduce as agitation intensity increases.

A possible cause of damage in stirred reactors
was suggested by Yang et al (1990). They claimed
that a bubble stretched until breakup in the trailing 
vortex near to the tip of the agitator impeller 
blade will release the built-up surface energy in a 
potentially damaging shock wave as it bursts
into many smaller bubbles. Recent experiments by
Oh, Nienow, Al-Rubeai and Emery (1992) suggest that
cell damage is increased by positioning the air sparger beneath the
impeller. The effects of this are however two-fold as it will 
increase the
number of bubble/impeller interactions but will also split large bubbles
into smaller ones which are known to be more lethal at the free-surface.

In the experiments of Kunas and Papoutsakis (1990),
bubbles are seen to be entrained at the gas/liquid interface of a vortex
produced in an agitated reactor. The death rate was found to be directly
related to the size of the surface indentation, namely less damage was 
recorded if the reactor volume was increased or if the head-space above it was 
removed altogether. In this latter case cell
growth, though slower, occurred even at high agitation rates. If, in addition,
tiny bubbles, moving around at high speeds were present,
the death rates were still very small compared to the situation where there 
was a free-surface. 
Significant agitation damage was only found 
when Kolmogorov eddy lengths became comparable to the dimensions of the cells.

It has been speculated (Chalmers and Bavarian, 1991) that the
high acceleration at the rim of a rupturing film,
such as occurs when a bubble bursts, is a possible cause
of cell damage. Along similar lines, Kowalski (1991) by concentrating on
the bursting of a single, large bubble manufactured by placing a pipette 
into a cell culture medium, claimed that cells trapped in the rupturing
film may be killed by large compressional forces which occur when the 
retreating rim breaks into threads which in turn split into droplets.
The walls of the experimental vessel were found to be littered with cell
debris, indicating a secondary effect of the burst as a means of
removing cells from the reactor. He also found that Pluronic allowed
the cells to drain away with the fluid in the film,
and thus escape the regions where high stresses
may be experienced.

It is therefore apparent, given the above overview of just some of the
recent literature, that there is no complete agreement as to the precise 
mechanisms responsible, but there is a growing inclination to view the role
of
bubbles and their interaction with the free-surface as the prime cause of 
cell damage in aerated bioreactors.
\vskip 15pt
\hbox{\bf 1.3 Summary.}
\vskip 5pt
In Chapter 2, an inviscid model of a two-dimensional bubble
is introduced. Boundary integral techniques, 
which have had much success previously in bubble dynamics, 
are then employed to solve for the
bubble motion. The results obtained are almost identical to those of Baker
and Moore (1989) and close to 
the experimental findings of Walters and Davidson (1962). Several different
boundary integral techniques are employed and compared.

In Chapter 3, the theory is extended to look at the problem of 
a two-dimensional bubble as it 
approaches a free-surface. A conformal map technique, similar to that of
Longuet-Higgins and Cokelet (1976), is used to enable integration over the
infinite free-surface. This is compared against an 
asymptotic solution valid for bubbles at large depths.

In Chapter 4, the motion of axisymmetric gas bubbles
is examined via a boundary integral method.
Two example problems are examined. The first, as in
Chapter 2, looks at the case
of bubbles rising from rest in an infinite fluid. This time we allow
for bubble-bubble interaction by including a second bubble into the
calculations.
The second problem uses an inviscid boundary integral together
with modified boundary conditions to include some effects
of viscosity and thus obtain an approximate solution to the 
steady state of a rising bubble when the drag and buoyancy
are in balance.

The thesis culminates in Chapter 5, with the development
of a model for bubbles bursting 
at a free-surface with a view to developing a greater understanding
of cell damage in bubble aerated bioreactors.
A method, based on that of Lundgren and Mansour (1988),
for including viscous forces via a boundary layer is developed.
This allows an approximation of the shedding of vorticity
as the boundary layer separates when a high speed liquid jet
rises above the fluid.

Finally, in Chapter 6, we summarise the important results and conclusions and
indicate briefly possible directions for
future research into the cell damage problem.

